\documentclass[boxes,pages,color=SeaGreen]{homework}

\hypersetup{
    colorlinks=true,
    urlcolor=SeaGreen!60!black,
    linkcolor=Bittersweet
}
\newcommand{\collab}[1]{\footnote{\href{mailto:#1}{\texttt{#1}}}}

\name{Nate Stemen}
\studentid{20906566}
\email{nate@stemen.email}
\term{Fall 2021}
\course{Theory of Quantum Information}
\courseid{QIC 820}
\hwnum{4}
\duedate{Dec 17, 2021}

\hwname{Assignment}

\usepackage{physics}
\usepackage{macros}
\usepackage{cleveref}
\usepackage{relsize}

%-----------------------------------------------------------------------------%
% Macros
%-----------------------------------------------------------------------------%

\newcommand{\tinyspace}{\mspace{1mu}}
\renewcommand{\op}[1]{\operatorname{#1}}

\newcommand{\X}{\mathcal{X}}
\newcommand{\Y}{\mathcal{Y}}
\newcommand{\Z}{\mathcal{Z}}
\newcommand{\W}{\mathcal{W}}
\newcommand{\V}{\mathcal{V}}
\newcommand{\U}{\mathcal{U}}
\newcommand{\I}{\mathbb{1}}
\renewcommand{\P}{\mathcal{P}}
\newcommand{\reg}[1]{\mathsf{#1}}
\newcommand{\smalltag}[1]{\tag*{\footnotesize (#1)}}

\newcommand{\Lin}{\mathrm{L}}
\newcommand{\Pos}{\mathrm{Pos}}
\newcommand{\Unitary}{\mathrm{U}}
\newcommand{\Density}{\mathrm{D}}
\newcommand{\Channel}{\mathrm{C}}
\newcommand{\Sep}{\mathrm{Sep}}
\newcommand{\SepD}{\mathrm{SepD}}
\newcommand{\PPT}{\mathrm{PPT}}
\DeclareMathOperator{\vect}{vec}
\DeclareMathOperator{\cone}{cone}


\begin{document}

%-----------------------------------------------------------------------------%

\begin{problem}
Let $\Phi \in \Channel(\X, \Y)$ be a channel, for complex Euclidean spaces
$\X$ and $\Y$.
Prove that the following three statements are equivalent:
\begin{parts}
  \part
  For every complex Euclidean space $\Z$ and every state
  $\rho\in\Density(\X\otimes\Z)$, we have
  \[
    \qty( \Phi \otimes \I_{\mathsmaller{\Lin(\Z)}})(\rho)
    \in \SepD(\Y \mathbin{:} \Z).
  \]\label{part:1a}
  \part
  $J(\Phi) \in \Sep(\Y \mathbin{:} \X)$.\label{part:1b}
  \part
  There exists an alphabet $\Sigma$, a measurement
  $\mu: \Sigma \rightarrow \Pos(\X)$, and a collection of states
  $\qty{\sigma_a : a \in \Sigma } \subseteq \Density(\Y)$ such that
  \[
    \Phi(X) = \sum_{a\in\Sigma} \ip{\mu(a)}{X} \sigma_a
  \]
  for all $X \in \Lin(\X)$.\label{part:1c}
\end{parts}
\end{problem}

\begin{solution}
  First we show \ref{part:1a} $\implies$ \ref{part:1b}.
  Since \ref{part:1a} holds for all $\Z$ and $\rho\in\Density(\X\otimes\Z)$, we can take $\Z = \X$, and $\rho\in\Density(\X\otimes\X)$ to be equal to $\vect(\1_\X)\vect(\1_\X)^* / \dim(\X)$.
  Thus we have
  \begin{equation*}
    (\Phi \otimes \1_{\mathsmaller{\Lin(\X)}})(\vect(\1_X)\vect(\1_\X)^*/\dim(\X))\in\SepD(\Y : \X)
  \end{equation*}
  and by the linearity of $\Phi\otimes\1_\mathsmaller{\Lin(\X)}$ we can pull $\frac{1}{\dim(\X)}$ out.
  Using the fact that $\cone(\SepD(\W)) = \Sep(\W)$, we can move the (positive) constant to the other side to get
  \begin{equation*}
    (\Phi \otimes \1_{\mathsmaller{\Lin(\X)}})(\vect(\1_X)\vect(\1_\X)^*) \eqdef J(\Phi) \in \Sep(\Y : \X).
  \end{equation*}

  We now prove \ref{part:1b} $\implies$ \ref{part:1c}.
  Recall from Equation 2.66 we have a channel can be written using the Choi representation as $\Phi(X) = \tr_\X(J(\Phi)\,\1_\Y\otimes X^\intercal)$.
  Since $J(\Phi)$ is separable, we can write it as
  \begin{equation*}
    J(\Phi) = \sum_{a\in\Sigma}A_a\otimes B_a
  \end{equation*}
  with $\Sigma$ being an alphabet, $A_a\in\Pos(\Y)$, and $B_a\in\Pos(\X)$ for all $a\in\Sigma$.
  Using Equation 2.66 we have a channel can be written using the Choi representation as $\Phi(X) = \tr_\X(J(\Phi)\,\1_\Y\otimes X^\intercal)$ and plugging in a separable Choi representation we can simplify.
  \begin{align*}
    \Phi(X) & = \tr_\X\qty(\sum_{a\in\Sigma}A_a\otimes B_a [\1_\Y\otimes X^\intercal])        \\
            & = \tr_\X\qty(\sum_{a\in\Sigma}A_a \otimes B_a X^\intercal)                      \\
            & = \sum_{a\in\Sigma}A_a\tr(B_a X^\intercal)                                      \\
            & = \sum_{a\in\Sigma}\frac{A_a}{\tr(A_a)}\tr(\tr(A_a)B_a X^\intercal)             \\
            & = \sum_{a\in\Sigma}\frac{A_a}{\tr(A_a)}\ip{\overline{\tr(A_a)B_a}}{X^\intercal}
  \end{align*}
  Defining $\sigma_a \defeq \frac{A_a}{\tr(A_a)}\in\Density(\Y)$ along with $\mu(a) \defeq \overline{\tr(A_a)B_a} \in \Pos(\X)$ allows us to express $\Phi$ as
  \begin{equation*}
    \Phi(X) = \sum_{a\in\Sigma}\ip{\mu(a)}{X}\sigma_a.
  \end{equation*}
  All that's left to show is that $\mu:\Sigma\to\Pos(\X)$ is indeed a measurement (that is the completeness relation holds: $\sum_{a\in\Sigma}\mu(a) = \1_\X$).
  To do this we will use the fact that $\tr_\Y(J(\Phi)) = \1_\X$ for all channels $\Phi$.
  Since the Choi representation is separable this gives us a condition on the $A_a$'s and $B_a$'s.
  \begin{equation*}
    \tr_\Y(J(\Phi)) = \sum_{a\in\Sigma}\tr_\Y(A_a\otimes B_a) = \sum_{a\in\Sigma}\tr(A_a)B_a = \1_\X
  \end{equation*}
  Summing up the measurement operators we then have
  \begin{align*}
    \sum_{a\in\Sigma}\mu(a) = \sum_{a\in\Sigma}\overline{\tr(A_a)B_a} = \overline{\sum_{a\in\Sigma}\tr(A_a)B_a} = \overline{\1_\X} = \1_\X.
  \end{align*}
  Thus $\mu$ is indeed a measurement.


  Finally, we show \ref{part:1c} $\implies$ \ref{part:1a} to complete the equivalences.
\end{solution}

\begin{problem}
For any channel $\Xi$, define the \emph{minimum output entropy} of $\Xi$ as
\[
  \op{H}_{\text{min}}(\Xi) = \min_{\rho} \tinyspace \op{H}(\Xi(\rho)),
\]
where the minimum is over all density operator inputs to $\Xi$.

Suppose that $\Phi\in\Channel(\X, \Y)$ and $\Psi\in\Channel(\Z, \W)$ are
channels, for complex Euclidean spaces $\X$, $\Y$, $\Z$, and $\W$, and assume
that $J(\Phi)\in\Sep(\Y \mathbin{:} \X)$.
Prove that
\[
  \op{H}_{\text{min}}(\Phi\otimes\Psi) =
  \op{H}_{\text{min}}(\Phi) + \op{H}_{\textup{min}}(\Psi).
\]

Remark: the equality is not true in general without the assumption
$J(\Phi)\in\Sep(\Y \mathbin{:} \X)$; a correct answer must make use of this
assumption.
\end{problem}

\noindent Solution completed in collaboration with Alev Orfi,\collab{akborfi@uwaterloo.ca} and Muhammad Usman Farooq.\collab{mu7faroo@uwaterloo.ca}

{\noindent\color{SeaGreen!30}\rule{\textwidth}{1.5pt}}

\begin{solution}
\end{solution}

\begin{problem}
Let $\X$ be a complex Euclidean space, let $n = \dim(\X)$, and let
$\Phi\in\Channel(\X)$ be a unital channel.
Following our usual convention for singular-value decompositions, let
$s_1(Y) \geq \cdots \geq s_n(Y)$ denote the singular values of a given
operator $Y\in\Lin(\X)$, ordered from largest to smallest, and taking
$s_k(Y) = 0$ when $k > \rank(Y)$.
Prove that, for every operator $X\in\Lin(\X)$, we have
\[
  s_1(X) + \cdots + s_m(X) \geq s_1(\Phi(X)) + \cdots + s_m(\Phi(X))
\]
for every $m \in \{1,\ldots,n\}$.

Hint: thinking about the block operator
\[
  \mqty(
  0   & X \\
  X^* & 0
  )
  = \ketbra{0}{1} \otimes X + \ketbra{1}{0} \otimes X^*
\]
may be helpful when solving this problem.
\end{problem}

\noindent Solution completed in collaboration with Mohammad Ayyash,\collab{mmayyash@uwaterloo.ca} and Nicholas Zutt.\collab{nzutt@uwaterloo.ca}

{\noindent\color{SeaGreen!30}\rule{\textwidth}{1.5pt}}

\begin{solution}
\end{solution}

\begin{problem}
Let $\Sigma$ be an alphabet, let $\X$ and $\Y$ be complex Euclidean spaces of
the form $\X = \C^{\Sigma}$ and $\Y = \C^{\Sigma}$, define the
swap operator
\[
  W = \sum_{a,b\in\Sigma} \ketbra{a}{b} \otimes \ketbra{b}{a},
\]
which we may regard as a unitary operator $W\in\Unitary(\X\otimes\Y)$,
define projections
\[
  \Pi_0 = \frac{\I_{\X} \otimes \I_{\Y} + W}{2}
  \quad\text{and}\quad
  \Pi_1 = \frac{\I_{\X} \otimes \I_{\Y} - W}{2},
\]
and define
\[
  \rho_0 = \frac{\Pi_0}{\binom{n+1}{2}}
  \quad\text{and}\quad
  \rho_1 = \frac{\Pi_1}{\binom{n}{2}},
\]
for $n = \abs{\Sigma}$.
These are the symmetric and anti-symmetric Werner states that were discussed
a few times, such as in Lecture 3.
(See also Example 6.10 in the text.)

Prove that if $\mu: \{0,1\} \rightarrow \Pos(\X\otimes\Y)$ is a measurement with
$\mu(0), \mu(1) \in \PPT(\X \mathbin{:} \Y)$ (i.e., $\mu$ is a
\emph{PPT measurement}), then
\[
  \frac{1}{2}\ip{\mu(0)}{\rho_0} + \frac{1}{2}\ip{\mu(1)}{\rho_1}
  \leq \frac{1}{2} + \frac{1}{n+1}.
\]
\end{problem}

\begin{solution}
  Let's first simplify the terms on the left hand side as much as possible.
  \begin{align*}
    \ip{\mu(1)}{\rho_1} & = \frac{1}{\binom{n}{2}}\tr(\mu(1)^*\Pi_1)                                     \\
                        & = \frac{1}{n(n-1)}\tr(\mu(1)(\1_\X\otimes\1_\Y - W))                           \\
    \ip{\mu(0)}{\rho_0} & = \frac{1}{\binom{n+1}{2}}\tr(\mu(0)^*\Pi_0)                                   \\
                        & = \frac{1}{n(n+1)}\tr(\mu(0)(\1_\X\otimes\1_\Y + W))                           \\
                        & = \frac{1}{n(n+1)}\tr(\qty[\1_\X\otimes\1_\Y - \mu(1)](\1_\X\otimes\1_\Y + W)) \\
                        & = \frac{1}{n(n+1)}\tr(\1_\X\otimes\1_\Y + W - \mu(1) - \mu(1)W)                \\
                        & = \frac{1}{n(n+1)}\qty(n^2 + n - \tr(\mu(1) - \mu(1)W))                        \\
                        & = 1 - \frac{1}{n(n+1)}\tr(\mu(1)(\1_\X\otimes\1_\Y + W))
  \end{align*}
  Putting these together, and doing some algebra we have the following equality.
  \begin{equation*}
    \ip{\mu(0)}{\rho_0} + \ip{\mu(1)}{\rho_1} = 1 + \frac{2}{n(n^2 - 1)}\qty\Big(\tr(\mu(1)) - n\tr(\mu(1)W))
  \end{equation*}
  Proving the desired inequality is now equivalent to showing $\tr(\mu(1)) - n\tr(\mu(1)W) \leq n(n - 1)$.
  Substituting the second occurrence of $\mu(1)$ for $\1_\X\otimes\1_\Y - \mu(0)$ we obtain $\tr(\mu(1)) - n^2 + n\tr(\mu(0)W)$ for the left hand side.
  Applying H\"older's inequality to $\ip{\mu(0)}{W}$ with $p = 1$ and $p^* = \infty$ we see $\ip{\mu(0)}{W} = \tr(\mu(0)W) \leq \norm{\mu(0)}_1\norm{W}_\infty$.
  The swap operator has infinity norm equal to $1$ (almost by definition), and $\norm{\mu(0)}_1 = \tr(\sqrt{(\mu(0)^*\mu(0))}) = \tr(\mu(0))$.
  In total we then have $\tr(\mu(0)W) \leq \tr(\mu(0))$.
  Applying this to our desired inequality we have
  \begin{align*}
    \tr(\mu(1)) - n^2 + n\tr(\mu(0)W) & = \tr(\mu(1)) - n^2 + (n-1)\tr(\mu(0)W) + \tr(\mu(0)W)   \\
                                      & \leq \tr(\mu(1)) - n^2 + (n-1)\tr(\mu(0)W) + \tr(\mu(0)) \\
                                      & = \tr(\mu(0) + \mu(1)) - n^2 + (n-1)\tr(\mu(0)W)         \\
                                      & = (n-1)\tr(\mu(0)W).
  \end{align*}
  Thus, in final we must show $\tr(\mu(0)W) \leq n$, or equivalently (by the completeness relation) $\tr(\mu(0)W) = \tr(W - \mu(1)W) = n - \tr(\mu(1)W)$ that $\tr(\mu(1)W) \geq 0$.
  \begin{align*}
    \tr(\mu(1)W) & = \ip{\mu(1)}{W}                                                 \\
                 & = \ip{(\mathsf{T}\otimes\1_\Y)\mu(1)}{(\mathsf{T}\otimes\1_\Y)W} \\
                 & = \ip{A}{\sum_{a,b\in\Sigma}E_{a,b}\otimes E_{a,b}}
  \end{align*}
  Since $\mu(1) \in \PPT(\X:\Y)$, we know $(\mathsf{T}\otimes\1_\Y)\mu(1)\in\Pos(\X\otimes\Y)$, and additionally it's not hard to show $\sum_{a,b\in\Sigma}E_{a,b}\otimes E_{a,b}\in\Pos(\X\otimes\Y)$.
  Since we have the inner product of positive semidefinite operators, it must be a nonnegative real number.
  Thus $\tr(\mu(1)W) \geq 0$, and the proof is complete.
\end{solution}

\end{document}
