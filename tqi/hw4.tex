\documentclass[boxes,pages,color=SeaGreen]{homework}

\hypersetup{
    colorlinks=true,
    urlcolor=SeaGreen!60!black,
    linkcolor=Bittersweet
}
\newcommand{\collab}[1]{\footnote{\href{mailto:#1}{\texttt{#1}}}}

\name{Nate Stemen}
\studentid{20906566}
\email{nate@stemen.email}
\term{Fall 2021}
\course{Theory of Quantum Information}
\courseid{QIC 820}
\hwnum{4}
\duedate{Dec 17, 2021}

\hwname{Assignment}

\usepackage{physics}
\usepackage{macros}
\usepackage{cleveref}
\usepackage{relsize}

%-----------------------------------------------------------------------------%
% Macros
%-----------------------------------------------------------------------------%

\newcommand{\tinyspace}{\mspace{1mu}}
\renewcommand{\op}[1]{\operatorname{#1}}

\newcommand{\X}{\mathcal{X}}
\newcommand{\Y}{\mathcal{Y}}
\newcommand{\Z}{\mathcal{Z}}
\newcommand{\W}{\mathcal{W}}
\newcommand{\V}{\mathcal{V}}
\newcommand{\U}{\mathcal{U}}
\newcommand{\I}{\mathbb{1}}
\renewcommand{\P}{\mathcal{P}}
\newcommand{\reg}[1]{\mathsf{#1}}
\newcommand{\smalltag}[1]{\tag*{\footnotesize (#1)}}

\newcommand{\Lin}{\mathrm{L}}
\newcommand{\Pos}{\mathrm{Pos}}
\newcommand{\Unitary}{\mathrm{U}}
\newcommand{\Density}{\mathrm{D}}
\newcommand{\Channel}{\mathrm{C}}
\newcommand{\Sep}{\mathrm{Sep}}
\newcommand{\SepD}{\mathrm{SepD}}
\newcommand{\PPT}{\mathrm{PPT}}


\begin{document}

%-----------------------------------------------------------------------------%

\begin{problem}
Let $\Phi \in \Channel(\X, \Y)$ be a channel, for complex Euclidean spaces
$\X$ and $\Y$.
Prove that the following three statements are equivalent:
\begin{parts}
  \part
  For every complex Euclidean space $\Z$ and every state
  $\rho\in\Density(\X\otimes\Z)$, we have
  \[
    \qty( \Phi \otimes \I_{\mathsmaller{\Lin(\Z)}})(\rho)
    \in \SepD(\Y \mathbin{:} \Z).
  \]\label{part:1a}
  \part
  $J(\Phi) \in \Sep(\Y \mathbin{:} \X)$.\label{part:1b}
  \part
  There exists an alphabet $\Sigma$, a measurement
  $\mu: \Sigma \rightarrow \Pos(\X)$, and a collection of states
  $\qty{\sigma_a : a \in \Sigma } \subseteq \Density(\Y)$ such that
  \[
    \Phi(X) = \sum_{a\in\Sigma} \ip{\mu(a)}{X} \sigma_a
  \]
  for all $X \in \Lin(\X)$.\label{part:1c}
\end{parts}
\end{problem}

\begin{solution}
  \ref{part:1a}
  \ref{part:1b}
  \ref{part:1c}
\end{solution}

\begin{problem}
For any channel $\Xi$, define the \emph{minimum output entropy} of $\Xi$ as
\[
  \op{H}_{\text{min}}(\Xi) = \min_{\rho} \tinyspace \op{H}(\Xi(\rho)),
\]
where the minimum is over all density operator inputs to $\Xi$.

Suppose that $\Phi\in\Channel(\X, \Y)$ and $\Psi\in\Channel(\Z, \W)$ are
channels, for complex Euclidean spaces $\X$, $\Y$, $\Z$, and $\W$, and assume
that $J(\Phi)\in\Sep(\Y \mathbin{:} \X)$.
Prove that
\[
  \op{H}_{\text{min}}(\Phi\otimes\Psi) =
  \op{H}_{\text{min}}(\Phi) + \op{H}_{\textup{min}}(\Psi).
\]

Remark: the equality is not true in general without the assumption
$J(\Phi)\in\Sep(\Y \mathbin{:} \X)$; a correct answer must make use of this
assumption.
\end{problem}

\begin{solution}
\end{solution}

\begin{problem}
Let $\X$ be a complex Euclidean space, let $n = \dim(\X)$, and let
$\Phi\in\Channel(\X)$ be a unital channel.
Following our usual convention for singular-value decompositions, let
$s_1(Y) \geq \cdots \geq s_n(Y)$ denote the singular values of a given
operator $Y\in\Lin(\X)$, ordered from largest to smallest, and taking
$s_k(Y) = 0$ when $k > \rank(Y)$.
Prove that, for every operator $X\in\Lin(\X)$, we have
\[
  s_1(X) + \cdots + s_m(X) \geq s_1(\Phi(X)) + \cdots + s_m(\Phi(X))
\]
for every $m \in \{1,\ldots,n\}$.

Hint: thinking about the block operator
\[
  \mqty(
  0   & X \\
  X^* & 0
  )
  = \ketbra{0}{1} \otimes X + \ketbra{1}{0} \otimes X^*
\]
may be helpful when solving this problem.
\end{problem}

\begin{solution}
\end{solution}

\begin{problem}
Let $\Sigma$ be an alphabet, let $\X$ and $\Y$ be complex Euclidean spaces of
the form $\X = \C^{\Sigma}$ and $\Y = \C^{\Sigma}$, define the
swap operator
\[
  W = \sum_{a,b\in\Sigma} \ketbra{a}{b} \otimes \ketbra{b}{a},
\]
which we may regard as a unitary operator $W\in\Unitary(\X\otimes\Y)$,
define projections
\[
  \Pi_0 = \frac{\I_{\X} \otimes \I_{\Y} + W}{2}
  \quad\text{and}\quad
  \Pi_1 = \frac{\I_{\X} \otimes \I_{\Y} - W}{2},
\]
and define
\[
  \rho_0 = \frac{\Pi_0}{\binom{n+1}{2}}
  \quad\text{and}\quad
  \rho_1 = \frac{\Pi_1}{\binom{n}{2}},
\]
for $n = \abs{\Sigma}$.
These are the symmetric and anti-symmetric Werner states that were discussed
a few times, such as in Lecture 3.
(See also Example 6.10 in the text.)

Prove that if $\mu: \{0,1\} \rightarrow \Pos(\X\otimes\Y)$ is a measurement with
$\mu(0), \mu(1) \in \PPT(\X \mathbin{:} \Y)$ (i.e., $\mu$ is a
\emph{PPT measurement}), then
\[
  \frac{1}{2}\ip{\mu(0)}{\rho_0} + \frac{1}{2}\ip{\mu(1)}{\rho_1}
  \leq \frac{1}{2} + \frac{1}{n+1}.
\]
\end{problem}

\begin{solution}
\end{solution}

\end{document}
