\documentclass[boxes,pages,color=SeaGreen]{homework}

\hypersetup{
    colorlinks=true,
    urlcolor=SeaGreen!60!black,
    linkcolor=Bittersweet
}
\newcommand{\collab}[1]{\footnote{\href{mailto:#1}{\texttt{#1}}}}

\name{Nate Stemen}
\studentid{20906566}
\email{nate@stemen.email}
\term{Fall 2021}
\course{Theory of Quantum Information}
\courseid{QIC 820}
\hwnum{1}
\duedate{Oct 8, 2021}

\hwname{Assignment}

\usepackage{physics}
\usepackage{relsize}
\usepackage{macros}
\usepackage{cleveref}

%-----------------------------------------------------------------------------%
% Macros
%-----------------------------------------------------------------------------%

\newcommand{\tinyspace}{\mspace{1mu}}
\renewcommand{\vec}{\operatorname{vec}}
\newcommand{\I}{\mathbb{1}}
\newcommand{\reg}[1]{\mathsf{#1}}
\renewcommand{\t}{{\scriptscriptstyle\mathsf{T}}}

\newcommand{\X}{\mathcal{X}}
\newcommand{\Y}{\mathcal{Y}}
\newcommand{\Z}{\mathcal{Z}}
\newcommand{\W}{\mathcal{W}}
\newcommand{\V}{\mathcal{V}}
\newcommand{\U}{\mathcal{U}}
\renewcommand{\P}{\mathcal{P}}

\newcommand{\Lin}{\mathrm{L}}
\newcommand{\Trans}{\mathrm{T}}
\newcommand{\Pos}{\mathrm{Pos}}
\newcommand{\Herm}{\mathrm{Herm}}
\newcommand{\Channel}{\mathrm{C}}
\newcommand{\Unitary}{\mathrm{U}}
\newcommand{\Density}{\mathrm{D}}
\newcommand{\CP}{\mathrm{CP}}


\begin{document}

All references to theorem/propositions/etc. are references to Watrous's book.

\begin{problem}
Let $\X$ and $\Y$ be complex Euclidean spaces and let
$\Phi\in\CP(\X,\Y)$ be a completely positive map.
Prove that there exists an operator $B\in\Lin(\X\otimes\Z,\Y)$, for some
choice of a complex Euclidean space $\Z$, such that
\[
    \Phi(X) = B (X \otimes \I_{\Z}) B^{\ast}
\]
for all $X\in\Lin(\X)$.
Identify a condition on the operator $B$ that is equivalent to $\Phi$
preserving trace.
\end{problem}

\begin{solution}
    By Proposition 2.18 we know that $\Phi^*$ is also a completely positive map $\Phi^*\in\CP(\Y, \X)$.
    Now, using the Stinespring representation of $\Phi^*$ and Theorem 2.22, there exists an $A\in \Lin(\Y, \X\otimes\Z)$ such that
    \begin{equation}\label{eq:adjoint}
        \Phi^*(Y) = \tr_\Z(AYA^*).
    \end{equation}
    Using the fact that $(\Phi^*)^* = \Phi$, and Equation 2.72 which gives us the dual representation of a Strinespring map we have
    \begin{equation*}
        \Phi(X) = \Phi^{**}(X) = A^*(X\otimes\I_\Z)A.
    \end{equation*}
    Finally, taking $B = A^*$ we see first $B\in\Lin(\X\otimes\Z, \Y)$ and indeed
    \begin{equation*}
        \Phi(X) = B (X \otimes \I_{\Z}) B^*.
    \end{equation*}
    {\color{SeaGreen!30}\rule{\textwidth}{1.5pt}}

    \noindent
    To derive a condition on $B$ such that $\Phi$ preserves trace we set $\tr(X) = \tr(\Phi(X))$.
    \begin{align*}
        \tr(X) = \tr(\Phi(X)) & = \tr(B(X\otimes\I_\Z)B^*)            \\
                              & = \tr_\X(\tr_\Z(B^*B(X\otimes\I_\Z))) \\
                              & = \tr(\tr_\Z(B^*B)X)
    \end{align*}
    Thus, in order for this mapping to preserve trace we must have $\tr_\Z(B^*B) = \I_\X$.

    Perhaps a simpler way of seeing this is using the fact that $\Phi$ preserving trace is equivalent to $\Phi^*$ being a unital map.
    Thus using \cref{eq:adjoint} we have
    \begin{equation*}
        \Phi^*(\1_\Y) = \tr_\Z(B^*\1_\Y B) = \tr_\Z(B^* B) = \1_\X.
    \end{equation*}
\end{solution}

%-----------------------------------------------------------------------------%

\begin{problem}
Let $\X$ and $\Y$ be complex Euclidean spaces, let $\Phi\in\Trans(\X,\Y)$
be a positive (but not necessarily completely positive) map, and let
$\Delta\in\Channel(\Y)$ denote the completely dephasing channel with
respect to the space $\Y$.
Prove that $\Delta\Phi$ is completely positive.
\end{problem}

\begin{solution}
    Let $\X = \C^\Sigma$ and $\Y = \C^\Gamma$ with $E_{a,b}$ being the standard basis for $\Lin(\X)$ and $\tilde{E}_{a,b}$ be the standard basis for $\Lin(\Y)$.
    First, the Choi representation of $\Phi$ can be expanded:
    \begin{align*}
        J(\Phi) & = \sum_{a,b\in\Sigma}\Phi(E_{a,b})\otimes E_{a,b}                                        \\
                & = \sum_{a,b\in\Sigma}\qty[\sum_{c,d\in\Gamma}\alpha_{c,d}\tilde{E}_{c,d}]\otimes E_{a,b} \\
                & = \sum_{\substack{a,b\in\Sigma                                                           \\ c,d\in\Gamma}}\alpha_{c,d}\tilde{E}_{c,d}\otimes E_{a,b}
    \end{align*}
    We can now use the fact that $J(\Delta\Phi) = \qty[\Delta\otimes\1_\X]J(\Phi)$ to write the Choi representation of $J(\Delta\Phi)$.
    \begin{align*}
        J(\Delta\Phi) & = \qty[\Delta\otimes\1_\X]J(\Phi) \\
                      & = \sum_{\substack{a,b\in\Sigma    \\ c,d\in\Gamma}}\alpha_{c,d}\Delta(\tilde{E}_{c,d})\otimes E_{a,b} \\
                      & = \sum_{\substack{a,b\in\Sigma    \\ c\in\Gamma}}\alpha_{c,c}\tilde{E}_{c,c}\otimes E_{a,b}
    \end{align*}
    With this we now aim to use Theorem 2.22 to prove $\Delta\Phi$ is completely positive by showing $J(\Delta\Phi)\in\Pos(\Y\otimes\X)$.
    First we derive a condition on the $\alpha_{c,d}$ coefficients using the fact that $\Phi$ is a positive map.
    Let $\qty{\ket{c}}_{c\in\Gamma}$ be a basis for $\Y$, and since $E_{a,b}$ is a positive operator, so should $\Phi(E_{a,b})$.
    \begin{equation*}
        \ev{\Phi(E_{a,b})}{\tilde{c}} = \ev{\sum_{c,d\in\Gamma}\alpha_{c,d}\tilde{E}_{c,d}}{\tilde{c}} = \sum_{c,d\in\Gamma}\alpha_{c,d}\braket{\tilde{c}}{c}\braket{d}{\tilde{c}} = \sum_{c,d\in\Gamma}\alpha_{c,d}\delta_{\tilde{c},c}\delta_{d,\tilde{c}} = \alpha_{\tilde{c},\tilde{c}}
    \end{equation*}
    Thus for all $c\in\Gamma$ we have $\alpha_{c,c} \geq 0$.

    Now take $\qty{\ket{c}\otimes\ket{a}}_{\substack{a\in\Sigma \\ c\in\Gamma}}$ to be a basis for $\Y\otimes\X$.
    We can now check if $J(\Delta\Phi)\in\Pos(\Y\otimes\X)$ by showing $\bra{c}\otimes\bra{a}J(\Delta\Phi)\ket{c}\otimes\ket{a}\geq 0$.
    \begin{align*}
        \bra{\tilde{c}}\otimes\bra{\tilde{a}}J(\Delta\Phi)\ket{\tilde{c}}\otimes\ket{\tilde{a}} & = \sum_{\substack{a,b\in\Sigma \\ c\in\Gamma}}\alpha_{c,c}\bra{\tilde{c}}\otimes\bra{\tilde{a}}\, \tilde{E}_{c,c}\otimes E_{a,b}\,\ket{\tilde{c}}\otimes\ket{\tilde{a}} \\
                                                                                                & = \sum_{\substack{a,b\in\Sigma \\ c\in\Gamma}}\alpha_{c,c}\braket{\tilde{c}}{c}\braket{c}{\tilde{c}}\braket{\tilde{a}}{a}\braket{b}{\tilde{a}} \\
                                                                                                & = \alpha_{\tilde{c},\tilde{c}}
    \end{align*}
    Since we've already shown $\alpha_{c,c} \geq 0$ for all $c\in\Gamma$, this shows that $J(\Delta\Phi)$ is positive semidefinite by extending this argument to linear combinations of the basis elements.
\end{solution}

%-----------------------------------------------------------------------------%

\begin{problem}
Let $\X$ and $\Y$ be complex Euclidean spaces and let $\Sigma$ be an
alphabet.
Suppose further that $\eta:\Sigma\rightarrow\Pos(\X)$ is a function
satisfying
\[
    \sum_{a\in\Sigma} \eta(a) \in \Density(\X),
\]
which is to say that $\eta$ represents an \emph{ensemble of states},
and $u\in\X\otimes\Y$ is a purification of the average state of this
ensemble:
\[
    \tr_{\Y}( u u^{\ast} ) = \sum_{a\in\Sigma} \eta(a).
\]
Prove that there exists a measurement $\mu:\Sigma\rightarrow\Pos(\Y)$ for
which
\[
    \eta(a) = \tr_{\Y} \qty\big[(\I_{\X}\otimes\mu(a))u u^{\ast}]
\]
for all $a\in\Sigma$.
\end{problem}
\noindent Solution completed in collaboration with Alev Orfi,\collab{akborfi@uwaterloo.ca} and Muhammad Usman Farooq.\collab{mu7faroo@uwaterloo.ca}

{\noindent\color{SeaGreen!30}\rule{\textwidth}{1.5pt}}

\begin{solution}
    The fact that $\vec:\Lin(\X,\Y)\to\Y\otimes\X$ is a bijection allows us to find an operator $A\in\Lin(\Y, \X)$ such that $\vec(A) = u$.
    We thus have
    \begin{equation*}
        \tr_\Y(uu^*) = \sum_{a\in\Sigma}\eta(a) = AA^*
    \end{equation*}
    using $\tr_\Y(\vec(C)\vec(B)^*) = CB^*$ for $C, B\in\Lin(\Y, \X)$.
    Combining Exercise 3 which says $\im(P)\subseteq\im(P + Q)$ for $P, Q\in\Pos(\X)$ and the fact that $\eta(A)\in\Pos(\X)$, we can write $\im(\eta(a))\subseteq\im(\sum\eta(a)) = \im(AA^*)\subseteq\im(A)$.
    Now, we need a lemma---Lemma 2.30 from Watrous.
    \begin{lemma}
        For $D\in\Lin(\Y, \X)$ we have the following equality of sets.
        \begin{equation*}
            \qty{P\in\Pos(\X) : \im(P)\subseteq\im(D)} = \qty{DQD^* : Q\in\Pos(\X)}
        \end{equation*}
    \end{lemma}
    Thus since $\eta(a)\in\Pos(\X)$ and $\im(\eta(a)) \subseteq \im(A)$, we can find a $Q_a\in\Pos(\X)$ such that $\eta(a) = AQ_aA^*$.
    In particular since the transpose is a positive map $Q_a^\intercal$ is also positive and we take as definition $\mu(a) \defeq Q_a^\intercal$.

    We now show $\mu: \Sigma \to \Pos(\X)$ is indeed a measurement.
    We've already shown $\mu(a)$ to be a positive semi-definite operator on $\X$, so we need to show $\mu$ resolves the identity.
    \begin{equation*}
        \sum_{a\in\Sigma}\eta(a) = \sum_{a\in\Sigma}A\mu(a)^\intercal A^* = A\qty[\sum_{a\in\Sigma}\mu(a)^\intercal]A^* = AA^*
    \end{equation*}
    This last equality follows from the definition of $A$, and in order for it to hold we must have $\sum_{a\in\Sigma}\mu(a)^\intercal = \I_\X$.
    Taking the transpose of both sides we have the required summation.

    Lastly we must show these $\mu(a)$ relate to $\eta(a)$ in the proper way.
    \begin{align*}
        \eta(a) & = A\mu(a)^\intercal A^*                              \\
                & = \tr_\Y\qty[\vec(A\mu(a)^\intercal)\vec(A)^*]       \\
                & = \tr_\Y\qty[\vec(\1_\X A\mu(a)^\intercal)\vec(A)^*] \\
                & = \tr_\Y\qty[(\1_X\otimes \mu(a))\vec(A)\vec(A)^*]   \\
                & = \tr_\Y\qty[(\1_X\otimes \mu(a))uu^*]
    \end{align*}
\end{solution}

%-----------------------------------------------------------------------------%

\begin{problem}
Let $\X$, $\Y$, and $\Z$ be complex Euclidean spaces and let
$\Phi\in\Channel(\X,\Y\otimes\Z)$ be a channel.
Prove that the following two statements are equivalent:
\begin{parts}
    \part
    There exists a complex Euclidean space $\W$, a state
    $\sigma\in\Density(\Y\otimes\W)$, and a channel
    $\Psi\in\Channel(\W\otimes\X,\Z)$ so that
    \[
        \Phi(X) = \qty\big[\I_{\mathsmaller{\Lin(\Y)}} \otimes
            \Psi](\sigma\otimes X)
    \]
    for all $X\in\Lin(\X)$.\label{part:4a}

    \part
    There exists a density operator $\rho\in\Density(\Y)$ for which
    \[
        \tr_{\Z}(J(\Phi)) = \rho \otimes \I_{\X}.
    \]\label{part:4b}
\end{parts}
\end{problem}

\begin{solution}
    \ref{part:4a}
    \ref{part:4b}
\end{solution}

\end{document}