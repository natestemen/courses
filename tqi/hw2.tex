\documentclass[boxes,pages,color=SeaGreen]{homework}

\hypersetup{
    colorlinks=true,
    urlcolor=SeaGreen!60!black,
    linkcolor=Bittersweet
}
\newcommand{\collab}[1]{\footnote{\href{mailto:#1}{\texttt{#1}}}}

\name{Nate Stemen}
\studentid{20906566}
\email{nate@stemen.email}
\term{Fall 2021}
\course{Theory of Quantum Information}
\courseid{QIC 820}
\hwnum{2}
\duedate{Nov 5, 2021}

\hwname{Assignment}

\usepackage{physics}
\usepackage{relsize}
\usepackage{macros}
\usepackage{cleveref}
\usepackage{multirow}
\usepackage{booktabs}
\usepackage{todonotes}

%-----------------------------------------------------------------------------%
% Macros
%-----------------------------------------------------------------------------%

\renewcommand{\vec}{\operatorname{vec}}

\newcommand{\X}{\mathcal{X}}
\newcommand{\Y}{\mathcal{Y}}
\newcommand{\Z}{\mathcal{Z}}
\newcommand{\W}{\mathcal{W}}
\newcommand{\V}{\mathcal{V}}
\newcommand{\U}{\mathcal{U}}
\renewcommand{\P}{\mathcal{P}}
\newcommand{\Fid}{\operatorname{F}}

\newcommand{\Lin}{\mathrm{L}}
\newcommand{\Trans}{\mathrm{T}}
\newcommand{\Pos}{\mathrm{Pos}}
\newcommand{\Herm}{\mathrm{Herm}}
\newcommand{\Channel}{\mathrm{C}}
\newcommand{\Unitary}{\mathrm{U}}
\newcommand{\Density}{\mathrm{D}}
\newcommand{\CP}{\mathrm{CP}}

\newcommand{\triplenorm}[1]{
  \lvert\!\lvert\!\lvert #1 
  \rvert\!\rvert\!\rvert}

\newcommand{\bigtriplenorm}[1]{
  \bigl\lvert\!\bigl\lvert\!\bigl\lvert #1 
  \bigr\rvert\!\bigr\rvert\!\bigr\rvert}

\newcommand{\Bigtriplenorm}[1]{
  \Bigl\lvert\!\Bigl\lvert\!\Bigl\lvert #1 
  \Bigr\rvert\!\Bigr\rvert\!\Bigr\rvert}

\newcommand{\biggtriplenorm}[1]{
  \biggl\lvert\!\biggl\lvert\!\biggl\lvert #1 
  \biggr\rvert\!\biggr\rvert\!\biggr\rvert}

\newcommand{\Biggtriplenorm}[1]{
  \Biggl\lvert\!\Biggl\lvert\!\Biggl\lvert #1 
  \Biggr\rvert\!\Biggr\rvert\!\Biggr\rvert}

\begin{document}


\def\arraystretch{1.2}

%-----------------------------------------------------------------------------%

\begin{problem}
Suppose that $\X$ and $\Y$ are complex Euclidean spaces and $M\in\Lin(\Y,\X)$
is a given operator.
Define a map $\Phi\in\Trans(\X\oplus\Y)$ as
\[
    \Phi\mqty(
    X & Z \\
    W & Y
    ) = \mqty(
    X & 0 \\
    0 & Y
    )
\]
for all $X\in\Lin(\X)$, $Y\in\Lin(\Y)$, $Z\in\Lin(\Y,\X)$, and
$W\in\Lin(\X,\Y)$ (i.e., $\Phi$ zeroes out the off-diagonal blocks of a
$2\times 2$ block operator of the form suggested in the equation), and
consider the semidefinite program
\[
    \qty(
    \Phi,
    \frac{1}{2}\mqty(0 & M \\ M^* & 0),
    \mqty(\1_{\X} & 0 \\ 0 & \1_{\Y})
    ).
\]

\begin{parts}
    \part
    Express the primal and dual problems associated with this semidefinite
    program in simple, human-readable terms.
    (There is no single, well-defined answer to this part of the problem---just
    do your best to make the primal and dual problems look as simple and
    elegant as possible.)\label{part:1a}

    \part
    Prove that strong duality holds for this semidefinite program.\label{part:1b}

    \part
    What is the optimal value of this semidefinite program?\label{part:1c}

\end{parts}
\end{problem}

\noindent Solution completed in collaboration with Alev Orfi,\collab{akborfi@uwaterloo.ca} and Muhammad Usman Farooq.\collab{mu7faroo@uwaterloo.ca}

{\noindent\color{SeaGreen!30}\rule{\textwidth}{1.5pt}}

\begin{solution}
    \ref{part:1a}
    Given the nice form of $\Phi$ we can start by doing some algebra to simplify the program.
    Starting with the definition of a semidefinite program we have
    \begin{center}
        \begin{tabular}{rl}
            $\underset{X}{\text{maximize}}$: & $\left\langle \frac{1}{2}\smqty(0 & M \\ M^* & 0), X\right\rangle$ \\ \cmidrule(lr{1em}){2-2}
            \multirow{2}{*}{subject to:}     & $\Phi(X) = \smqty(\1_\X           & 0 \\ 0 & \1_\Y)$                         \\
                                             & $X\in\Pos(\X\oplus\Y)$.
        \end{tabular}
    \end{center}
    Taking $X = \smqty(A & B \\ C & D)$ together with the condition that $\Phi(X) = \smqty(A & 0 \\ 0 & D) = \smqty(\1_\X & 0 \\ 0 & \1_\Y)$ we see that $A = \1_\X$ and $B = \1_\Y$.
    Since $X\in\Pos(\X)$ it must also be hermitian ($X = X^*$), and hence $C = B^*$ making $X = \smqty(\1_\X & B \\ B^* & \1_\Y)$.
    We can now expand out the inner product we are trying to maximize as follows (temporarily omitting the $\frac{1}{2}$):
    \begin{align*}
        \left\langle\smqty(0 & M                                               \\ M^* & 0), \smqty(\1_\X & B \\ B^* & \1_\Y)\right\rangle & = \tr\qty[\smqty(MB^* & M \\ M^* & M^*B)] \\
                             & = \tr(MB^*) + \tr(M^*B)                         \\
                             & = \langle B, M\rangle + \langle B^*, M^*\rangle
    \end{align*}
    This allows us to rewrite the primal problem as
    \begin{center}
        \begin{tabular}{rl}
            $\underset{B}{\text{maximize}}$: & $\frac{1}{2}\langle B, M\rangle + \frac{1}{2}\langle B^*, M^*\rangle$     \\ \cmidrule(lr{1em}){2-2}
            \multirow{2}{*}{subject to:}     & $\smqty(\1_\X                                                         & B \\ B^* & \1_Y) \in \Pos(\X\oplus\Y)$                         \\
                                             & $B\in\Lin(\Y, \X)$.
        \end{tabular}
    \end{center}
    In order to understand the dual problem we have to first know what $\Phi^*$ is.
    This can be calculated by the definition of a adjoint map: $\langle\Phi(X), Y\rangle = \langle X, \Phi^*(Y)\rangle$.
    \begin{align*}
        \langle\Phi\smqty(X & Z \\ W & Y), \smqty(A & B \\ C & D)\rangle & = \tr(\smqty(X^*A & \cdot \\ \cdot & Y^*D)) = \tr(X^*A) + \tr(Y^*D) = \langle X, A\rangle + \langle Y , D \rangle \\
        \langle\smqty(X     & Z \\ W & Y), \Phi^*\smqty(A & B \\ C & D)\rangle & = \tr(\smqty(X^*A' + W^*C' & \cdot \\ \cdot & Z^*B' + Y^*D'))
    \end{align*}
    In order for this to be equal to $\tr(X^*A) + \tr(Y^*D)$ we must have $A' = A$, $D' = D$, as well as $C' = 0 = B'$.
    This gives us the action of $\Phi^*$ as $\Phi^*\smqty(A & B \\ C & D) = \smqty(A & 0 \\ 0 & D)$ and hence $\Phi = \Phi^*$.

    To simplify the dual problem we first write it in it's full generality:
    \begin{center}
        \begin{tabular}{rl}
            $\underset{Y}{\text{minimize}}$: & $\left\langle \smqty(\1_\X        & 0 \\ 0 & \1_\Y), Y\right\rangle$ \\ \cmidrule(lr{1em}){2-2}
            \multirow{2}{*}{subject to:}     & $\Phi(Y) \geq \frac{1}{2}\smqty(0 & M \\ M^* & 0)$    \\
                                             & $Y\in\Herm(\X\oplus \Y)$.
        \end{tabular}
    \end{center}
    With a little bit of algebra, together with the fact that $Y\in\Herm(\X\oplus\Y)$ this problem transforms to
    \begin{center}
        \begin{tabular}{rl}
            $\underset{A, D}{\text{minimize}}$: & $\tr(A) + \tr(D)$                 \\ \cmidrule(lr{1em}){2-2}
            \multirow{2}{*}{subject to:}        & $\smqty(A         & -\frac{1}{2}M \\ -\frac{1}{2}M^* & D)\in\Pos(\X\oplus \Y)$    \\
                                                & $A\in\Herm(\X)$                   \\
                                                & $D\in\Herm(\Y)$.
        \end{tabular}
    \end{center}
    \todo{Should $A,D\in\Pos(\X/\Y)$?}

    \ref{part:1b}
    \ref{part:1c}
\end{solution}

%-----------------------------------------------------------------------------%

\begin{problem}
Let $\X$ be a complex Euclidean space, and define
\[
    \delta(P,Q) = \sqrt{\tr(P) + \tr(Q) - 2 \Fid(P,Q)}
\]
for all positive semidefinite operators $P,Q\in\Pos(\X)$.
Prove that $\delta$ satisfies these three properties:
\begin{parts}
    \part
    $\delta(P,Q) \geq 0$ for all $P,Q\in\Pos(\X)$, with $\delta(P,Q) = 0$ if
    and only if $P = Q$.\label{part:2a}
    \part
    $\delta(P,Q) = \delta(Q,P)$ for all $P,Q\in\Pos(\X)$.\label{part:2b}
    \part
    $\delta(P,Q) \leq \delta(P,R) + \delta(R,Q)$
    for all $P,Q,R\in\Pos(\X)$.\label{part:2c}
\end{parts}
(These are the three defining properties of a \emph{metric}.)

Hint: to prove that property (c) holds, first prove that if
$\Y$ is a complex Euclidean space with $\dim(\Y)\geq\dim(\X)$,
and $u\in\X\otimes\Y$ is any vector satisfying $\tr_{\Y}(u u^{\ast}) = P$,
then
\[
    \delta(P,Q) = \min_{v\in\X\otimes\Y}\qty\big{\norm{u - v} : \tr_{\Y}(v v^*) = Q}.
\]
\end{problem}

\begin{solution}
    \ref{part:2a}
    By the symmetry of $\delta$ (proved next) we can without loss of generality take $\tr(P)\geq \tr(Q)$.
    Then, because $\tr(X) \geq 0$ for all $X\in\Pos(\X)$ we have
    \begin{align*}
        \qty(\sqrt{\tr(P)} - \sqrt{\tr(Q)})^2 & = \tr(P) + \tr(Q) - 2\sqrt{\tr(P)\tr(Q)} \geq 0 \\
        \tr(P) + \tr(Q)                       & \geq 2\sqrt{\tr(P)\tr(Q)}                       \\
                                              & \geq 2\Fid(P, Q) \tag{Proposition 3.12:6}
    \end{align*}
    By the last inequality we have that the radicand is greater than or equal to 0, and hence $\delta(P, Q) \geq 0$.
    In order for $\delta(P, Q) = 0$, we must have both $\tr(P) = \tr(Q)$ and $Q = \lambda P$ again by proposition 3.12:6.
    Then we have $\tr(P) = \tr(\lambda P) = \lambda \tr(P)$ and hence $\lambda = 1$ implying $P = Q$.

    \ref{part:2b}
    The symmetry of $\delta$ follows immediately from Proposition 3.12:2 which states that $\Fid(P, Q) = \Fid(Q, P)$.
    \begin{equation*}
        \delta(P, Q) = \sqrt{\tr(P) + \tr(Q) - 2\Fid(P, Q)} = \sqrt{\tr(Q) + \tr(P) - 2\Fid(Q, P)} = \delta(Q, P)
    \end{equation*}

    \ref{part:2c}
    \todo{prove the hint}

    With the hint proved, let $p, q, r\in\X\otimes\Y$ be a purifications of $P$, $Q$ and $R$ respectively.
    \begin{align*}
        \delta(P, Q) & = \min_{q}\qty{\norm{p - q}}                                 \\
                     & = \min_{q}\qty{\norm{p - q - r + r}}                         \\
                     & \leq \min_{q}\qty{\norm{p - r} + \norm{r - q}}               \\
                     & \leq \min_{r}\qty{\norm{p - r}} + \min_{q}\qty{\norm{r - q}} \\
                     & = \delta(P, R) + \delta(R, Q)
    \end{align*}

\end{solution}

%-----------------------------------------------------------------------------%

\begin{problem}
Let $\Phi\in\Trans(\X,\Y)$ be a map, for complex Euclidean spaces $\X$ and
$\Y$.
Prove that
\[
    \triplenorm{\Phi}_1 =
    \max_{\rho_0,\rho_1\in\Density(\X)}
    \norm{
        \qty\big( \1_{\Y} \otimes \sqrt{\rho_0} ) J(\Phi)
        \qty\big( \1_{\Y} \otimes \sqrt{\rho_1} )}_1.
\]
\end{problem}

\noindent Solution completed in collaboration with Mohammad Ayyash,\collab{mmayyash@uwaterloo.ca} and Nicholas Zutt.\collab{nzutt@uwaterloo.ca}

{\noindent\color{SeaGreen!30}\rule{\textwidth}{1.5pt}}

\begin{solution}

\end{solution}

\end{document}
