% !TEX program = xelatex
\documentclass[11pt,aspectratio=1610]{beamer}

\usetheme[
    background=light,
    numbering=fraction,
    block=fill,
    progressbar=frametitle
]{metropolis}
\usepackage{appendixnumberbeamer}

\usepackage{wasysym}
\usepackage{booktabs}
\usepackage{physics}
\usepackage{amsmath}
\usepackage{mathtools}
\usepackage{bm}
\usepackage{tcolorbox}
\newtcolorbox{idea}{colback=green!5!white,colframe=green!75!black}
\newtcolorbox{warning}{colback=red!5!white,colframe=red!75!black}

\usepackage{tikz}
\usetikzlibrary{quantikz}

\usefonttheme[onlymath]{serif}
\usefonttheme{professionalfonts}
\usepackage{unicode-math}

\defaultfontfeatures{ Scale=MatchLowercase }
% \setmainfont{TeX Gyre Pagella}[Scale = 1.0]
\setmathfont{Asana Math}
% \setmathfont{Neo Euler}[
%   range={up/{Latin, latin, Greek, greek},
%          bfup/{Latin, latin, Greek, greek},
%          cal, bfcal,
%          frak, bffrak
%         },
%   script-features={},
%   sscript-features={} ]


\newcommand{\iu}{\mathrm{i}\mkern1mu}
\newcommand{\unitary}[1]{\raisebox{-1pt}{\scalebox{1.2}{$\bm{\mathsf{U}}$}}\qty(#1)}


\title{The Gottesman-Knill Theorem}
\subtitle{What is it and what does it mean?}
\date{10/12/2020}
\author{Nate Stemen (he/they)}
\institute{QIC 710 Final Project}
\titlegraphic{\hfill\includegraphics[height=2cm]{uw-logo-blue.png}}


\begin{document}

\maketitle

\begin{frame}{The Gottesman-Knill Theorem}
	\begin{theorem}[\cite{gottesman-knill}]
		A quantum circuit using only the following elements can be efficiently \textbf{simulated} on a classical computer:
		\begin{enumerate}
			\item Qubits prepared in computational basis states
			\item Quantum gates from the \textbf{Clifford group}
			\item Measurements in the computational basis
		\end{enumerate}
	\end{theorem}
\end{frame}

\begin{frame}{What does a classical simulation of a quantum computer mean?}
	Two different main kinds of simulation possible:\pause
	\begin{exampleblock}{Strong Simulation}
		Given an input $x$ to our quantum computer, compute $p(x)$.
	\end{exampleblock}\pause
	\begin{exampleblock}{Weak Simulation}
		Given an input $x$, compute a sample from $p(x)$.
	\end{exampleblock}\pause
	Gottesman-Knill theorem deals with weak simulation.

	Strong simulation of quantum computers shown to be $\#\bm{\mathsf{P}}$-hard \cite{nest}.
\end{frame}

\begin{frame}[t]{How can we (na\"ively) simulate a quantum computer?}
	Suppose we have $n$ qubits and we want to run them through $D$ gates.
	\begin{center}
		\begin{quantikz}
			\lstick[wires=3]{$\mathbb{C}^{2^n}\ni \ket{\psi}$} & \gate[wires=3]{A_1} & \gate[wires=3]{A_2} & \ \ldots\ \qw &  \gate[wires=3]{A_D} & \qw \rstick[wires=3]{$A_D\cdots A_2A_1\ket{\psi}$} \\
			& \qwbundle[alternate]{}  &  \qwbundle[alternate]{}     & \ \ldots\ \qwbundle[alternate]{} & \qwbundle[alternate]{} & \qwbundle[alternate]{} \\
			&                     &                          & \ \ldots\ \qw & & \qw
		\end{quantikz}
	\end{center}

	\begin{onlyenv}<1>
		Final state contains $D-1$ matrix multiplications, each costing $O\qty(2^{3n})$\footnote{Theoretically possible to get $O\qty(2^{2.373 n})$.}, so total cost is $O\qty(D2^{3n})$.

		Simulating Grover's algorithm on 40 qubits took nearly a full day! \cite{slowsim}
	\end{onlyenv}
	\begin{onlyenv}<2>
		\begin{idea}
			\begin{center}
				What if we restrict the gates $A_i$?
			\end{center}
		\end{idea}
	\end{onlyenv}

\end{frame}

\begin{frame}{Stabilizer Formalism}
	\begin{itemize}[<+->]
		\item Let $\ket{\psi} = \frac{1}{\sqrt{2}}\qty(\ket{00} + \ket{11})$
		\item $X_1X_2\ket{\psi} = \ket{\psi}$ and $Z_1Z_2\ket{\psi} = \ket{\psi}$
		\item $\ket{\psi}$ is the \emph{unique} state stabilized by both of these operators.
		\item This hints at the possibility of describing some states not as vectors in $\mathbb{C}^{2^N}$, but of operators.
	\end{itemize}
\end{frame}

\begin{frame}[t]{Pauli Group}
	Let $X, Y, Z$ denote the standard single-qubit Pauli operators:
	\begin{align*}
		X & = \mqty(0 & \phantom{+}1 \\ 1 & \phantom{+}0) & Y & = \mqty(0 & -\iu \\ \iu & \phantom{-}0) & Z & = \mqty(1 & \phantom{-}0 \\ 0 & -1)
	\end{align*}
	\begin{onlyenv}<2->
		Take $X_i, Y_i, Z_i$ to denote $X, Y$ and $Z$ acting on the $i$-th qubit, and with the identity everywhere else.
	\end{onlyenv}
	\begin{onlyenv}<2>
		\begin{columns}
			\begin{column}{0.7\textwidth}
				\begin{equation*}
					X_i\coloneqq \mathbb{1}\otimes\cdots\otimes\overbrace{X}^{i\text{th operator}}\otimes \cdots \otimes \mathbb{1}
				\end{equation*}
			\end{column}
			\begin{column}{0.3\textwidth}
				\vspace{-0.8cm}
				\begin{quantikz}
					\lstick{$1$} & \qw & \qw \\
					\lstick{$\vdots$} & \qwbundle[alternate]{} & \qwbundle[alternate]{} \\
					\lstick{$i$} & \gate{X} & \qw \\
					\lstick{$\vdots$} & \qwbundle[alternate]{} & \qwbundle[alternate]{} \\
					\lstick{$n$} & \qw & \qw \\
				\end{quantikz}
			\end{column}
		\end{columns}
	\end{onlyenv}

	\begin{onlyenv}<3>
		\begin{equation*}
			P_n\coloneqq\qty{\pm\mathbb{1}, \pm\iu\mathbb{1}, \pm A_i, \pm\iu A_i: A_i\in\{\mathbf{1}, X_i, Y_i, Z_i\}}\equiv \langle X_i, Z_i\rangle
		\end{equation*}
		\begin{itemize}
			\item $P_n$ forms a group under matrix multiplication.
			\item Every pair of elements either commute or anti-commute.
			\item $\abs{P_n} = 4\cdot 4^n$
		\end{itemize}
	\end{onlyenv}
\end{frame}

\begin{frame}{Stabilizer States}
	Let $S$ be a subgroup of $P_n$. Define the \emph{vector space} $V_S$ as the states stabilized by everything in $S$.
	\begin{equation*}
		V_S\coloneqq \qty{\ket{\psi}\in\mathbb{C}^{2^n}: g\ket{\psi} = \ket{\psi}, \forall g\in S}
	\end{equation*}
	\begin{exampleblock}{Example}
		Take $P_3$ and subgroup $S = \qty{\mathbb{1}, Z_1Z_2, Z_2Z_3, Z_1Z_3}$. Note that $\ket{000}, \ket{001}, \ket{110}, \ket{111}$ are stabilized by $Z_1Z_2$, and $\ket{000}, \ket{100}, \ket{011}, \ket{111}$ are stabilized by $Z_2Z_3$. These, together with the fact that $Z_1Z_3 = (Z_1Z_2)(Z_2Z_3)$ tell us that $V_S = \qty{\ket{000}, \ket{111}}$. In this case we can write $S = \langle Z_1Z_2, Z_2Z_3\rangle$.
	\end{exampleblock}
	\onslide<2->{\begin{idea}\begin{center}$S$ and $V_S$ uniquely determine each other!\end{center}\end{idea}}
\end{frame}

\begin{frame}[t]{What about other gates?}
	Let $U$ by an arbitary unitary gate from $\unitary{2^n}$, $\ket{\psi}\in V_S$ and $g\in S$.\pause
	\begin{equation*}
		U\ket{\psi} = Ug\ket{\psi} = UgU^\dagger U\ket{\psi} = g'U\ket{\psi}
	\end{equation*}\pause
	So $U\ket{\psi}$ is stabilized by $UgU^\dagger$, and in general $UV_S$ is stabilized by $USU^\dagger = \qty{UgU^\dagger : g\in S}$.

	\begin{corollary}
		If $S$ is generated by $g_1,\ldots, g_n$, then $USU^\dagger$ is generated by $Ug_1U^\dagger,\ldots, Ug_nU^\dagger$.
	\end{corollary}\pause

	If $G$ is an Abelian group with $\abs{G}$ the number of elements in $G$, then the number of generators of $G$ is bounded by $\log\abs{G}$.\pause

	\begin{warning}
		\begin{center}
			What if $UgU^\dagger$ doesn't land back in $P_n$?
		\end{center}
	\end{warning}
\end{frame}

\begin{frame}[t]{Clifford Group}
	\only<1-2>{
		\begin{warning}
			\begin{center}
				What if $UgU^\dagger$ doesn't land back in $P_n$?
			\end{center}
		\end{warning}}

	\begin{theorem}
		Suppose $U$ in any unitary on $n$ qubits with the property that for $g\in P_n$ we have $UgU^\dagger\in P_n$. Then $U$ can be composed from $O(n^2)$ Hadamard, CNOT, and phase gates.
	\end{theorem}\pause

	\begin{block}{Definition}
		The \emph{Clifford Group} is defined to be the set of operators that leave Pauli operators invariant under conjugation.
		\begin{equation*}
			C_n \coloneqq \qty{V\in \unitary{2^n} : VP_nV^\dagger = P_n}
		\end{equation*}
	\end{block}
	\onslide<3>{\begin{align*}
			H & = \frac{1}{\sqrt{2}}\mqty(1 & \phantom{-}1 \\ 1 & -1) & P & = \mqty(1 & \phantom{+}0 \\ 0 & \phantom{+}\iu) & \textbf{CNOT}
		\end{align*}}
\end{frame}

\begin{frame}{Recap}
	\begin{enumerate}[<+->]
		\item Simulating a quantum computer in general is \emph{really} hard!
		\item What can we simulate more easily?
		\item Stabilizer formalism gave us a way to track operators instead of state vectors (duality between subgroup $S$ of Paulis and vector space of stabilised states $V_S$)
		\item Keeping track of the generators of a stabilizer $S$ provide a succinct way to understand how $S$ is changing ($\log\abs{S}$ generators)
		\item Found that elements of the Clifford group can efficiently build elements that conjugate the Pauli group back to the Pauli group
	\end{enumerate}
\end{frame}

\begin{frame}{Back to the Theorem}
	\begin{theorem}[\cite{gottesman-knill}]
		A quantum circuit using only the following elements can be efficiently simulated on a classical computer:
		\begin{enumerate}
			\item Qubits prepared in computational basis states
			\item Quantum gates from the Clifford group
			\item Measurements in the computational basis
		\end{enumerate}
	\end{theorem}
	\begin{itemize}
		\item Take $\ket{\psi} = \ket{0}^{\otimes n}$. Now we can say $S = \langle Z_1, \ldots, Z_n\rangle$.
		\item Under some action $U\in C_n$ state will evolve to $U\ket{\psi} = UgU^\dagger U\ket{\psi}$ for $g\in S$
		\item Switch over to describing the change in generators of $S$
		\item Need to compute $UZ_1U^\dagger,\ldots, UZ_nU^\dagger$
	\end{itemize}

\end{frame}

\begin{frame}{Back to the Theorem}
	\begin{exampleblock}{Recap}
		We have $\ket{\psi} = \ket{0}^{\otimes n}$, $S = \langle Z_1,\ldots , Z_n\rangle$, and $U\in C_n$. We know that $U\ket{\psi} = g'U\ket{\psi}$, so in order to figure out where it evolves to, need to compute the new generators of the space $UV_S$ which are $UZ_1U^\dagger,\ldots, UZ_nU^\dagger$.
	\end{exampleblock}
	\begin{columns}
		\begin{column}{0.7\textwidth}
			\begin{itemize}
				\item Only takes $2n+1$ bits to keep track of the generators: 2 for the $n$ Pauli generators, and 1 for the phase of $\pm 1$
				\item Specifying $\ket{\psi}$ requires all $n$ generators, so $n(2n+1)$ bits
				\item Updating the generators only takes $O(n)$
				\item Total cost $O(n^2)$
			\end{itemize}
		\end{column}
		\begin{column}{0.3\textwidth}
			\begin{table}
				\centering
				\begin{tabular}{c}
					$\mathbb{1}\longrightarrow$ 00 \\
					$X\longrightarrow$ 01          \\
					$Y\longrightarrow$ 10          \\
					$Z\longrightarrow$ 11
				\end{tabular}
				\caption{Pauli Encoding}
				\label{tab:encoding}
			\end{table}
		\end{column}
	\end{columns}
\end{frame}

\begin{frame}[t]{What does this mean?}
	What makes quantum computers powerful?\pause
	\begin{warning}
		\begin{center}
			Not entanglement!
		\end{center}
	\end{warning}\pause
	\only<3>{Circuits built with the Clifford group don't actually provide us with any more power than a classical computer, despite CNOTs, Hadamards, and phase gates being used to construct purely quantum phenomenon.}
	\onslide<4->{\begin{columns}
			\begin{column}{0.2\textwidth}
				\only<4>{Quantum teleportation}
				\only<5>{Superdense coding}
			\end{column}
			\begin{column}{0.8\textwidth}
				\begin{center}
					\only<4>{\begin{quantikz}[row sep=0.8cm]
							\lstick{$\ket{\psi}$} & \qw & \qw & \ctrl{1}\gategroup[2,steps=5,style={dashed,rounded corners,fill=blue!20, inner xsep=2pt},background, label style={label position=above,yshift=-1.7cm, xshift=0.4cm}]{Alice}   & \gate{H} & \meter{$M_1$}  & \cw & \cwbend{2}    \\
							\lstick{$\ket{0}$} & \gate{H} & \ctrl{1} & \targ{}  & \qw & \meter{$M_2$} & \cwbend{1}           \\
							\lstick{$\ket{0}$} & \qw      & \targ{} & \qw \gategroup[1,steps=5,style={dashed,rounded corners,fill=green!20, inner xsep=2pt},background, label style={label position=below,xshift=-0.3cm}]{Bob}     & \qw & \qw & \gate{X^{M_2}} & \gate{Z^{M_1}} & \qw \rstick{$\ket{\psi}$}
						\end{quantikz}}
					\only<5>{\begin{quantikz}[row sep=0.8cm]
							\lstick{$\ket{0}$} & \gate{H} & \ctrl{1} & \gate{X^a} & \gate{Z^b} & \ctrl{1} & \gate{H} & \meter{} & \cw\rstick{$a$} \\
							\lstick{$\ket{0}$} & \qw      & \targ{}  & \qw      & \qw      & \targ{}  & \qw      & \meter{} & \cw\rstick{$b$}
						\end{quantikz}}
				\end{center}
			\end{column}
		\end{columns}}
\end{frame}

\begin{frame}{Where do we go from here?}
	\begin{itemize}[<+->]
		\item Clifford gates aren't enough for universal quantum computation
		\item Clifford group shown to be $\oplus\bm{\mathsf{L}}$-complete \cite{aaronson}
		\item Adding \emph{any} 1 or 2-qubit gate\footnote<3->{That doesn't map computational basis states to computational basis states} will turn the Cliffords into a universal set \cite{shi}
	\end{itemize}
	\onslide<2->{\begin{center}
			\includegraphics[width=0.65\textwidth]{complexity.pdf}
		\end{center}}
\end{frame}

\begin{frame}{Conclusion}
	\begin{alertblock}{Assuming}
		$\bm{\mathsf{BQP}}\neq\bm{\mathsf{P}}\neq\oplus\bm{\mathsf{L}}$
	\end{alertblock}
	Strong simulation of quantum computers is \emph{really} hard.

	Clifford group is only capable of solving relatively easy problems (both from classical and quantum POV).

	Quantum entanglement is not the only contributing factor to the power of quantum computers!
\end{frame}

\begin{frame}[allowframebreaks]{References}
	\nocite{*}
	\bibliographystyle{apalike}
	\bibliography{refs}
\end{frame}

\begin{frame}[standout]
	\smiley Thank you!\smiley

	Questions?
\end{frame}


\begin{frame}[t]{Example}
	\begin{columns}[T,onlytextwidth]
		\column{0.8\textwidth}
		\begin{exampleblock}{Alice's Broken Quantum Computer \frownie}
			Alice's quantum computer is working too well. Instead of performing single controlled-NOT gates, it does three at a time. What is it actually doing?
		\end{exampleblock}

		\column{0.2\textwidth}
		\begin{center}
			\begin{quantikz}
				& \ctrl{1}  & \targ{}   & \ctrl{1} & \qw \\
				& \targ{}   & \ctrl{-1} & \targ{}  & \qw
			\end{quantikz}
		\end{center}
	\end{columns}

	\onslide<2->{Because $X_1, X_2, Z_1, Z_2$ generate the Pauli group we can follow what happens to them under the evolution of this circuit.}
	\begin{align*}
		\only<2>{X_1     & = X\otimes\mathbb{1} \xrightarrow{\text{CNOT 1}} \text{CNOT}\cdot \qty(X\otimes\mathbb{1}) \cdot\text{CNOT}^\dagger = X\otimes X\\}
		\onslide<3->{X_1 & = X\otimes\mathbb{1}\xrightarrow{\text{CNOT 1}}X\otimes X\xrightarrow{\text{CNOT 2}}\mathbb{1}\otimes X\xrightarrow{\text{CNOT 3}}\mathbb{1}\otimes X = X_2 \\}
		\onslide<3->{Z_1 & = Z\otimes\mathbb{1}\xrightarrow{\text{CNOT 1}}Z\otimes \mathbb{1}\xrightarrow{\text{CNOT 2}}Z\otimes Z\xrightarrow{\text{CNOT 3}}\mathbb{1}\otimes Z = Z_2}
	\end{align*}
	\onslide<3->{Further, we can show $X_1\longleftrightarrow X_2$ and $Z_1\longleftrightarrow Z_2$. This is exactly a swap operation!}
\end{frame}

\begin{frame}[t]{Example, continued}
	\begin{columns}[T,onlytextwidth]
		\column{0.8\textwidth}
		\begin{exampleblock}{Alice's less Broken Quantum Computer \smiley}
			By dint of no little hard work, Alice has partially fixed her quantum computer. Now it only does 2 CNOTs at a time. Unfortunately, she can only get this improvement if she puts a $\ket{0}$ as the second input qubit. What does it do now?
		\end{exampleblock}

		\column{0.2\textwidth}
		\vspace{.5cm}
		\begin{center}
			\begin{quantikz}
				\lstick{$\ket{\alpha}$} & \ctrl{1} & \targ{} & \qw \\
				\lstick{$\ket{0}$}      & \targ{}  & \ctrl{-1}  & \qw
			\end{quantikz}
		\end{center}
	\end{columns}
	In this case we see the initial state $\ket{\psi_0} = \ket{\alpha}\otimes\ket{0}$ is stabilized by $Z_2$.
	\only<2>{\begin{align*}
			X_1 & = X\otimes\mathbb{1}\xrightarrow{\text{CNOT 1}}X\otimes X\xrightarrow{\text{CNOT 2}}\mathbb{1}\otimes X \\
			Z_1 & = Z\otimes\mathbb{1}\xrightarrow{\text{CNOT 1}}Z\otimes \mathbb{1}\xrightarrow{\text{CNOT 2}}Z\otimes Z \\
			Z_2 & = \mathbb{1}\otimes Z\xrightarrow{\text{CNOT 1}}Z\otimes Z\xrightarrow{\text{CNOT 2}}Z\otimes\mathbb{1}
		\end{align*}}
	\onslide<3->{\begin{align*}
			X_1 & \longrightarrow \mathbb{1}\otimes X & Z_1 & \longrightarrow Z\otimes Z & Z_2 & \longrightarrow Z\otimes\mathbb{1}
		\end{align*}}
	State will always be a $+1$ eigenvector of $Z_2$, so it follows that $(Z\otimes\mathbb{1})(Z\otimes Z) = \mathbb{1}\otimes Z$.
	\onslide<4->{\begin{align*}
			X_1 & \longrightarrow \mathbb{1}\otimes X & Z_1 & \longrightarrow \mathbb{1}\otimes Z
		\end{align*}}
	Circuit still performs a swap!
\end{frame}

\end{document}
