\documentclass[boxes,pages]{homework}

\name{Nate Stemen}
\studentid{20906566}
\email{nate.stemen@uwaterloo.ca}
\term{Fall 2020}
\course{Quantum Information Processing}
\courseid{QIC 710}
\hwnum{3}
\duedate{Thur, Oct 1, 2020 11:59 PM}

\hwname{Assignment}
% \problemname{Exercise}
%\solutionname{(Name)}

\usepackage{physics}
\usepackage{cancel}
\usepackage{bbold}
% \usepackage{nicefrac}
\usepackage{qcircuit}
\usepackage{booktabs}
\usepackage{cleveref}

\newcommand{\iu}{\mathrm{i}\mkern1mu}

\begin{document}

\noindent
I worked with Chelsea Komlo and Wilson Wu on this assignment.

\begin{problem}
Measuring the control qubit of a CNOT gate.
\end{problem}

\begin{solution}
	Lets start with the most general 2 qubit state.
	\begin{equation}\label{eq:genstate}
		\alpha_{00}\ket{00} + \alpha_{01}\ket{01} + \alpha_{10}\ket{10} + \alpha_{11}\ket{11}
	\end{equation}
	Starting with the gate on the left, we first have to make a partial measurement of the first qubit. This collapses the state into one of two possibilities.
	\begin{center}
		\begin{tabular}{ r c | c }
			First qubit state       & $\ket{0}$                                                                                      & $\ket{1}$                                                                                       \\
			State after measurement & $\frac{1}{n}\ket{0}\otimes\underbrace{\qty(\alpha_{00}\ket{0} + \alpha_{01}\ket{1})}_{\phi_0}$ & $\frac{1}{n'}\ket{1}\otimes\underbrace{\qty(\alpha_{10}\ket{0} + \alpha_{11}\ket{1})}_{\phi_1}$
		\end{tabular}
	\end{center}
	Where $n = \abs{a_{00}}^2 + \abs{\alpha_{01}}^2$ and similarly with $n'$ as factors for normalization. We can now feed these two states into a controlled $U$ gate to see what comes out.
	\begin{center}
		\begin{tabular}{c@{\hskip 0.75in} c}
			Measured $\ket{0}$ & Measured $\ket{1}$ \\ \toprule
			$\frac{1}{n}\mqty[\mathbb{1} & 0 \\ 0 & U]\mqty[\phi_0 \\ 0] = \frac{1}{n}\mqty[\phi_0 \\ 0]$ & $\frac{1}{n'}\mqty[\mathbb{1} & 0 \\ 0 & U]\mqty[0 \\ \phi_1] = \frac{1}{n'}\mqty[0 \\ U\phi_1]$
		\end{tabular}
	\end{center}
	Now we can switch our attention to the circuit on the right hand side. First we have to apply a controlled $U$ to our general state defined in \cref{eq:genstate}.
	\begin{equation*}
		\mqty[\mathbb{1} & 0 \\ 0 & U]\smqty[\alpha_{00} \\ \alpha_{01} \\ \alpha_{10} \\ \alpha_{11}] = \smqty[\alpha_{00} \\ \alpha_{01} \\ u_{11}\alpha_{10} + u_{12}\alpha_{11} \\ u_{21}\alpha_{10} + u_{22}\alpha_{11}]
	\end{equation*}
	Now with this state we must make a partial measurement of the first qubit. As before we will only measure $\ket{0}$ or $\ket{1}$ so we have two branches.
	\begin{center}
		\begin{tabular}{c@{\hskip 0.5in} c}
			Measure $\ket{0}$ & Measure $\ket{1}$ \\ \toprule
			$\smqty[\alpha_{00} \\ \alpha_{01} \\ u_{11}\alpha_{10} + u_{12}\alpha_{11} \\ u_{21}\alpha_{10} + u_{22}\alpha_{11}] \to \frac{1}{n}\smqty[\alpha_{00} \\ \alpha_{01} \\ 0 \\ 0]$ & $\smqty[\alpha_{00} \\ \alpha_{01} \\ u_{11}\alpha_{10} + u_{12}\alpha_{11} \\ u_{21}\alpha_{10} + u_{22}\alpha_{11}] \to \frac{1}{n'}\smqty[0 \\ 0 \\ u_{11}\alpha_{10} + u_{12}\alpha_{11} \\ u_{21}\alpha_{10} + u_{22}\alpha_{11}]$
		\end{tabular}
	\end{center}
	We now have the four states we need to conclude whether or not these circuits are the same. Measuring 0 in the first circuit led us to a state we called $\frac{1}{n}\smqty[\phi_0 \\ 0]$ which is shorthand for $\frac{1}{n}\smqty[\alpha_{00} \\ \alpha_{01} \\ 0 \\ 0]$. This is exactly what we got when first applying the controlled $U$ and then making the measurement so that works out.

	The case for measuring 1 then applying the controlled $U$ gate got us $\frac{1}{n'}\smqty[0 \\ U\phi_1]$ which again was shorthand for $\frac{1}{n'}\smqty[0 \\ 0 \\ u_{11}\alpha_{10} + u_{12}\alpha_{11} \\ u_{21}\alpha_{10} + u_{22}\alpha_{11}]$. This is what we got for the first circuit. At first I thought there might be a different normalization constant that needed to be added instead of just $n'$, but because unitary operators are isometries (distance preserving maps), the length of the vector remains unchanged and hence still normalized.
\end{solution}

\begin{problem}\label{prob:two}
Unitary between two triples of states with same inner products.
\end{problem}

\begin{solution}
	First assume the sets $\{\phi_i\}$ and $\{\psi_i\}$ are both linearly independent sets of vectors when considered on their own. Let us apply Gram-Schmidt orthonormalization to $\{\phi_i\}$ to obtain $\{\tilde{\phi}_i\}$ and similarly with $\{\psi_i\}$ to obtain $\{\tilde{\psi}_i\}$. We note that we now have two orthonormal bases for $\mathbb{C}^3$, and we know basis transformations are unitary maps. This means we can find a $U$ such that $U\ket{\tilde{\phi}_i} = \ket{\tilde{\psi}_i}$.

	Now we need to show $U\ket{\phi_i} = \ket{\psi_i}$. To do this lets look at how the $\{\tilde{\phi}_i\}$ were created.
	\begin{align*}
		\ket{\tilde{\phi}_0} & = \ket{\phi_0}                                                                       & \ket{\tilde{\psi}_0} & = \ket{\psi_0}                                                                       &   \\
		\ket{\tilde{\phi}_1} & = \ket{\phi_1} - \frac{\braket{\phi_1}{\phi_0}}{\braket{\phi_0}{\phi_0}}\ket{\phi_0} & \ket{\tilde{\psi}_1} & = \ket{\psi_1} - \frac{\braket{\psi_1}{\psi_0}}{\braket{\psi_0}{\psi_0}}\ket{\psi_0} &   \\
		\ket{\tilde{\phi}_2} & = \ket{\phi_2} - \frac{\braket{\phi_2}{\tilde{\phi}_0}}{\braket{\tilde{\phi}_0}{\tilde{\phi}_0}}\ket{\tilde{\phi}_0} - \frac{\braket{\phi_2}{\tilde{\phi}_1}}{\braket{\tilde{\phi}_1}{\tilde{\phi}_1}}\ket{\tilde{\phi}_1} & \ket{\tilde{\psi}_2} & = \ket{\psi_2} - \frac{\braket{\psi_2}{\tilde{\psi}_0}}{\braket{\tilde{\psi}_0}{\tilde{\psi}_0}}\ket{\tilde{\psi}_0} - \frac{\braket{\psi_2}{\tilde{\psi}_1}}{\braket{\tilde{\psi}_1}{\tilde{\psi}_1}}\ket{\tilde{\psi}_1}
	\end{align*}
	It clear that $U\ket{\tilde{\phi}_0} = \ket{\tilde{\psi}_0}$ implies $U\ket{\phi_0} = \ket{\psi_0}$. Now let's apply $U$ to $\ket{\tilde{\phi}_1}$.
	\begin{align*}
		U\ket{\tilde{\phi}_1} & = U\ket{\phi_1} - \braket{\phi_1}{\phi_0}U\ket{\phi_0}                                                                               \\
		                      & = U\ket{\phi_1} - \braket{\psi_1}{\psi_0}\ket{\psi_0} \tag{\footnotesize{bc inner prods equal, and $U\ket{\phi_0} = \ket{\psi_0}$.}} \\
		                      & = \ket{\psi_1} - \braket{\psi_1}{\psi_0}\ket{\psi_0} \tag{\footnotesize{RHS of $U\ket{\tilde{\phi}_1} = \ket{\tilde{\psi}_1}$.}}
	\end{align*}
	Where the last two lines show us $U\ket{\phi_1} = U\ket{\psi_1}$ as desired. A similar computation can be done with the third vector.
	\begin{align*}
		U\ket{\tilde{\phi}_2} & = U\ket{\phi_2} - \frac{\braket{\phi_2}{\tilde{\phi}_0}}{\braket{\tilde{\phi}_0}{\tilde{\phi}_0}}U\ket{\tilde{\phi}_0} - \frac{\braket{\phi_2}{\tilde{\phi}_1}}{\braket{\tilde{\phi}_1}{\tilde{\phi}_1}}U\ket{\tilde{\phi}_1} \\
		                      & = U\ket{\phi_2} - \frac{\braket{\phi_2}{\tilde{\phi}_0}}{\braket{\tilde{\phi}_0}{\tilde{\phi}_0}}\ket{\tilde{\psi}_0} - \frac{\braket{\phi_2}{\tilde{\phi}_1}}{\braket{\tilde{\phi}_1}{\tilde{\phi}_1}}\ket{\tilde{\psi}_1}   \\
		                      & = \ket{\psi_2} - \frac{\braket{\psi_2}{\tilde{\psi}_0}}{\braket{\tilde{\psi}_0}{\tilde{\psi}_0}}\ket{\tilde{\psi}_0} - \frac{\braket{\psi_2}{\tilde{\psi}_1}}{\braket{\tilde{\psi}_1}{\tilde{\psi}_1}}\ket{\tilde{\psi}_1}
	\end{align*}
	So as long as we can show the coefficients for the $\ket{\tilde{\psi}_i}$ terms are the same, we can conclude $U\ket{\phi_2} = \ket{\psi_2}$. The $\ket{\tilde{\psi}_0}$ is obviously the same because one can drop the tildes and use the fact that the vectors are normalized, and the inner products are pairwise equal. For the $\ket{\tilde{\psi}_1}$ term we need to compute some stuff.
	\begin{equation*}
		\braket{\phi_2}{\tilde{\phi}_1} = \overline{\alpha} - \overline{\alpha}^2
	\end{equation*}
	Here we've used $\alpha = \braket{\phi_0}{\phi_1}$ and the fact that the inner product is conjugated when flipped. It's important to note here that this calculation is exactly the same for $\braket{\psi_2}{\tilde{\psi}_1}$ because the inner products are equal (i.e. the indices match). The same logic and computation mean $\braket{\tilde{\phi}_1}{\tilde{\phi}_1} = \braket{\tilde{\psi}_1}{\tilde{\psi}_1}$. With this we conclude the problem solved for linearly independent sets of vectors.

	Now if we assume the sets $\{\phi_i\}, \{\psi_i\}$ are linearly dependent, then we only have to show the first two vectors land in the right spots, because if $\ket{\phi_2} = \alpha\ket{\phi_0} + \beta\ket{\phi_1}$, then $U\ket{\phi_2} = \alpha\ket{\psi_0} + \beta\ket{\psi_1} = \ket{\psi_2}$ which is the only possibly option for $\ket{\psi_2}$ in order to preserve the inner products. The problem is trivial for all three vectors linearly dependent because a simple rotation will take a line to a line.
\end{solution}

\end{document}
