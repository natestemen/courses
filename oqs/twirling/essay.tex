\documentclass[11pt,english]{article}

\usepackage[utf8]{inputenc}
\usepackage[T1]{fontenc}

\usepackage[
  margin=1in,
  headheight=14pt]{geometry}


\usepackage{mathpazo}
\usepackage{graphicx}
\usepackage{marginnote}
\usepackage{parskip}
\usepackage{microtype}
\usepackage{babel}
\usepackage{setspace}
\usepackage{lastpage}
\usepackage{fancyhdr}
\pagestyle{fancy}
\fancyhf{}
\lhead{Nathaniel Stemen}
\rhead{Open Quantum Systems}
\rfoot{\thepage/\pageref{LastPage}}
% \usepackage{varioref}
\usepackage[dvipsnames]{xcolor}
\usepackage[linktocpage]{hyperref}
\hypersetup{
  linktoc=all,
  colorlinks=true,
  linkcolor=MidnightBlue,
  filecolor=RubineRed,
  urlcolor=Bittersweet,
  citecolor=Fuchsia,
}
\usepackage{physics}
\usepackage{mathtools}
\usepackage{titletoc}
\usepackage{natbib}
\bibliographystyle{unsrtnat}
\usepackage[nottoc]{tocbibind}
\usepackage[noabbrev]{cleveref}
\usepackage{amsthm}
\theoremstyle{definition}
\newtheorem{definition}{Definition}[section]
\newtheorem{theorem}{Theorem}[section]

\usepackage{../macros}


\titlecontents{section}[1.8pc]
  {\addvspace{3pt}\bfseries}
  {\contentslabel[\thecontentslabel]{1.8pc}}
  {}
  {\quad\thecontentspage}

\titlecontents{subsection}[1.8pc]
  {\addvspace{1pt}\small}
  {\thecontentslabel\enspace{}}
  {}
  {\quad\thecontentspage}

\titlecontents{subsubsection}[3.2pc]
  {\addvspace{1pt}\small}
  {\thecontentslabel\enspace{}}
  {}
  {\quad\thecontentspage}


\begin{document}

\thispagestyle{empty}
\begin{center}
  \textsf{\textbf{\LARGE Twirling and Unitary Designs}}
\end{center}
\textbf{Name:} Nate Stemen (20906566) \hspace{\fill} \textbf{Due:} March 31, 2021 \\
\textbf{Email:} \href{mailto:nate@stemen.email}{\texttt{nate@stemen.email}} \hspace{\fill} \textbf{Course:} \textsc{AMATH} 876


\begingroup
\colorlet{darkblue}{MidnightBlue!10!black}
\hypersetup{linkcolor=darkblue}
\tableofcontents
\endgroup

\setstretch{1.1}
\vspace{0.5cm}

In these notes we work to understand the action of quantum twirling and it's implementation which leads naturally into the development of unitary $t$-designs.
Throughout we provide some of the mathematical detail needed to base our work on.

\section{Preliminaries}
It'll be helpful to have some concepts defined before we try and understand unitary designs in full.

\begin{definition}
  Let $P_1$ be the standard 1 qubit Pauli group generated by $X, Y, Z$ as seen in~\cite{lecture}.
  Define, in general, the ($n$ qubit) \textbf{Pauli Group} to be the $n$-fold tensor product of $P_1$ with itself.
  That is $P_n = P_1^{\otimes n}$.
  This naturally forms a group under matrix multiplication and can be shown to be generated by $X_i$ and $Z_i$ (where $i$ ranges to $n$).
\end{definition}

\begin{definition}\label{def:cliffords}
  The \textbf{Clifford group} is the set of unitary matrices\footnote{Equipped with the group operation of matrix multiplication} that leave the Pauli group unchanged upon conjugation:
  \begin{equation}
    C_n\defeq \qty{U\in\U{2^n} : UP_n U^\dagger = P_n}.
  \end{equation}
  Again, this naturally forms a group under matrix multiplication.
\end{definition}

\section{Twirling}

Let $\mathcal{H}_A$ and $\mathcal{H}_B$ be Hilbert spaces, and denote by $\mathrm{L}(\mathcal{H})$ by the set of linear operators on $\mathcal{H}$.
Given a quantum channel $\Lambda: \mathrm{L}(\mathcal{H}_A)\to\mathrm{L}(\mathcal{H}_B)$, then we can define twirling $\Lambda$ with respect to the unitary group as
\begin{equation}
  \mathcal{T}[\Lambda](\rho)\defeq \int_{\U{n}}U^\dagger\Lambda(U\rho U^\dagger)U\dd{\eta(U)}
\end{equation}
where $\eta$ is the unique unitary invariant measure on $\U{n}$ called the Haar measure.
In fact a Haar measure can be constructed on any locally compact topological group, but we will not go further than the unitary group here.
To better understand this integral we will have a quick aside on what the Haar measure is, and why it's necessary.

\subsection{\emph{The} Haar Measure}
The set of all ideal operations that one can perform on $n$ qubits---also known as the Unitary group $\U{n}$---has some very nice properties that allow us to define integration on this space.
In particular the group is \emph{compact} which can be seen from the fact that $UU^\dagger = \1$ for all $U\in\U{n}$ and hence the column vectors $U = \mqty[u_1 & \cdots & u_n]$ are orthonormal for all $U$.
Thus each $U$ is characterized by $n$ points on the surface of a hypersphere.
This implies, when embedded into $\R^{2n^2}$, $\U{n}$ is a space of finite volume, and by the usual Lebesgue integral in $\R^N$, we can integrate over $\U{n}$.
That said, this sense of measure inherited from $\R^N$ is not invariant under the action of an element $U\in\U{n}$.
To make it invariant, we use something similar to ``Weyl's Unitary Trick'' to average measure over the action of each element in $\U{n}$.
One can then show this is first of all invariant under unitary action\footnote{By both multiplication from the left and right.}, and is also the unique up to a complex number which explains why we say \emph{the} Haar measure.


\subsection{A simple example}
Suppose we have a 2-level quantum system represented by $\rho$.
We can then take the identity channel $\hat{I}(\rho) = \1\rho\1 = \rho$ and twirl it to see it's effect.
Here we denote superoperators with a hat $\hat{U}$ and their action is defined as $\hat{U}(\rho) = U\rho U^\dagger$.
\begin{equation}
  \mathcal{T}[\hat{I}](\rho) = \int_{\U{2}}U^\dagger\hat{I}(U\rho U^\dagger) U\dd{\eta(U)} = \int_{\U{2}}\rho\dd{\eta(U)}                                = \rho = \hat{I}(\rho)
\end{equation}

\subsection{Implementing Twirls}

As we've seen twirling quantum channels can yield lots of useful information about a channel and it's properties, but the question of implementing such a thing remains.
As modern quantum computers stand, doing an arbitrary unitary $U\in\U{n}$ is nearly impossible to do efficiently.
In~\cite{pseudo-random} the authors find an algorithm to implement a random unitary gate using $\order{n^2 2^{2n}}$ single and double qubit gates.
Not only is this exponential, but if we had to do this for all unitaries (or even an $\varepsilon$-net for that matter) we would need significantly more compute power than we have now.
This leads naturally to the question: can we find a finite set of unitaries that ``represent'' or ``simulate'' the entire unitary group?
If we could such a set that reproduced some of the statistical properties of $\U{n}$, then we might have a way to physically realize a twirl.
Luckily, mathematicians have been studying the objects we need for sometime, albeit in different contexts, in the form of designs.

\section{Designs}

The original concept of a design came in the form of spherical $t$-designs introducted by~\cite{spherical-designs}.
In particular the authors were interested in real polynomials whose average over a set of points in real Euclidean space was invariant under \emph{all} orthogonal transformations.
This notion was shown to be equivalent to sets of $N$ points $x_i\in\R^n$ such that for all real polynomials $f$ of degree $t$ we have
\begin{equation}
  \int_{S^{n-1}} f(x)\dd{\omega(x)} = \frac{1}{N}\sum_{i = 1}^n f(x_i)
\end{equation}
where $S^{n-1}\subset \R^n$ is the $(n-1)$-sphere, and $\dd{\omega}$ is an appropriated normalized measure on $S^{n-1}$.
In other words, we can find the average values of any polynomial on the sphere by looking at these particular $n$ points.

In~\cite{state-designs} the authors generalized this idea to polynomial functions operating on quantum states $\ket{\psi}$ instead of points in Euclidean space.
Again in~\cite{exact-approx-designs}, the idea is taken a step further to polynomial function of unitary operators $U$.
In particular~\cite{exact-approx-designs} gives the following definition.
\begin{definition}
  A \textbf{unitary $t$-design} (of dimension $n$) is a finite set $\qty{U_i}_{i = 1}^k\subset\U{n}$ such that for all polynomials $f_{(t, t)}(U)$ of degree $t$ in the matrix elements of $U$, and degree $t$ in the complex conjugate of those matrix elements we have
  \begin{equation}
    \int_{\U{n}}f_{(t,t)}(U)\dd{\eta(U)} = \frac{1}{k}\sum_{i = 1}^k f_{(t,t)}(U_i).
  \end{equation}
  Here $\eta$ represents the unique unitary invariant measure called the Haar measure on $\U{n}$.
\end{definition}

A particularly interesting example is $t=2$-designs where we have the following equivalent characterziation due to~\cite{even-unitaries}.
\begin{theorem}
  A set $\qty{U_i}_{i=1}^k$ is a unitary 2-design if and only if for any quantum channel $\Lambda$ we have
  \begin{equation}
    \frac{1}{k}\sum_{i=1}^k U_k^\dagger \Lambda(U_k\rho U_k^\dagger)U_k = \int_{\U{n}}U^\dagger\Lambda(U\rho U^\dagger)U\dd{\eta(U)}
  \end{equation}
\end{theorem}
This result gives us a much more manageable way to implement a twirling operation since we no longer have to try to implement every possible unitary.
The next natural question to ask then is do we know of any 2-designs?
Thankfully~\cite{cliffords} showed that in fact the Clifford group defined in~\cref{def:cliffords} is a 3-design.
Well a 3-design is not a 2-design, but thankfully if we have a $t$-design we have \emph{for free} an $s$-design for an $s\leq t$ by the fact that a degree $s$ polynomial can always be a degree $t$ polynomial with 0 as coefficients for any degree $n > s$.

The Clifford group is an important group in quantum information theory not just because it normalizes the Pauli group, but as shown in~\cite{gottesman-knill}, circuits using only the Clifford group can be efficiently simulated on a classical computer.
Clifford circuits have also been shown to to be in a much smaller computational complexity class ($\oplus\mathsf{L}$) than the full power of quantum computers ($\mathsf{BQP}$)~\cite{aaronson}.\footnote{Although, like many things in theoretical computer science, they have not been shown to be completely distinct classes, even though it is believed by many in the field.}
These facts demonstrate that even though the Clifford group ``represents'' the entire unitary group in some statistical ways, it is not strong enough to do universal quantum computation.

\clearpage

\bibliography{refs}

\end{document}