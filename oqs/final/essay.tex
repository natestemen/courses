\documentclass[11pt,english]{article}

\usepackage[utf8]{inputenc}
\usepackage[T1]{fontenc}

\usepackage[
  margin=1.3in,
  headheight=14pt]{geometry}


\usepackage{mathpazo}
\usepackage{graphicx}
\usepackage{marginnote}
\usepackage{parskip}
\usepackage{microtype}
\usepackage{babel}
\usepackage{setspace}
\usepackage{lastpage}
\usepackage{fancyhdr}
\pagestyle{fancy}
\fancyhf{}
\lhead{Nathaniel Stemen}
\rhead{Open Quantum Systems}
\rfoot{\thepage/\pageref{LastPage}}
% \usepackage{varioref}
\usepackage[dvipsnames]{xcolor}
\usepackage[linktocpage]{hyperref}
\hypersetup{
  linktoc=all,
  colorlinks=true,
  linkcolor=MidnightBlue,
  filecolor=RubineRed,
  urlcolor=Bittersweet,
  citecolor=Fuchsia,
}
\usepackage{physics}
\usepackage{mathtools}
\usepackage{titletoc}
\usepackage{natbib}
\bibliographystyle{unsrtnat}
\usepackage[nottoc]{tocbibind}
\usepackage[noabbrev]{cleveref}
\usepackage{amsthm}
\theoremstyle{definition}
\newtheorem{definition}{Definition}[section]
\newtheorem{example}{Example}[section]

\usepackage{../macros}


\titlecontents{section}[1.8pc]
  {\addvspace{3pt}\bfseries}
  {\contentslabel[\thecontentslabel]{1.8pc}}
  {}
  {\quad\thecontentspage}

\titlecontents{subsection}[1.8pc]
  {\addvspace{1pt}\small}
  {\thecontentslabel\enspace{}}
  {}
  {\quad\thecontentspage}

\titlecontents{subsubsection}[3.2pc]
  {\addvspace{1pt}\small}
  {\thecontentslabel\enspace{}}
  {}
  {\quad\thecontentspage}


\begin{document}

\thispagestyle{plain}
\begin{center}
  \textsf{\textbf{\LARGE Twirling and Unitary Designs}}
\end{center}
\textbf{Name:} Nate Stemen (20906566) \hspace{\fill} \textbf{Due:} March 31, 2021 \\
\textbf{Email:} \href{mailto:nate@stemen.email}{\texttt{nate@stemen.email}} \hspace{\fill} \textbf{Course:} \textsc{AMATH} 876


\begingroup
\colorlet{darkblue}{MidnightBlue!10!black}
\hypersetup{linkcolor=darkblue}
\tableofcontents
\endgroup

\setstretch{1.1}
\vspace{0.5cm}

\section{Preliminaries}
It'll be helpful to have some concepts defined before we try and understand unitary designs in full.

\begin{definition}
  Take the following basis for $\U{2}$\footnote{The unitary group of $2\times 2$ matrices. In general we have $\U{n}\defeq \qty{U\in\mats{n}{\C} : UU^\dagger = \1}$ with $\dim\U{n} = n^2$.}
  \begin{align}
    \1 & = \mqty[1 & 0 \\ 0 & 1] & X & = \mqty[0 & 1 \\ 1 & 0] & Y & = \mqty[0 & -\iu \\ \iu & \phantom{-}0] & Z & = \mqty[1 & \phantom{-}0 \\ 0 & -1]
  \end{align}
  and define
  \begin{equation}
    A_i\defeq \1\otimes \1 \cdots \1 \otimes \!\!\!\!\!\!\overbrace{A}^{i\text{th operator}}\!\!\!\!\!\! \otimes \1 \cdots \1 \otimes \1
  \end{equation}
  to be the action of $A\in \qty{\1, X, Y, Z}$ on only the $i$th qubit, and identity elsewhere. From here we define the \textbf{Pauli Group} as
  \begin{equation}
    P_n\defeq \qty{\pm\1, \pm\iu\1, \pm A_i, \pm\iu A_i : A_i\in\qty{X_i, Y_i, Z_i}, i\in\Z_n}.
  \end{equation}
  This naturally forms a group under matrix multiplication and can be shown to be generated by $X_i$ and $Z_i$ (where $i$ ranges to $n$).
\end{definition}

\begin{definition}
  The \textbf{Clifford group} is the set of unitary matrices\footnote{equipped with the group operation of matrix multiplication} that leave the Pauli group unchanged upon conjugation:
  \begin{equation}
    C_n\defeq \qty{U\in\U{2^n} : UP_n U^\dagger = P_n}.
  \end{equation}
  Again, this naturally forms a group under matrix multiplication.
\end{definition}


\section{Designs}

The original concept of a design came in the form of spherical $t$-designs introducted by~\cite{spherical-designs}.
In particular the authors were interested in real polynomials whose average over a set of points in real Euclidean space was invariant under \emph{all} orthogonal transformations.
This notion was shown to be equivalent to sets of $N$ points $x_i\in\R^n$ such that for all real polynomials $f$ of degree $t$ we have
\begin{equation}
  \int_{S^{n-1}} f(\chi)\dd{\omega(\chi)} = \frac{1}{N}\sum_{i = 1}^n f(x_i)
\end{equation}
where $S^{n-1}\subset \R^n$ is the $(n-1)$-sphere, and $\dd{\omega}$ is an appropriated normalized measure on $S^{n-1}$.
In other words, we can find the average values of any polynomial on the sphere by looking at these particular $n$ points.


\nocite{*}

\clearpage

\bibliography{refs}

\end{document}