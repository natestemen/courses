\documentclass[
	pages,
	boxes,
	color=RoyalPurple
]{homework}


\usepackage{macros}
\usepackage[footnote,nolist,smaller]{acronym}
\usepackage{cleveref}
\name{Nate Stemen}
\studentid{20906566}
\email{nate@stemen.email}
\term{Fall 2021}
\course{General Relativity for Cosmology}
\courseid{AMATH 875}

\hwname{Lecture}
\hwnum{2}
\duedate{Thur, Apr 8, 2020 11:59 PM}

\colorlet{Ppink}{WildStrawberry!20}
\newtcbox{\mafbox}[1][]{on line, math upper,
boxsep=4pt, left=0pt,right=0pt,top=0pt,bottom=0pt,
colframe=white,colback=Ppink,
highlight math style={enhanced}}


\makeatletter
\numberwithin{tcb@cnt@prob}{section}
\makeatother


\begin{document}

\setcounter{section}{11}
% \problemnumber{2}

\begin{problem}
Prove the Kraus representation theorem in the case $\mathcal{H} = \C^D$.
\end{problem}

\begin{solution}
\end{solution}

\begin{problem}
Show that there is a unitary freedom in choosing the set of Kraus operators associated with any fixed unitary acting on an extended Hilbert space. This is connected with the freedom one has in assigning a state to the auxiliary system in the extended Hilbert space.
\end{problem}

\begin{solution}
\end{solution}

\problemnumber{4}
\begin{problem}
Generalize the above example of two successive $U_\text{CNOT}$ operations to show that the second transformation can not be modelled by a linear map even when the input state (at time $t = 1$) has no entanglement between systems $A$ and $B$.
\end{problem}

\begin{solution}
\end{solution}

\begin{problem}
Find upper and lower bounds on $p$ from the requirement that the depolarizing channel is a CPTP map.
\end{problem}

\begin{solution}
\end{solution}

\setcounter{section}{12}
\problemnumber{2}
\begin{problem}
Determine the full freedom in defining sets of measurement operators $M_{k,j}$ associated with a given POVM $\qty{E_k}$. For example, is the unitary $V$ on $\mathcal{H}_A$ (defined above) the most general map on the output state that is consistent with the Born rule probabilities?
\end{problem}

\begin{solution}
\end{solution}

\setcounter{section}{15}
\problemnumber{1}
\begin{problem}
Assuming that $B_1 = \1_D / \sqrt{D}$ where $\alpha\in\qty{1,\ldots,D^2}$, prove that for any trace-preserving map $\mathcal{E}$ one has $t(\mathcal{E}) = 1$ and $\va{m} = \va{0}$, and, if $\mathcal{E}$ is a unital map, then $t(\mathcal{E}) = 1$ and $\va{m} = \va{n} = \va{0}$.
\end{problem}

\begin{solution}
\end{solution}

\end{document}