\documentclass[
	pages,
	boxes,
	color=RoyalPurple
]{homework}


\usepackage{macros}
\usepackage{cleveref}
\usepackage{tikz}
\usetikzlibrary{quantikz,trees,arrows}
\name{Nate Stemen}
\studentid{20906566}
\email{nate@stemen.email}
\term{Fall 2021}
\course{General Relativity for Cosmology}
\courseid{AMATH 875}

\hwname{Lecture}
\hwnum{2}
\duedate{Thur, Apr 8, 2020 11:59 PM}

\colorlet{Ppink}{WildStrawberry!20}
\newtcbox{\mafbox}[1][]{on line, math upper,
boxsep=4pt, left=0pt,right=0pt,top=0pt,bottom=0pt,
colframe=white,colback=Ppink,
highlight math style={enhanced}}


\makeatletter
\numberwithin{tcb@cnt@prob}{section}
\makeatother


\begin{document}

\setcounter{section}{11}

\begin{problem}
Prove the Kraus representation theorem in the case $\mathcal{H} = \C^n$.
\end{problem}

\begin{solution}
    Let $\mathcal{E}: \End{\mathcal{H}}\to\End{\mathcal{H}}$ be our channel and assume it has the following operator-sum decomposition
    \begin{equation}\label{eq:opsum}
        \mathcal{E}(\rho) = \sum_k A_k\rho A_k^\dagger
    \end{equation}
    with $\sum_k A_k^\dagger A_k = \1$. Clearly~\cref{eq:opsum} is linear and it's trace preserving because
    \begin{equation*}
        \tr(\mathcal{E}(\rho)) = \sum_k\tr(A_k\rho A_k^\dagger) = \sum_k \tr(A_k^\dagger A_k\rho) = \tr(\qty[\sum_k A_k^\dagger A_k]\rho) = \tr(\rho).
    \end{equation*}
    To see~\cref{eq:opsum} is completely positive consider
    \begin{align*}
        \qty(\id_A\otimes\mathcal{E})(\rho_{A\mathcal{H}}) & = \sum_k\qty(\1_A\otimes A_k)\rho_{A\mathcal{H}}\qty(\1_A\otimes A_k^\dagger)  \\
                                                           & = \sum_k \qty(\1_A\otimes A_k)\rho_{A\mathcal{H}}\qty(\1_A\otimes A_k)^\dagger
    \end{align*}
    Now since $\qty(\1_A\otimes A_k)\rho_{A\mathcal{H}}\qty(\1_A\otimes A_k)^\dagger \geq 0$ for all $k$ %the same must be true for the sum. This also holds for all systems $A$, and hence the map defined in~\cref{eq:opsum} is a CPTP map.

    To go in the other direction again take $\mathcal{E}$ to be our CPTP map. Since the space of endomorphisms of $\mathcal{H}$ is isomorphic to the space of $n\times n$ matrices $\End{\mathcal{H}}\cong\mats{n}{\C}$ which is in term isomorphic to $\C^{n^2}$ we can treat $\mathcal{E}$ as an $n^2\times n^2$ matrix acting on $\vect{\rho}\in\C^{n^2}$ where $\rho\in\mats{n}{\C}$. From here I'm not totally sure where to go. I would think we should decompose the matrix representaiton of $\mathcal{E}$ using some sort of rank decomposition, but I have no idea how to start getting terms that looks like $A\rho A^\dagger$.
\end{solution}

\begin{problem}
Show that there is a unitary freedom in choosing the set of Kraus operators associated with any fixed unitary acting on an extended Hilbert space. This is connected with the freedom one has in assigning a state to the auxiliary system in the extended Hilbert space.
\end{problem}

\begin{solution}
    When deriving the Kraus operator representation we took the evolution of system $A$ to be
    \begin{equation*}
        \rho_A\to\rho_A' = \tr_B(\mathcal{E}(\rho_{AB})) = \tr_B(\hat{U}\rho_A\otimes\rho_B\hat{U}^\dagger)
    \end{equation*}
    and then we explicitly calculated that value in a chosen basis of $\mathcal{H}_B$. That said, there are many bases we could chose to calculate the trace over, and in particular any one is related to another by some unitary operatory $V$. Repeating the calculation in a new basis $\qty{\ket{b_i}}$ which is related to the old basis by
    \begin{equation*}
        \ket{a_i} = \sum_j V_{ij}\ket{b_j}.
    \end{equation*}
    We can now calculate new Kraus operators with respect to this basis. In what follows I use $\1\otimes\ket{x}$ in full pedantry because the notation in the book thoroughly confused me.
    \begin{align*}
        \rho_A' & = \sum_i \1\otimes\bra{b_i}U\,\rho_A\otimes\qty[\sum_{j,k,l}\lambda_j V_{jk}\overline{V_{jl}}\ketbra{b_k}{b_l}]U^\dagger \, \1\otimes \ket{b_i}                                              \\
                & = \sum_{i,j}\qty[\sqrt{\lambda_j}\sum_k V_{jk}\1\otimes\bra{b_i}U\, \1\otimes\ket{b_k}]\rho_A\qty[\sqrt{\lambda_j}\sum_{l}\overline{V_{jl}}\1\otimes\bra{b_l}U^\dagger\, \1\otimes\ket{b_i}] \\
                & = \sum_{i,j}\hat{B}_{ij}\rho_A\hat{B}_{ij}^\dagger
    \end{align*}
    Where I've defined $\hat{B}_{ij}\defeq \sqrt{\lambda_j}\sum_{k}V_{jk}\1\otimes \bra{b_i}U\,\1\otimes\ket{b_k}$ whereas in the book/notes we have $\hat{A}_{ij} \defeq\sqrt{\lambda_j}\1\otimes\bra{a_i} U \, \1\otimes\ket{a_j}$. I'll now show these are related by the unitary $V$.
    \begin{align*}
        \hat{A}_{ij} & = \sqrt{\lambda_j}\1\otimes\bra{a_i} U \, \1\otimes\ket{a_j}                                                                               \\
                     & = \sqrt{\lambda_j}\sum_{k, l}\overline{V_{ik}}V_{jl}\1\otimes\bra{b_k}U\,\1\otimes\ket{b_l}                                                \\
                     & = \sum_k\overline{V_{ik}}\qty[\sqrt{\lambda_j}\sum_l V_{jl}\1\otimes\bra{b_k}U\,\1\otimes \ket{b_l}] = \sum_k\overline{V_{ik}}\hat{B}_{kj}
    \end{align*}
    Thus $\hat{A}_{ij}$ are related to the the $\hat{B}_{kl}$ via a unitary transformation.
\end{solution}

\problemnumber{4}
\begin{problem}
Generalize the above example of two successive $U_\text{CNOT}$ operations to show that the second transformation can not be modelled by a linear map even when the input state (at time $t = 1$) has no entanglement between systems $A$ and $B$.
\end{problem}

\begin{solution}
    We first lay out the circuit we're working with and the associated states.
    \begin{center}
        \begin{quantikz}
            \slice{$\ket{\psi_0}$} & \qw & \ctrl{1} & \qw\slice{$\ket{\psi_1}$} & \qw & \ctrl{1} & \qw\slice{$\ket{\psi_2}$} & \qw  \\[.5cm]
            & \qw & \targ{}  & \qw                       & \qw & \targ{}  & \qw                       & \qw
        \end{quantikz}
    \end{center}
    Take $\ket{\psi_0} = \qty(a\ket{0} + b\ket{1})\otimes\qty(c\ket{0} + d\ket{1})$ and we first apply a CNOT. % chktex 13
    \begin{align*}
        \ket{\psi_1} & = \hat{U}_\text{CNOT}\ket{\psi_0} = ac\ket{00} + ad\ket{01} + bc\ket{10} + bd\ket{11} \\
                     & = a\ket{0}\otimes(c\ket{0} + d\ket{1}) + b\ket{1}\otimes(c\ket{1} + d\ket{0})
    \end{align*}
    Since $\hat{U}_\text{CNOT}^2 = \1$ we know the final state $\ket{\psi_2}$ is $\ket{\psi_0}$. Now if we look at the evolution of system $B$ alone we have
    \begin{center}
        \tikzstyle{level 1}=[level distance=30mm, sibling distance=30mm]
        \tikzstyle{level 2}=[level distance=40mm]
        \begin{tikzpicture}[grow=right,->,>=angle 60]
            \node(id){$c\ket{0} + d\ket{1}$}
            child {node {$c\ket{1} + d\ket{0}$}
                    child {node{$c\ket{0} + d\ket{1}$}}
                }
            child {node {$c\ket{0} + d\ket{1}$}
                    child {node{$c\ket{0} + d\ket{1}$}}
                };
        \end{tikzpicture}
    \end{center}
    Not only is the first evolution under a CNOT not even a function, but the second CNOT acts as a swap on one qubit, but the identity on another. If there was a linear map that performed this $N$, then by the top equation $N(c\ket{0} + d\ket{1}) = cN\ket{0} + dN\ket{1} = c\ket{0} + d\ket{1}$ and hence $N$ is the identiy operator $\1$. The second equation, however shows $N(c\ket{1} + d\ket{0}) = cN\ket{1} + dN\ket{0} = c\ket{0} + d\ket{1}$ and hence is the swap. These two operators are clearly not compatible, so it is not possible.

    I'm not sure I've really done what was asked of me here, because of how similar the argument is to in the notes, but the question is somewhat ambiguous.
\end{solution}

\begin{problem}
Find upper and lower bounds on $p$ from the requirement that the depolarizing channel is a CPTP map.
\end{problem}

\begin{solution}
    The Kraus representation theorem tells us that a channel is a CPTP map if and only if it admits an operator sum decomposition which we found in class with the following Kraus operators.
    \begin{align*}
        \hat{A}_{0,0} & = \qty(1 - p + \frac{p}{D^2})^{1/2}\1 & \hat{A}_{a,b} & = \frac{\sqrt{p}}{D}X^a Z^b
    \end{align*}
    To derive bounds on $p$ we need to ensure this \emph{is} a CPTP map and so it must satisfy the completeness relation.
    \begin{align*}
        \1  = \sum_{a, b = 0}^{D-1}\hat{A}_{a,b}^\dagger\hat{A}_{a,b} & = \qty(1 - p + \frac{p  }{D^2}) \1 + \frac{p}{D^2}\sum_{a, b = 1}^{D-1}\qty(X^a Z^b)^\dagger X^a Z^b %         \\
        %   & = \abs{1 - p + \frac{p  }{D^2}} \1 + \frac{\abs{p}}{D^2}\sum_{a, b = 1}^{D-1}{Z^\dagger}^b {X^\dagger}^{a} X^a Z^b \\
        %   & = \abs{1 - p + \frac{p  }{D^2}} \1 + \frac{\abs{p}}{D^2}(D - 1)\1
    \end{align*}
    Even without going any further we know that each term must have a positive coefficient since $A^\dagger A \geq 0$ for any operator $A$. The $a, b \neq 0$ term tells us $p \geq 0$, and the first is slightly more complicated.
    \begin{align*}
        1 - p + \frac{p}{D^2}    & \geq 0                                           \\
        p\qty(\frac{1}{D^2} - 1) & \geq -1                                          \\
        p                        & \leq\frac{-1}{\frac{1}{D^2} - 1}                 \\
        p                        & \leq \frac{D^2}{D^2 - 1} = 1 + \frac{1}{D^2 - 1}
    \end{align*}
    Thus in order for the depolarizing channel to be a CPTP map we must have
    \begin{equation*}
        0\leq p \leq 1 + \frac{1}{D^2 - 1}.
    \end{equation*}
\end{solution}

\setcounter{section}{12}
\problemnumber{2}
\begin{problem}
Determine the full freedom in defining sets of measurement operators $M_{k,j}$ associated with a given POVM $\qty{E_k}$. For example, is the unitary $V$ on $\mathcal{H}_A$ (defined above) the most general map on the output state that is consistent with the Born rule probabilities?
\end{problem}

\begin{solution}
    Suppose we transform the $M_{k, j}$ via a CPTP map. By Kraus's representation theorem this can be done with an operator sum decomposition as
    \begin{equation*}
        \widetilde{M}_{k, j} = \sum_{i}A_i M_{k, j} A_i^\dagger.
    \end{equation*}
    In order to preserve the Born rule we must leave $\Pr(k)$ invariant so let's check that.
    \begin{align*}
        \Pr(k) & = \tr(E_k\rho)                                                                                         \\
               & = \sum_{j}\tr(\widetilde{M}_{k, j}\rho\widetilde{M}_{k, j}^\dagger)                                    \\
               & = \sum_{j}\tr(\qty[\sum_{i}A_i M_{k, j}A_i^\dagger]\rho\qty[\sum_{l}A_l M_{k, j}^\dagger A_l^\dagger]) \\
               & = \sum_{i, j, l}\tr(A_i M_{k, j}A_i^\dagger\rho A_l M_{k, j}A_l^\dagger)
    \end{align*}
    This only reduces to $\sum_j \tr(M_{k, j}\rho M_{k, j}^\dagger)$ in the case where there is only one non-zero Kraus operator, which must be unitary. Thus we cannot use an arbitrary CPTP map. I know this doesn't prove we cannot use other linear/nonlinear maps, but I cannot figure out how to show we \emph{can} do those things.
\end{solution}

\setcounter{section}{15}
\problemnumber{1}
\begin{problem}
Assuming that $B_1 = \1_D / \sqrt{D}$ where $\alpha\in\qty{1,\ldots,D^2}$, prove that for any trace-preserving (TP) map $\mathcal{E}$ one has $t(\mathcal{E}) = 1$ and $\va{m} = \va{0}$, and, if $\mathcal{E}$ is a unital map, then $t(\mathcal{E}) = 1$ and $\va{m} = \va{n} = \va{0}$.
\end{problem}

\begin{solution}
    To calculate $t(\mathcal{E})$ or $\mathcal{E}_{1,1}$ which we can calculate with $\mathcal{E}_{\alpha\beta} = \tr(B_\alpha^\dagger\mathcal{E}(B_\beta))$.
    \begin{equation*}
        t(\mathcal{E}) = \mathcal{E}_{11} = \frac{1}{D}\tr(\1_D\mathcal{E}(\1_D)) = \frac{1}{D} \tr(\mathcal{E}(\1_D)) \stackrel{\text{(TP)}}{=} \frac{1}{D}\tr(\1_D) = 1
    \end{equation*}
    Now we want to show $\va{m} = \mathcal{E}_{1\alpha} = \va{0}$ for $\alpha > 1$.
    \begin{equation*}
        \mathcal{E}_{1\alpha} = \frac{1}{\sqrt{D}}\tr(B_1^\dagger\mathcal{E}(P_\alpha)) = \frac{1}{D}\tr(\mathcal{E}(P_\alpha)) \stackrel{\text{(TP)}}{=} \frac{1}{D}\tr(P_\alpha) = 0
    \end{equation*}
    Using the fact that all the Pauli's are traceless, and hence so are the tensor products.

    Now take $\mathcal{E}$ to be unital in addition to trace preserving, and hence we must show $\mathcal{E}_{\alpha 1} = \va{n} = \va{0}$. Here we will use the fact that a unital map takes the identity channel to the identity channel: $\mathcal{E}(\1) = \1$.
    \begin{equation*}
        \mathcal{E}_{\alpha 1} = \frac{1}{D}\tr(P_\alpha^\dagger\mathcal{E}(\1_D)) = \frac{1}{D}\tr(P_\alpha\1_D) = \frac{1}{D}\tr(P_\alpha) = 0
    \end{equation*}
    Thus when $\mathcal{E}$ is both trace preserving and unital it's natural representatation takes the following block form.
    \begin{equation*}
        \mqty[0 & \va{0}^\intercal \\ \va{0} & \vb{R}(\mathcal{E})]
    \end{equation*}
\end{solution}

\end{document}