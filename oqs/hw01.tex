\documentclass[
	pages,
	boxes,
	color=RoyalPurple
]{homework}


\usepackage{macros}
\usepackage[footnote,nolist,smaller]{acronym}
\usepackage{cleveref}
\usepackage{todonotes}
\name{Nate Stemen}
\studentid{20906566}
\email{nate@stemen.email}
\term{Fall 2021}
\course{General Relativity for Cosmology}
\courseid{AMATH 875}

\hwname{Lecture}
\hwnum{1}
\duedate{Mon, Feb 22, 2020 11:59 PM}

\colorlet{Ppink}{WildStrawberry!20}
\newtcbox{\mafbox}[1][]{on line, math upper,
boxsep=4pt, left=0pt,right=0pt,top=0pt,bottom=0pt,
colframe=white,colback=Ppink,
highlight math style={enhanced}}


\makeatletter
\numberwithin{tcb@cnt@prob}{section}
\makeatother


\begin{document}

\begin{acronym}
    \acro{PVM}{Projection-Valued Measure}
    \acro{POVM}{Positive Operator-Valued Measure}
\end{acronym}

\setcounter{section}{2}
\problemnumber{2}

\begin{problem}
Show that the reduced state obtained via partial trace is a density operator, i.e., a non-negative operator satisfying $\tr\hat{\rho}_A = 1$.
\end{problem}

\begin{solution}
    Take $\hat{\rho}_{AB}$ to be a density operator on Hilbert space $\mathcal{H}_{AB}$. In a (tensor product) basis this looks like
    \begin{equation*}
        \hat{\rho}_{AB} = \sum_{i,j,k,\ell}p_{ijk\ell}\ket{a_i}\otimes\ket*{b_j}\bra{a_k}\otimes\bra{b_\ell}
    \end{equation*}
    In order to get a condition on it's coefficients $p_{ijk\ell}$ we'll take it's trace and force it to be 1.
    \begin{align}
        \tr\hat{\rho}_{AB} & = \sum_{n, m}\bra{a_n}\otimes\bra{b_m}\qty\Big[\hat{\rho}_{AB}]\ket{a_n}\otimes\ket{b_m} \nonumber        \\
                           & = \sum_{n, m, i, j, k, \ell}p_{ijk\ell}\,\delta_{in}\,\delta_{jm}\,\delta_{kn}\,\delta_{\ell m} \nonumber \\
                           & = \sum_{i, j}p_{ijij} = 1 \tag{$*$}\label{eq:pcond}
    \end{align}
    We'll start by computing $\rho_{A}$ in this basis.
    \begin{align*}
        \tr_B\hat{\rho}_{AB} & = \sum_{n}\1\otimes\bra{b_n}\Big[\hat{\rho}_{AB}\Big]\, \1\otimes\ket{b_n}          \\
                             & = \sum_{n, i, j, k, \ell}p_{ijk\ell}\,\delta_{nj}\,\delta_{\ell n}\ketbra{a_i}{a_k} \\
                             & = \sum_{n, i, k}p_{inkn}\ketbra{a_i}{a_k}
    \end{align*}
    Now we can ensure $\tr\hat{\rho}_A = 1$.
    \begin{align*}
        \tr\hat{\rho}_A & = \sum_\ell \bra{a_\ell}\Big[\hat{\rho}_A\Big]\,\ket{a_\ell}    \\
                        & = \sum_{\ell, n, i, k}p_{inkn}\,\delta_{\ell i}\,\delta_{k\ell} \\
                        & = \sum_{\ell, n}p_{\ell n \ell n}
    \end{align*}
    Hence by \cref{eq:pcond} we can conclude $\tr\hat{\rho}_A = 1$.

    To show $\hat{\rho}_A$ is positive semi-definite we can use the fact that the sum of the diagonal terms of a density operator are both real and positive. Real because density operators are Hermitian and positive because they are probabilities.
    \begin{align*}
        \expval{\hat{\rho}_A}{\phi} & = \sum \overline{\phi}_n \bra{a_n}p_{ijkj}\ketbra{a_i}{a_k}\phi_n\ket{a_n}      \\
                                    & = \sum\abs{\phi_n}^2 p_{ijkj}\,\delta_{ni}\,\delta_{kn}                         \\
                                    & = \sum_{ij}\underbrace{\abs{\phi_i}^2}_{\in[0,1]} \underbrace{p_{ijij}}_{\in\R} \\
                                    & \geq \sum_{ij}p_{ijij} = 1 \geq 0
    \end{align*}

    Hence we've shown the partial trace $\tr_B:\mathrm{L}(H_A\otimes H_B)\to \mathrm{L}(H_A)$ preserves density operators.
\end{solution}

\begin{problem}
Prove that these three pure-state conditions are equivalent.
\end{problem}

\begin{solution}
    First we'll show $\mafbox{\hat{\rho}^2 = \hat{\rho} \implies \tr\hat{\rho}^2 = 1}$.
    \begin{align*}
        \tr\hat{\rho}^2 = \tr\hat{\rho} = 1
    \end{align*}
    Next we have $\mafbox{\tr\hat{\rho}^2 = 1 \implies \hat{\rho} = \ketbra{\psi}}$.
    Let's first calculate $\hat{\rho}^2$ in an othormal basis.
    \begin{align*}
        \hat{\rho}^2 & = \qty(\sum_{i = 1}^n p_i\ketbra{i}{i})^2 = \sum_{i = 1}^n p_i\ketbra{i}{i}\sum_{j = 1}^n p_j\ketbra{j}{j} \\
                     & = \sum_{i,j = 1}^n p_ip_j\ket{i}\braket{i}{j}\bra{j} \tag*{$\mafbox{\braket{i}{j} = \delta_{ij}}$}         \\
                     & = \sum_{i = 1}^n p_i^2\ketbra{i}{i}
    \end{align*}
    Now taking the trace of $\hat{\rho}^2$ we get a condition on the $p_i$'s.
    \begin{align*}
        \tr \hat{\rho}^2 & = \sum_{k = 1}^n\bra{k}\hat{\rho}^2\ket{k}               \\
                         & = \sum_{k,i = 1}^n\bra{k}\qty(p_i^2\ketbra{i}{i})\ket{k} \\
                         & = \sum_{i = 1}^n p_i^2 = 1
    \end{align*}
    So $\sum p_i = 1 = \sum p_i^2$. By basic properties of real numbers, we know $p_i^2 \leq p_i$ when $p_i\in[0, 1]$ and equality only holding when $p_i \in\{0, 1\}$. Using this fact we can write
    \begin{equation*}
        \sum_{i = 1}^n p_i^2 \leq \sum_{i = 1}^n p_i
    \end{equation*}
    where again equality only holds if $p_i\in\{0, 1\}$ for all $i$. The only way this can be true is if one of the $p_i$'s is 1 and all of the rest are 0. In that case our summation collapses to one term, and we are left with $\hat{\rho} = \ketbra{i}$ as desired.

    Lastly we have $\mafbox{\hat{\rho} = \ketbra{\psi} \implies \hat{\rho}^2 = \hat{\rho}}$.
    \begin{align*}
        \hat{\rho}^2 = \qty(\ketbra{\psi})\qty(\ketbra{\psi}) = \ket{\psi}\underbrace{\braket{\psi}}_{1}\bra{\psi} = \ketbra{\psi} = \hat{\rho}
    \end{align*}

\end{solution}

\problemnumber{5}

\begin{problem}
Prove the existence of the Schmidt decomposition.
\end{problem}

\begin{solution}
    Let $\{\ket{a_i}\}$ be an orthonormal basis for $H_A$ and $\{\ket{b_i}\}$ be an orthonormal basis for $H_B$. We then know $\ket{a_n}\otimes \ket{b_m}$ forms a basis for $H_{AB} = H_A\otimes H_B$, and hence we can expand any vector $\ket{\psi}\in H_{AB}$ as
    \begin{equation*}
        \ket{\psi} = \sum_{ij}\psi_{ij}\ket{a_i}\otimes \ket*{b_j}.
    \end{equation*}
    We can think of the coefficients $\psi_{ij}$ as a matrix using the association $\ket{a_i}\otimes\ket*{b_j}\cong \ketbra*{a_i}{b_j}$.\footnote{Using the underlying isomorphism of $U^*\otimes V$ and the space $\Hom(U, V)$ of linear maps from $U$ to $V$.} This operator is then taking an element of $H_B$ to an element of $H_A$. That is we can think of the \emph{vector} $\ket{\psi}$ as a linear map $\ket{\psi}_\mathrm{op}: H_B\to H_A$.

    With this picture in place we can apply the singular decomposition to write
    \begin{equation*}
        \ket{\psi}_\mathrm{op} = \sum_{n = 1}^r\lambda_n\ketbra{a_n}{b_n}
    \end{equation*}
    where $r$ is the rank of $\ket{\psi}_\mathrm{op}$ the \emph{operator}. Now we can map back into $\ket{\psi}$ again using $\ketbra{a_n}{b_m}\cong \ket{a_n}\otimes \ket{b_m}$ and see we have
    \begin{equation*}
        \ket{\psi} = \sum_{n = 1}^r\lambda_n\ket{a_n}\otimes \ket{b_n}.
    \end{equation*}
\end{solution}

\setcounter{section}{3}
\problemnumber{1}

\begin{problem}
Prove the two properties given by Eqns. 3.1.
\end{problem}

\begin{solution}
    First we'll show $\sum_\nu E_\nu = \1$.
    \begin{align*}
        \sum_\nu E_\nu & = \sum_\nu \tr_B\qty(\Pi_\nu\cdot \1_A\otimes \hat{\rho}_B)                                                           \\
                       & = \tr_B\!\qty(\sum_\nu\Pi_\nu\cdot \1_A\otimes\hat{\rho}_B)                                                           \\
                       & = \tr_B\!\qty(\qty[\sum_\nu\Pi_\nu]\cdot \1_A\otimes\hat{\rho}_B) \tag*{$\mafbox{\sum_\nu\Pi_\nu = \1_{A\otimes B}}$} \\
                       & = \tr_B\!\qty(\1_A\otimes\hat{\rho}_B)                                                                                \\
                       & = \1_A
    \end{align*}
    Now we'll show $E_\nu\geq 0$, but first we need some setup. Since $\Pi_\nu$ is a projector, by the spectral theorem it can be written as
    \begin{equation*}
        \Pi_\nu = \sum_{ij}\lambda_{ij}\ketbra{a_i b_j}
    \end{equation*}
    where $\lambda_{ij}$ are real and non-negative. We also have $\hat{\rho}_B = \sum p_n\ketbra{b_n}$ and $\1_A = \sum\ketbra{a_n}$.
    \begin{align*}
        \Pi_\nu\cdot \1_A\otimes \hat{\rho}_B & = \sum \lambda_{ij} p_n\ket{a_i b_j}\braket{a_i b_j}{a_\ell b_n}\bra{a_\ell b_n} \\
                                              & = \sum\lambda_{ij}p_n\,\delta_{i\ell}\,\delta_{jn}\ketbra{a_i b_j}{a_\ell b_n}   \\
                                              & = \sum_{\ell n}\lambda_{\ell n}p_n\ketbra{a_\ell b_n}
    \end{align*}
    Now we can take the partial trace over $B$.
    \begin{align*}
        \tr_B\qty(\Pi_\nu\cdot \1_A\otimes\hat{\rho}_B) & = \sum \1_A\otimes\bra{b_i}\lambda_{\ell n}p_n\ketbra{a_\ell b_n}\1\otimes\ket{b_i} \\
                                                        & = \sum \lambda_{\ell n}p_n\ketbra{a_\ell}\,\delta_{in}\,\delta_{ni}                 \\
                                                        & = \sum_{\ell n}\lambda_{\ell n}p_n\ketbra{a_\ell}
    \end{align*}
    Now finally we can can take the expectation values to show $E_\nu$ is positive semi-definite.


    I'm a little iffy if this actually works. I feel like there should be a more elegant way to show $E_\nu$ is positive semi-definite, but I can't come up with anything.
\end{solution}

\begin{problem}
Apply Naimark's theorem to identify a \ac{PVM} in an extended Hilbert space that generates the trine.
\end{problem}

\begin{solution}
\end{solution}

\begin{problem}
\begin{parts}
    \part{Verify that the map $E$ defined in terms of projectors onto coherent states in the example above satisfies the postulates of a \ac{POVM}.}\label{part:33a}
    \part{What is the operational interpretation of $\Pr(X) = \tr(E(X)\rho)$ for this \ac{POVM}, noting that $\alpha = \qty(\expval{q}, \expval{p})$ denotes the expectations of the position $q$ and momentum $p$ operatores in the associated coherent state, and that $\Omega = \R^2$ means we are measuring the position and momentum of some particle?}\label{part:33b}
\end{parts}
\end{problem}

\begin{solution}
    \ref{part:33a}
    \ref{part:33b}
\end{solution}

\end{document}