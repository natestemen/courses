\documentclass[
	pages,
	boxes,
	color=RoyalPurple
]{homework}


\usepackage{macros}
\usepackage[footnote,nolist,smaller]{acronym}
\name{Nate Stemen}
\studentid{20906566}
\email{nate@stemen.email}
\term{Fall 2021}
\course{General Relativity for Cosmology}
\courseid{AMATH 875}

\hwname{Lecture}
\hwnum{1}
\duedate{Mon, Feb 22, 2020 11:59 PM}


\makeatletter
\numberwithin{tcb@cnt@prob}{section}
\makeatother


\begin{document}

\begin{acronym}
    \acro{PVM}{Projection-Valued Measure}
    \acro{POVM}{Positive Operator-Valued Measure}
\end{acronym}

\setcounter{section}{2}
\problemnumber{2}

\begin{problem}
Show that the reduced state obtained via partial trace is a density operator, i.e., a non-negative operator satisfying $\tr\rho_A = 1$.
\end{problem}

\begin{solution}
\end{solution}

\begin{problem}
Prove that these three pure-state conditions are equivalent.
\end{problem}

\begin{solution}
\end{solution}

\problemnumber{5}

\begin{problem}
Prove the existence of the Schmidt decomposition.
\end{problem}

\begin{solution}
\end{solution}

\setcounter{section}{3}
\problemnumber{1}

\begin{problem}
Prove the two properties given by Eqns. 3.1.
\end{problem}

\begin{solution}
\end{solution}

\begin{problem}
Apply Naimark's theorem to identify a \ac{PVM} in an extended Hilbert space that generates the trine.
\end{problem}

\begin{solution}
\end{solution}

\begin{problem}
\begin{parts}
    \part{Verify that the map $E$ defined in terms of projectors onto coherent states in the example above satisfies the postulates of a \ac{POVM}.}\label{part:33a}
    \part{What is the operational interpretation of $\Pr(X) = \tr(E(X)\rho)$ for this \ac{POVM}, noting that $\alpha = \qty(\expval{q}, \expval{p})$ denotes the expectations of the position $q$ and momentum $p$ operatores in the associated coherent state, and that $\Omega = \R^2$ means we are measuring the position and momentum of some particle?}
\end{parts}
\end{problem}

\begin{solution}
\end{solution}

\end{document}