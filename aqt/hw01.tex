\documentclass[boxes]{homework}


\name{Nate Stemen}
\studentid{20906566}
\email{nate.stemen@uwaterloo.ca}
\term{Fall 2020}
\course{Advanced Quantum Theory}
\courseid{AMATH 673}
\hwnum{1}
\duedate{Fri, Sep 18, 2020 11:59 AM}

%\hwname{(Name)}
\problemname{Exercise}
%\solutionname{(Name)

\usepackage{physics}
\usepackage[makeroom,Smaller]{cancel}
\usepackage{marginnote}
\renewcommand*{\marginfont}{\footnotesize}

\newcommand{\mmtm}[2]{p_#1^{(#2)}}
\newcommand{\pstn}[2]{x_#1^{(#2)}}

\makeatletter
\numberwithin{@problem}{section}
\makeatother

\begin{document}

\setcounter{section}{2}

\begin{problem}
Prove $\pb{c}{f} = 0$.
\end{problem}

\begin{solution}
	The antisymmetry of the Poisson bracket allows us to equivalently prove $\pb{f}{c} = 0$.
	\begin{align*}
		\pb{f}{c} & = \pb{f}{\sqrt{c}}\sqrt{c} + \sqrt{c}\pb{f}{\sqrt{c}} \\
		          & = 2\sqrt{c}\pb{f}{\sqrt{c}}                           \\
		          & = -2\sqrt{c}\pb{\sqrt{c}}{f}                          \\
		          & = -2\pb{c}{f}                                         \\
		          & = 2\pb{f}{c}
	\end{align*}
	Since we now have $\pb{f}{c} = 2\pb{f}{c}$ we can subtract $\pb{f}{c}$ from both sides to obtain $\pb{f}{c} = 0$.
\end{solution}

\begin{problem}
Show that $\pb{f}{f} = 0$ for any $f$.
\end{problem}

\begin{solution}
	Again using the antisymmetry axiom we can see $\pb{f}{f} = -\pb{f}{f}$ and hence adding $\pb{f}{f}$ to both sides we obtain $\pb{f}{f} = 0$.
\end{solution}

\begin{problem}
Assume that $n$ is a positive integer.
\begin{parts}
	\part{Evaluate $\pb{x_1}{p_1^n}$}\label{part:3a}
	\part{Evaluate $\pb{x_1^n}{p_1}$}\label{part:3b}
\end{parts}
\end{problem}

\begin{solution}
	\ref{part:3a}
	I will use a proof by induction to prove $\pb{x_1}{p_1^n} = (2p)^{n-1}$. Starting with the base case of $n = 1$. Then we have $\pb{x_1}{p_1} = (2p)^0 = 1$ which agrees with the definition of the Poisson bracket for position and momentum. Assuming the formula is true for $n$ we will show it's true for $n+1$.
	\begin{align*}
		\pb{x_1}{p_1^{n+1}} & = \pb{x_1}{p_1^n}p_1 + p_1\pb{x_1}{p_1^n} \\
		                    & = 2p_1\pb{x_1}{p_1^n}                     \\
		                    & = 2p_1(2p_1)^{n-1}                        \\
		                    & = (2p_1)^n = (2p_1)^{(n + 1) - 1}
	\end{align*}


	\ref{part:3b}
	I know this is no longer part of the homework, but I did it before the changes were made, so I figured I would leave it in.

	Here we will use the fact that $\pb{fg}{h} = \pb{f}{h}g + f\pb{g}{h}$ which can be derived from the product rule as follows.
	\begin{align*}
		\pb{fg}{h} & = -\pb{h}{fg}              \\
		           & = -\pb{h}{f}g - f\pb{h}{g} \\
		           & = \pb{f}{h}g + f\pb{g}{h}
	\end{align*}
	Again I will use proof by induction to prove $\pb{x_2^n}{p_2} = nx_2^{n-1}$. Starting with the base case of $n = 1$. Then we have $\pb{x_2}{p_2} = 1\cdot x_2^0 = 1$ which agrees with the definition. Assuming the formula is true for $n$ we will show it's true for $n + 1$.
	\begin{align*}
		\pb{x_2^{n+1}}{p_2} & = \pb{x_2^n}{p_2}x_2 + x_2^n\pb{x_2}{p_2} \\
		                    & = \qty(nx_2^{n-1})x_2 + x_2^n             \\
		                    & = nx_2^n + x_2^n                          \\
		                    & = (n+1)x_2^n = (n+1)x_2^{(n+1)-1}
	\end{align*}
\end{solution}

\begin{problem}
Verify $\pb{-2p_1}{3x_1^2 + 7p_3^4 - 2x_2^2p_1^3 + 6} = 12x_1$.
\end{problem}

\begin{solution}
	\begin{align*}
		\text{LHS} & = -2\pb{p_1}{3x_1^2 + 7p_3^4 - 2x_2^2p_1^3 + 6}                                                                  \\
		           & = -2\qty(3\pb{p_1}{x_1^2} + 7\cancelto{0}{\pb{p_1}{p_3^4}} - 2 \pb{p_1}{x_2^2p_1^3} + \cancelto{0}{\pb{p_1}{6}}) \\
		           & = -2\qty(-6x_1 - 2\qty(\cancelto{0}{\pb{p_1}{x_2^2}}p_1^3 + x_2^2\cancelto{0}{\pb{p_1}{p_1^3}}))                 \\
		           & = 12x_1
	\end{align*}
\end{solution}

\begin{problem}
Show that the Poisson bracket is not associative by giving a counterexample.
\end{problem}

\begin{solution}
	If the Poisson bracket was associative it would mean the following: $\pb{f}{\pb{g}{h}} = \pb{\pb{f}{g}}{h}$ for $f, g, h$ polynomials in $x_i, p_i$. Take $f = p_1^2$, $g = x_1$, and $h = p_1$. We can then evaluate both sides to see this does not hold.
	\begin{align*}
		\text{LHS} & = \pb{p_1^2}{\pb{x_1}{p_1}} = \pb{p_1^2}{1} = 0   \\
		\text{RHS} & = \pb{\pb{p_1^2}{x_1}}{p_1} = \pb{-2p_1}{p_1} = 1
	\end{align*}
	Hence the Poisson bracket is not associative.
\end{solution}

\begin{problem}
Look up and state the axioms of
\begin{parts}
	\part{a Lie algebra}\label{part:6a}
	\part{an associative algebra}\label{part:6b}
	\part{a Poisson algebra}\label{part:6c}
\end{parts}
\end{problem}

\begin{solution}
	\ref{part:6a}
	A \emph{Lie algebra} is a vector space $\mathfrak{g}$ equipped with bilinear map (called a Lie Bracket) $[\cdot, \cdot]: \mathfrak{g}\times\mathfrak{g}\to\mathfrak{g}$ which is anticommutative ($[x, y] = -[y, x]$) and satisfies the Jacobi Identity.
	\begin{equation*}
		[x, [y, z]] + [y, [z, x]] + [z, [x, y]] = 0 \quad\forall x,y,z\in\mathfrak{g}
	\end{equation*}

	\ref{part:6b}
	If $R$ is a commutative ring, then an \emph{associative algebra} is an $R$-module $V$ together with a bilinear map $V\times V \to V$ that is associative and has an identity.

	\ref{part:6c}
	A \emph{Poisson algebra} is a vector space equipped with two bilinear products $-\cdot-$ and $\pb{\cdot}{\cdot}$ such that
	\begin{itemize}
		\item $-\cdot-$ forms an associative algebra,
		\item $\pb{\cdot}{\cdot}$ is antisymmetric, obeys the Jacobi identity and forms a Lie algebra
		\item The Poisson bracket $\pb{\cdot}{\cdot}$ acts as $\pb{x}{y\cdot z} = \pb{x}{y}\cdot z + y\cdot\pb{x}{z}$ for all $x,y,z$ in the Poisson algebra.
	\end{itemize}
\end{solution}

\begin{problem}
Prove that both methods of calculating $\dot{f}$ yield the same result.
\end{problem}

\begin{solution}
	\begin{align*}
		\dv{f}{t} & = \dot{g}h + g\dot{h}            \\
		          & = \pb{g}{H}h + g\pb{h}{H}        \\
		          & = -\qty(\pb{H}{g}h + g\pb{H}{h}) \\
		          & = -\pb{H}{gh}                    \\
		          & = \pb{gh}{H} = \pb{f}{H}
	\end{align*}
\end{solution}

\begin{problem}
Use the Jacobi identity to prove that
\begin{equation*}
	\dv{t}\pb{f}{g} = \pb{\dot{f}}{g} + \pb{f}{\dot{g}}
\end{equation*}
\end{problem}

\begin{solution}
	\begin{align*}
		\dv{t}\pb{f}{g} & = \pb{\pb{f}{g}}{H}                                           \\
		                & = -\pb{H}{\pb{f}{g}}                                          \\
		                & = \pb{f}{\pb{g}{H}} + \pb{g}{\pb{H}{f}} \tag{Jacobi identity} \\
		                & = \pb{f}{\dot{g}} + \pb{g}{-\dot{f}}                          \\
		                & = \pb{\dot{f}}{g} + \pb{f}{\dot{g}}
	\end{align*}
\end{solution}

\begin{problem}
Show that if $H$ is a polynomial in the positions and momenta with arbitrary (and possibly time-dependent) coefficients, it is true that $\dv{H}{t} = \pdv{H}{t}$.
\end{problem}

\begin{solution}
	\begin{align*}
		\dv{t}H(x, p, t) & = \pdv{H}{x}\dv{x}{t} + \pdv{H}{p}\dv{p}{t} + \pdv{H}{t}   \\
		                 & = \pdv{H}{x}\dot{x} + \pdv{H}{p}\dot{p} + \pdv{H}{t}       \\
		                 & = \pdv{H}{x}\pdv{H}{p} - \pdv{H}{p}\pdv{H}{x} + \pdv{H}{t} \\
		                 & = \pdv{H}{t}
	\end{align*}
\end{solution}

\begin{problem}
Show that the total momentum is conserved.
\end{problem}

\begin{solution}
	To show $\mmtm{i}{1} + \mmtm{i}{2}$ is conserved for all $i$ we will show it's time derivative is 0 by taking it's Poisson bracket with the Hamiltonian.
	\begin{align*}
		\dv{t}\qty(\mmtm{i}{1} + \mmtm{i}{2}) & = \pb{\mmtm{i}{1} + \mmtm{i}{2}}{H}                                                                                                                                              \\
		                                      & = \pb{\mmtm{i}{1} + \mmtm{i}{2}}{\frac{k}{2}\sum_j\qty(\pstn{j}{1} - \pstn{j}{2})^2} \marginnote{no contribution from kinetic}                                                   \\
		                                      & = \frac{k}{2}\pb{\mmtm{i}{1} + \mmtm{i}{2}}{\qty(\pstn{i}{1} - \pstn{i}{2})^2}                                                                                                   \\
		                                      & = \frac{k}{2}\qty[\pb{\mmtm{i}{1}}{{\pstn{i}{1}}^2} - 2\pb{\mmtm{i}{1}}{\pstn{i}{1}\pstn{i}{2}} + \pb{\mmtm{i}{2}}{{\pstn{i}{2}}^2} - 2\pb{\mmtm{i}{2}}{\pstn{i}{1}\pstn{i}{2}}] \\
		                                      & = \frac{k}{2}\qty[-2\pstn{i}{1} + 2\pstn{i}{2} - 2\pstn{i}{2} + 2\pstn{i}{1}]                                                                                                    \\
		                                      & = 0
	\end{align*}
	With this, and the fact that has no explicit time dependence, we can conclude that momentum is conserved.
\end{solution}

\begin{problem}
Use the derivative definition of the Poisson bracket to evaluate $\pb{x^8p^6}{x^3p^4}$.
\end{problem}

\begin{solution}
	\begin{align*}
		\pb{x^8p^6}{x^3p^4} & = \pdv{x}\qty(x^8p^6)\pdv{p}\qty(x^3p^4) - \pdv{p}\qty(x^8p^6)\pdv{x}\qty(x^3p^4) \\
		                    & = \qty(8x^7p^6)\qty(4x^3p^3) - \qty(6x^8p^5)\qty(3x^2p^4)                         \\
		                    & = 32x^{10}p^9 - 18x^{10}p^9                                                       \\
		                    & = 14x^{10}p^9
	\end{align*}
\end{solution}

\begin{problem}
Show that the derivative definition of the Poisson bracket is indeed a representation of the Poisson bracket defined by the axioms.
\end{problem}

\begin{solution}
	First, antisymmetry.
	\begin{align*}
		\pb{f}{g} & = \sum_{r=1}^{n} \sum_{i=1}^{3}\qty(\pdv{f}{x_{i}^{(r)}} \pdv{g}{p_{i}^{(r)}} - \pdv{f}{p_{i}^{(r)}} \pdv{g}{x_{i}^{(r)}}) \\
		          & = -\sum_{r=1}^{n} \sum_{i=1}^{3}\qty(\pdv{g}{x_{i}^{(r)}}\pdv{f}{p_{i}^{(r)}} - \pdv{g}{p_{i}^{(r)}}\pdv{f}{x_{i}^{(r)}})  \\
		          & = -\pb{g}{f}
	\end{align*}
	Second, linearity.
	\begin{align*}
		\pb{cf}{g} & = \sum_{r=1}^{n} \sum_{i=1}^{3}\qty(\pdv{(cf)}{x_{i}^{(r)}} \pdv{g}{p_{i}^{(r)}} - \pdv{(cf)}{p_{i}^{(r)}} \pdv{g}{x_{i}^{(r)}}) \\
		           & = c\sum_{r=1}^{n} \sum_{i=1}^{3}\qty(\pdv{f}{x_{i}^{(r)}} \pdv{g}{p_{i}^{(r)}} - \pdv{f}{p_{i}^{(r)}} \pdv{g}{x_{i}^{(r)}})      \\
		           & = c\pb{f}{g}
	\end{align*}
	Third, the addition rule.
	\begin{align*}
		\pb{f}{g + h} & = \sum_{r=1}^{n} \sum_{i=1}^{3}\qty(\pdv{f}{x_{i}^{(r)}} \pdv{(g + h)}{p_{i}^{(r)}} - \pdv{f}{p_{i}^{(r)}} \pdv{(g + h)}{x_{i}^{(r)}})                                                                                                                \\
		              & = \sum_{r=1}^{n} \sum_{i=1}^{3}\qty(\pdv{f}{x_{i}^{(r)}} \pdv{g}{p_{i}^{(r)}} - \pdv{f}{p_{i}^{(r)}} \pdv{g}{x_{i}^{(r)}} + \pdv{f}{x_{i}^{(r)}} \pdv{h}{p_{i}^{(r)}} - \pdv{f}{p_{i}^{(r)}} \pdv{g}{x_{i}^{(r)}})                                    \\
		              & = \sum_{r=1}^{n} \sum_{i=1}^{3}\qty(\pdv{f}{x_{i}^{(r)}} \pdv{g}{p_{i}^{(r)}} - \pdv{f}{p_{i}^{(r)}} \pdv{g}{x_{i}^{(r)}}) + \sum_{r=1}^{n} \sum_{i=1}^{3}\qty(\pdv{f}{x_{i}^{(r)}} \pdv{h}{p_{i}^{(r)}} - \pdv{f}{p_{i}^{(r)}} \pdv{g}{x_{i}^{(r)}}) \\
		              & = \pb{f}{g} + \pb{f}{h}
	\end{align*}
	Fourth, the product rule.
	\begin{align*}
		\pb{f}{gh} & = \sum_{r=1}^{n} \sum_{i=1}^{3}\qty(\pdv{f}{x_{i}^{(r)}} \pdv{(gh)}{p_{i}^{(r)}} - \pdv{f}{p_{i}^{(r)}} \pdv{(gh)}{x_{i}^{(r)}})                                                       \\
		           & = \sum_{r=1}^{n} \sum_{i=1}^{3}\qty(\pdv{f}{x_{i}^{(r)}}\qty[\pdv{g}{p_{i}^{(r)}}h + g\pdv{h}{p_{i}^{(r)}}] - \pdv{f}{p_{i}^{(r)}}\qty[\pdv{g}{x_{i}^{(r)}}h + g\pdv{h}{x_{i}^{(r)}}]) \\
		           & = h\sum_{r=1}^{n} \sum_{i=1}^{3} \qty(\pdv{f}{x_{i}^{(r)}}\pdv{g}{p_{i}^{(r)}} - \pdv{f}{p_{i}^{(r)}}\pdv{g}{x_{i}^{(r)}})                                                             \\
		           & \qquad + g\sum_{r=1}^{n} \sum_{i=1}^{3} \qty(\pdv{f}{x_{i}^{(r)}}\pdv{h}{p_{i}^{(r)}} - \pdv{f}{p_{i}^{(r)}}\pdv{h}{x_{i}^{(r)}})                                                      \\
		           & = \pb{f}{g}h + g\pb{f}{h}
	\end{align*}
	Fifth, six, and seventh: the rules for positions and momenta.
	\begin{align*}
		\pb{x_i^{(r)}}{p_j^{(s)}} & = \sum_{r=1}^{n} \sum_{i=1}^{3}\qty(\cancelto{1}{\pdv{x_i^{(r)}}{x_{i}^{(r)}}} \pdv{p_j^{(s)}}{p_{i}^{(r)}} - \cancelto{0}{\pdv{x_i^{(r)}}{p_{i}^{(r)}} \pdv{p_j^{(s)}}{x_{i}^{(r)}}}) \\
		                          & = \sum_{r=1}^{n} \sum_{i=1}^{3}\delta_{i,j}\delta_{s,r}                                                                                                                                \\
		                          & = \delta_{i,j}\delta_{s,r}                                                                                                                                                             \\
	\end{align*}
	\begin{align*}
		\pb{x_i^{(r)}}{x_j^{(s)}} & = \sum_{r=1}^{n} \sum_{i=1}^{3}\qty(\pdv{x_i^{(r)}}{x_{i}^{(r)}} \cancelto{0}{\pdv{x_j^{(s)}}{p_{i}^{(r)}}} - \cancelto{0}{\pdv{x_i^{(r)}}{p_{i}^{(r)}}} \pdv{x_j^{(s)}}{x_{i}^{(r)}}) \\
		                          & = 0
	\end{align*}
	\begin{align*}
		\pb{p_i^{(r)}}{p_j^{(s)}} & = \sum_{r=1}^{n} \sum_{i=1}^{3}\qty(\cancelto{0}{\pdv{p_i^{(r)}}{x_{i}^{(r)}}} \pdv{p_j^{(s)}}{p_{i}^{(r)}} - \pdv{p_i^{(r)}}{p_{i}^{(r)}} \cancelto{0}{\pdv{p_j^{(s)}}{x_{i}^{(r)}}}) \\
		                          & = 0
	\end{align*}
\end{solution}

\begin{problem}
Find the representation of the Hamilton equations
\end{problem}

\begin{solution}
	We start with eq. 2.21 and eq. 2.22 and then use those results for eq. 2.19.
	\begin{align*}
		\dv{t}\pstn{i}{r} & = \pb{\pstn{i}{r}}{H}                                                                                                             \\
		                  & = \sum_{s,j}\pdv{\pstn{i}{r}}{\pstn{j}{s}}\pdv{H}{\mmtm{j}{s}} - \cancelto{0}{\pdv{\pstn{i}{r}}{\mmtm{j}{s}}\pdv{H}{\pstn{j}{s}}} \\
		                  & = \sum_{s,j}\delta_{i,j}\delta_{r,s}\pdv{H}{\mmtm{j}{s}}                                                                          \\
		\dot{x}_i^{(r)}   & = \pdv{H}{\mmtm{i}{r}}
	\end{align*}
	\begin{align*}
		\dv{t}\mmtm{i}{r} & = \pb{\mmtm{i}{r}}{H}                                                                                                             \\
		                  & = \sum_{s,j}\cancelto{0}{\pdv{\mmtm{i}{r}}{\pstn{j}{s}}\pdv{H}{\mmtm{j}{s}}} - \pdv{\mmtm{i}{r}}{\mmtm{j}{s}}\pdv{H}{\pstn{j}{s}} \\
		                  & = -\sum_{s,j}\delta_{i,j}\delta_{r,s}\pdv{H}{\pstn{j}{s}}                                                                         \\
		\dot{p}_i^{(r)}   & = -\pdv{H}{\pstn{i}{r}}
	\end{align*}
	\begin{align*}
		\dv{f}{t} & = \pb{f}{H} + \pdv{f}{t}                                                                                     \\
		          & = \sum_{r,i}\pdv{f}{\pstn{i}{r}}\pdv{H}{\mmtm{i}{r}} - \pdv{f}{\mmtm{i}{r}}\pdv{H}{\pstn{i}{r}} + \pdv{f}{t} \\
		          & = \sum_{r,i}\pdv{f}{\pstn{i}{r}}\dot{x}_i^{(r)} + \pdv{f}{\mmtm{i}{r}}\dot{p}_i^{(r)} + \pdv{f}{t}
	\end{align*}
	Which is exactly the chain rule for a function $f(\mathbf{x}^{(1)},\ldots,\mathbf{x}^{(n)}, \mathbf{p}^{(1)},\ldots,\mathbf{p}^{(n)}, t)$.

\end{solution}

% \begin{problem}
% Derive the equations of motion for $\phi(x, t)$ and $\pi(x, t)$.
% \end{problem}

% \begin{solution}
% 	Take $f = \phi(x, t)$ first. Then
% \end{solution}



\end{document}