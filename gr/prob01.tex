% !TEX program = xelatex
\documentclass[11pt,aspectratio=1610]{beamer}

\usetheme[
    background=light,
    numbering=fraction,
    block=fill,
    progressbar=frametitle
]{metropolis}

\usepackage{physics}
\usepackage{tcolorbox}
\newtcolorbox{prob}{colback=green!5!white,colframe=green!75!black}
\usefonttheme[onlymath]{serif}


\title{General Relativity Exam Problem 1}
\subtitle{Is the induced map on tangent spaces linear?}
\date{Sep 13, 2021}
\author{Nate Stemen (he/they)}
\institute{AMATH 875}


\begin{document}

\maketitle

\begin{frame}{Problem Statement}
	\large
	\begin{prob}
		Recall that when $M$ and $N$ are smooth manifolds, $\phi: M \to N$ is a diffeomorphism, and $p$ is a point in $M$, there is an induced map on the tangent spaces $T_p\phi: T_p(M) \to T_{\phi(p)}(N)$ as defined in lecture.
		Show that $T_p\phi$ is linear.
	\end{prob}
\end{frame}

\begin{frame}{Solution}
	\begin{columns}
		\begin{column}{0.3\textwidth}
			\begingroup
			\tiny
			\begin{prob}
				Recall that when $M$ and $N$ are smooth manifolds, $\phi: M \to N$ is a diffeomorphism, and $p$ is a point in $M$, there is an induced map on the tangent spaces $T_p\phi: T_p(M) \to T_{\phi(p)}(N)$ as defined in lecture.
				Show that $T_p\phi$ is linear.
			\end{prob}
			\endgroup
		\end{column}
		\begin{column}{0.7\textwidth}
			\begin{solution}
				Recall how the map $T_p\phi$ is defined:
				\begin{align*}
					T_p\phi: T_p(M) & \longrightarrow T_{\phi(p)}(N) \\
					\xi             & \longmapsto \xi \circ \phi^*
				\end{align*}
				With $f\in \mathcal{F}(\phi(p))$ we use the induced map $\phi^*$ as follows.
				\begin{align*}
					\qty\big[T_p\phi(\alpha \xi_1 + \beta \xi_2)](f\,) & = \qty\Big[(\alpha\xi_1 + \beta\xi_2)\circ \phi^*](f\,)                  \\
					                                                   & = (\alpha\xi_1 + \beta\xi_2)\qty\big(f\circ\phi)                         \\
					                                                   & = \alpha (\xi_1 \circ f \circ \phi) + \beta(\xi_2 \circ f \circ \phi)    \\
					                                                   & = \alpha (\xi_2 \circ \phi^*)(f\,) + \beta (\xi_2 \circ \phi^*)(f\,)     \\
					                                                   & = \alpha \qty\big[T_p\phi(\xi_1)](f\,) + \beta \qty\big[T_p(\xi_2)](f\,)
				\end{align*}
			\end{solution}
		\end{column}
	\end{columns}
\end{frame}


\end{document}
