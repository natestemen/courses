\documentclass{homework}

\usepackage{cleveref}
\usepackage{macros}

\name{Nate Stemen}
\studentid{20906566}
\email{nate@stemen.email}
\term{Fall 2021}
\course{General Relativity for Cosmology}
\courseid{AMATH 875}

\hwname{Lecture}
\hwnum{5}
\duedate{Wed, Sep 22, 2021 11:59 PM}

\begin{document}

In this lecture we saw two important topics:
\begin{enumerate}
	\item the inner derivation on the algebra of completely antisymmetric differential forms\label{it:inner}
	\item the definition of the Lie derivative\label{it:lie}
\end{enumerate}

As we saw in the last lecture, the exterior derivative $\dd : \Lambda^k(M)\to\Lambda^{k+1}(M)$ gave us a way to differentiate rank $(0, k)$ covariant tensors.
That said, in order to do more advanced physics, and presumably general relativity, we need a way to differentiate more advance objects, such as any rank $(a, b)$ tensor with respect to any other.
To this end we construct the inner derivation $\iota_X :\Lambda^k(M)\to \Lambda^{k-1}(M)$ and use it in conjunction with the exterior derivative to arrive at a definition of derivative that is general enough to handle our wildest dreams.

\paragraph{\Cref{it:inner}:}
We first \emph{define} the inner derivation to send $0$-forms to 0, and $1$-forms to a $0$-form under $\omega\mapsto \iota_\xi(\omega) = \omega(\xi)$ where $\xi$ is vector field defined on our manifold $M$.
With this definition we then saw the way this map works if $\gamma\in\Lambda^k(M)$ then we have
\begin{equation*}
	\iota_\xi(\gamma)(\eta_1,\ldots,\eta_{k-1}) = \gamma(\xi,\eta_1,\ldots,\eta_{k-1}).
\end{equation*}
Some call this map a contraction as we are simply using $\xi$ as the first positional argument.

\paragraph{\Cref{it:lie}:}
We now have a map that raises the degree of the form (the exterior derivative $\dd$), and a map that lowers the degree (the interior/inner derivative $\iota_\xi$).
Our goal is to construct a degree-0 derivation on $\Lambda(M)$ such that it extends the notion of derivative, and aligns with our previous notions of directional derivatives on functions (0-forms).
This goal is accomplished by taking
\begin{equation*}
	\mathcal{L}_\xi\defeq \dd\circ\iota_\xi + \iota_\xi\circ\dd.
\end{equation*}
Well indeed this does what we want, but if we define the Lie derivative this way, then it won't actually be defined on all tensor fields because $\dd$ and $\iota_\xi$ are not defined on all tensor fields.
Hence we go back to our roots and the first definition of the derivative we encounter: the Newton-Leibniz definition of a derivative.
There are some complications here since a tensor field at a point $p$ and a nearby point $p + \varepsilon$ cannot be subtracted due to living in completely different spaces.
The first way I thought this would be solved is via parallel transport, and maybe that's what Achim is doing, but using a slightly different language of ``flows'' instead.

\end{document}