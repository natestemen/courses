\documentclass{homework}

\usepackage{cleveref}
\usepackage{macros}

\name{Nate Stemen}
\studentid{20906566}
\email{nate@stemen.email}
\term{Fall 2021}
\course{General Relativity for Cosmology}
\courseid{AMATH 875}

\hwname{Lecture}
\hwnum{4}
\duedate{Sun, Sep 19, 2021 11:59 PM}

\begin{document}

In this lecture we saw three important topics:
\begin{enumerate}
	\item the introduction of differential forms and the exterior algebra\label{it:forms}
	\item derivations on the exterior algebra,\label{it:derivation}
	\item and the basic idea of a cohomology theory\label{it:cohomology}
\end{enumerate}

\paragraph{\Cref{it:forms}:}
My main takeaway for the question of ``what is a differential form?'' is\dots, well basically just an antisymmetric tensor!
More formally a differential $k$-form is a tensor of rank $(0, k)$ which is antisymmetric.
By antisymmetric we mean for $\nu: T_p(M)_r\to \R$ then
\begin{equation*}
	\nu(\xi_1, \ldots, \xi_i, \ldots, \xi_j, \ldots \xi_k) = - \nu(\xi_1, \ldots, \xi_j,\ldots, \xi_i, \ldots, \xi_k).
\end{equation*}
We use the following notation for the space of differential $k$-forms:
\begin{equation*}
	\Lambda_k(p) \defeq AT_p(M)_k,
\end{equation*}
where $A$ is the antisymmetrization operator which pulls out the antisymmetric part of any tensor.
We can then define a product on $\Lambda_k(p)$ which we call the wedge product which takes a $k$-form, and an $\ell$-form into a $(k + \ell)$-form.
This product can be extended to $\Lambda(p) \defeq \bigoplus_{i = 0}^n \Lambda_i(p)$ and turns this into an algebra which is called the exterior algebra.
Very importantly, we can then extend these concepts from only existing at points $p\in M$, to existing on the entire manifold.
We use $\Lambda(M)$ to denote ``differential form fields'' on the entire manifold (which is still an algebra).

\paragraph{\Cref{it:derivation}:}
With an algebra structure on $\Lambda(M)$ we can ask what are the derivations on such an algebra.
Nicely enough, we actually have one we've been working with that turns out to be an \emph{anti-derivation}:
\begin{align*}
	\dd{} : \Lambda(M) & \longrightarrow \Lambda(M)                                                            \\
	f                  & \longmapsto \dd{f}\quad\text{where }f\in\Lambda_0(M)\text{ and }\dd{f}\in\Lambda_1(M) \\
	\dd{x_i}           & \longmapsto 0\quad\text{for all }i.
\end{align*}

\paragraph{\Cref{it:cohomology}:}
Without getting too much into it (because I don't understand it well enough, nor did we get too into it), a cohomology theory is an attempt to classify differentiable manifolds.
To do this we use the fact that it's easier to tell if two abelian groups are isomorphic than it is two manifolds are diffeomorphic.
Therefore, we construct\footnote{This word \emph{construct} is doing a \textbf{lot} of heavy lifting.} a map from smooth manifolds to abelian groups such that if the image of two manifolds are not isomorphic, then the manifolds must not have been diffeomorphic.
This helps us classify manifolds up to a ``global topology''.
We then also saw how this cohomology can be seen as a functor from the category of smooth manifolds $\mathsf{Smooth}$ to the category of Abelian groups $\mathsf{Ab}$.
This birds eye perspective allows us to see many similarities between seeemingly disparate fields.

\end{document}