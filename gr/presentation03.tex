% !TEX program = xelatex
\documentclass[11pt,aspectratio=1610,xcolor=dvipsnames]{beamer}

\usetheme[
    background=light,
    numbering=fraction,
    block=fill,
    progressbar=frametitle
]{metropolis}

\usepackage{physics}
\usepackage{bbm}
\usepackage[most]{tcolorbox}
\newtcolorbox{prob}{colback=green!5!white,colframe=green!75!black}
\colorlet{LightLavender}{Lavender!40!}
\usefonttheme[onlymath]{serif}
\newcommand{\GL}[2]{\mathsf{GL}(#1\,; #2)}
\newcommand{\R}{\mathbb{R}}
\newcommand{\1}{\mathbbm{1}}
\newcommand{\defeq}{\stackrel{\text{\tiny def}}{=}}
\DeclareMathOperator{\id}{id}


\newcommand{\problemstatement}{Does the topology on Minkowski space agree with the ``metric topology'' induced by the metric $\eta_{\mu\nu} = \smqty[-1 & & &  \\ & \phantom{+}1 & & \\ & & \phantom{+}1 & \\ & & & \phantom{+}1]$?}
\title{General Relativity Exam Problem 3}
\date{Sep 30, 2021}
\author{Nate Stemen (he/they)}
\institute{AMATH 875}


\begin{document}

\maketitle

\begin{frame}{Problem Statement}
	\large
	\begin{prob}
		\problemstatement
	\end{prob}
\end{frame}

\begin{frame}{What is Minkowski space?}
	% \begin{definition}[Minkowski Space]
	% 	$\R\times\R^3 = \R^{1,3}$
	% \end{definition}
	\begin{definition}[Metric Space]
		A \emph{metric space} is a pair $(M, d)$ where $M$ is a set, and $d: M \times M \to \R$ is a function satisfying
		\begin{itemize}
			\item $d(x, y) = 0$ if and only if $x = y$,
			\item $d(x, y) = d(y, x)$, and
			\item $d(x, z) \leq d(x, y) + d(y, z)$.
		\end{itemize}
	\end{definition}
	\pause
	\begin{definition}[Open Balls]
		For any $x\in M$ and $r > 0$, we define the open ball of radius $r$ to be
		\begin{equation*}
			B(x; r) = \qty{y\in M : d(x, y) < r}.
		\end{equation*}
	\end{definition}
	% \begin{itemize}[<+->]
	% 	\item $\R^{n\times n} \cong \R^{n^2}$ is trivially a manifold
	% 	\item $\GL{n}{\R}\subset \R^{n\times n}$
	% 	\item $\det: \R^{n\times n} \to \R$ is a continuous function
	% 	\item $\GL{n}{\R} = \det^{-1}\qty(\R\setminus\qty{0})$
	% 	\item Thus $\GL{n}{\R}$ is an open subset of a smooth manifold
	% \end{itemize}
\end{frame}

\begin{frame}
	\huge THIS PRESENTATION IS NOT YET COMPLETE.
\end{frame}
\end{document}
