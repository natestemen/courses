\documentclass{homework}

\usepackage{cleveref}
\usepackage{macros}

\name{Nate Stemen}
\studentid{20906566}
\email{nate@stemen.email}
\term{Fall 2021}
\course{General Relativity for Cosmology}
\courseid{AMATH 875}

\hwname{Lecture}
\hwnum{6 \& 7}
\duedate{Wed, Sep 29, 2021 11:59 PM}

\begin{document}

In these two lectures we added structure to our manifold so it becomes a more familiar place upon which we can do physics.
To start we first introduced the notion of orientability which is a requirement that the normal vector to a manifold varies continuously around the manifold $M$ (and that it's uniquely defined at every point).
A classic example of non-orientability is the M\"obius strip.
Now if $M$ is orientable we can define integration of a top form $\omega$ using a chart $\alpha$ and our pre-existing notion of integration on $\R^n$.
\begin{equation*}
	\int_M \omega \defeq \int_{\alpha(M)}f(x)\dd{x}^1\dd{x}^2\cdots\dd{x}^n
\end{equation*}
This is exactly why we introduced the notion of $k$-forms, as they are the things that we integrate over.
An important theorem of differential geometry is that of Stokes' Theorem:
\begin{equation*}
	\int_M\dd{\omega} = \int_{\partial M}\omega
\end{equation*}
where $\omega\in\Lambda_{n-1}(M)$ and $\partial M$ is the boundary of $M$.
We then went on to show how many of the theorems we saw in calculus and multi-variable calculus can all be derived from Stokes's Theorem.

In lecture 7 we add another piece to our manifold: a metric, which we define to be rank $(0, 2)$\footnote{This implies covariance, right?} tensor.
A metric is serious business, and it adds lots of structure which gives us the ability to measure other things.
For example the metric makes it possible to measure angles, volumes, and curvature which we'll need in the future for GR.
We also saw how the metric $g_{\mu\nu}$ yields the Hodge star operator $\star: \Lambda_p \to \Lambda_{n - p}$ and in turn also a duality between the domain and range.
This Hodge star also turns $\Lambda_p$ into a Hilbert space which we know and love from quantum.
And finally---through a few steps---we can define the Laplacian $\laplace: \Lambda_p \to \Lambda_p$ and hence formulate wave equations on our manifold!
This is exciting because lots of physics is formulated in terms of wave equations.

We then used some of these new tools to show how Maxwell's equations can be formulated using exterior derivatives and the Hodge star:
\begin{align*}
	\dd{F} & = 0 & \delta F & = \star j.
\end{align*}

\end{document}