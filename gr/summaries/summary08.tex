\documentclass{homework}

\usepackage{cleveref}
\usepackage{macros}
\usepackage{tensor}

\newcommand{\psifield}{\tensor{\psi}{_{(i)}^{a\ldots b}_{c\ldots d}}}
\newcommand{\psifieldder}{\tensor{\psi}{_{(i)}^{a\ldots b}_{c\ldots d;e}}}

\newtheorem*{principle}{Action Principle}

\name{Nate Stemen}
\studentid{20906566}
\email{nate@stemen.email}
\term{Fall 2021}
\course{General Relativity for Cosmology}
\courseid{AMATH 875}

\hwname{Lecture}
\hwnum{12 \& 13}
\duedate{Wed, Oct 27, 2021 11:59 PM}

\begin{document}

In these lectures we start doing things that seem --- and sound --- like physics.
In particular we begin thinking about how matter and light behave on curved spacetimes.

We start with the \emph{action principle} from classical mechanics that we are familiar, and reformulate it on a general manifold with curvature specified by a metric $g$.
\begin{principle}
    A theory of matter fields is specified by Lagrangian function $L$ (scalar function of the matter fields $\psifield$), their first covariant derivatives, and the metric $g$.
    \begin{equation*}
        L(\psi) = L^{\mathrm{(matter)}}\qty(\qty{\psifield}, \qty{\psifieldder}, g)
    \end{equation*}
    If $B\subset M$ is some compact subset of our manifold, take the action to be
    \begin{equation*}
        S[\psi]\defeq \int_B L(\psi)\sqrt{g}\dd[4]{x}.
    \end{equation*}
    Extrapolating from classical mechanics where physical fields $\psi$ are extremal points of the action we have
    \begin{equation*}
        \fdv{S[\psi]}{\psi} = 0.
    \end{equation*}
\end{principle}
Without fully defining the functional derivative $\fdv{\psi}$ we can massage the last equation to read $\eval{\pdv{S[\psi]}{\lambda}}_{\lambda = 0} = 0$ where $\lambda$ is a parameter is a small number that keeps track of the deformation of $\psi$.

With some familiar machinery for those with experience with Lagrangian mechanics we can now write down the ``Euler-Lagrange equations''.
\begin{equation*}
    \pdv{L}{\psifield} - \qty(\pdv{L}{\psifieldder})_{;e} = 0
\end{equation*}
With this we now have ways to calculate the equations of motions of matter fields (and hence particles) on curved spacetimes!

We proceed to then study the classical analogs of matter, namely energy and momentum.
In classical mechanics symmetries of these things were given to us by Noether's theorem, but in a general curved spacetime we don't have the luxury of things like translations invariance!
Instead we consider automorphisms of the manifold $\phi: M \to M$ such that $\phi^*(g) = g$.
These often arrise as infinitesimal symmetries which can be extended to symmetries in a finite volume of spacetime.
A vector field $\xi$ is said to be a ``Killing vector field'' (in $B$) if $L_\xi(g) = 0$ in all of $B\subset M$.
These Killing vector fields allow us to study concepts like momentum as an incompressible liquid on our manifold that flows.
In particular it allows us to make statements like ``as much momentum flows into a volume $B$ as much flows out''.
The energy-momentum tensor was also defined, but it's very confusing.
\end{document}