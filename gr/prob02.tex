% !TEX program = xelatex
\documentclass[11pt,aspectratio=1610]{beamer}

\usetheme[
    background=light,
    numbering=fraction,
    block=fill,
    progressbar=frametitle
]{metropolis}

\usepackage{physics}
\usepackage{tcolorbox}
\newtcolorbox{prob}{colback=green!5!white,colframe=green!75!black}
\usefonttheme[onlymath]{serif}
\newcommand{\GL}[2]{\mathsf{GL}(#1;\, #2)}
\newcommand{\R}{\mathbb{R}}
\newcommand{\1}{\mathbb{1}}
\DeclareMathOperator{\id}{id}


\title{General Relativity Exam Problem 2}
% \subtitle{Is the induced map on tangent spaces linear?}
\date{Sep 16, 2021}
\author{Nate Stemen (he/they)}
\institute{AMATH 875}


\begin{document}

\maketitle

\begin{frame}{Problem Statement}
	\large
	\begin{prob}
		Compute $T_{\id}(\GL{n}{\R})$.
	\end{prob}
\end{frame}

\begin{frame}{Solution}
	\begin{columns}
		\begin{column}{0.25\textwidth}
			\begingroup
			\tiny
			\begin{prob}
				Compute $T_{\id}(\GL{n}{\R})$.
			\end{prob}
			\endgroup
		\end{column}
		\begin{column}{0.75\textwidth}
			\begin{solution}
				Recall $\GL{n}{\R}$ is defined as the \emph{group} of invertible $n\times n$ real matrices.
				Let $\gamma$ be a path in $\GL{n}{\R}$ that passes through the identity (matrix).
				To first order, in a neighborhood of the identity, we can approximate this path as $\id + t X$ for $t\in \R$ and $X\in\R^{n\times n}$.
				Differentiating we find $\dv{t}\gamma(t) = X$, where $X$ is arbitrary. Thus we have
				\begin{equation*}
					T_{\id}(\GL{n}{\R}) = \R^{n \times n} \cong \R^{n^2}
				\end{equation*}
			\end{solution}
		\end{column}
	\end{columns}
\end{frame}


\end{document}
