% !TEX program = xelatex
\documentclass[11pt,aspectratio=1610,xcolor=dvipsnames]{beamer}

\usetheme[
    background=light,
    numbering=fraction,
    block=fill,
    progressbar=frametitle
]{metropolis}

\usepackage{physics}
\usepackage{bbm}
\usepackage[most]{tcolorbox}
\newtcolorbox{prob}{colback=green!5!white,colframe=green!75!black}
\colorlet{LightLavender}{Lavender!40!}
\usefonttheme[onlymath]{serif}
\newcommand{\GL}[2]{\mathsf{GL}(#1\,; #2)}
\newcommand{\R}{\mathbb{R}}
\newcommand{\1}{\mathbbm{1}}
\newcommand{\defeq}{\stackrel{\text{\tiny def}}{=}}
\DeclareMathOperator{\id}{id}


\newcommand{\problemstatement}{Prove $\GL{n}{\R}$ is a smooth manifold, and compute $T_{\1}(\GL{n}{\R})$.}
\title{General Relativity Exam Problem 2}
\date{Sep 16, 2021}
\author{Nate Stemen (he/they)}
\institute{AMATH 875}


\begin{document}

\maketitle

\begin{frame}{Problem Statement}
	\large
	\begin{prob}
		\problemstatement
	\end{prob}
\end{frame}

\begin{frame}{Is $\GL{n}{\R}$ a manifold?}
	\begin{definition}
		$\GL{n}{\R}$ is defined to be the \emph{group} of all real invertible matrices.
		\begin{equation*}
			\GL{n}{\R} \defeq \qty{A\in \R^{n\times n} \mid \det(A) \neq 0}
		\end{equation*}
	\end{definition}
	\pause
	\begin{itemize}[<+->]
		\item $\R^{n\times n} \cong \R^{n^2}$ is trivially a manifold
		\item $\GL{n}{\R}\subset \R^{n\times n}$
		\item $\det: \R^{n\times n} \to \R$ is a continuous function
		\item $\GL{n}{\R} = \det^{-1}\qty(\R\setminus\qty{0})$
		\item Thus $\GL{n}{\R}$ is an open subset of a smooth manifold
	\end{itemize}
\end{frame}

\begin{frame}{What's the tangent space at the identity?}
	\begin{itemize}[<+->]
		\item For any $A\in\GL{n}{\R}$, there must be an open neighborhood around $A$, such that $\det(X) \neq 0$ for $X$ in such neighborhood
		\item In particular this holds for $A = \1$
		\item $\det(\1 + \varepsilon X) \neq 0$ for all $X\in\R^{n\times n}$ and some (possibly very small) $\varepsilon$
		\item This defines a path $\gamma_X(t) = \1 + tX$, such that $\gamma_X(0) = \1$ and $\dv{t}\gamma_X(0) = X$
		\item Thus \tcbox[on line,boxsep=4pt, left=0pt,right=0pt,top=0pt,bottom=0pt,colframe=white,boxrule=0pt,colback=LightLavender,highlight math style={enhanced}]{$T_\1(\GL{n}{\R}) = \R^{n\times n}$}
	\end{itemize}
\end{frame}


\end{document}
