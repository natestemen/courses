\documentclass[aspectratio=1610,xcolor=dvipsnames,mathserif]{beamer}

\usetheme[
    background=light,
    numbering=fraction,
    block=fill,
    progressbar=frametitle
]{metropolis}

\usepackage{physics}
\usepackage[most]{tcolorbox}
\newtcolorbox{prob}{colback=green!5!white,colframe=green!75!black}
\newcommand{\R}{\mathbb{R}}
\newcommand{\SO}{\mathsf{SO}}
\newcommand{\GL}[2]{\mathsf{GL}(#1\,;#2)}
\newcommand{\defeq}{\stackrel{\text{\tiny def}}{=}}


\newcommand{\problemstatement}{Give an example of two isometric (pseudo) Riemannian manifolds.}
\title{General Relativity Exam Problem 4}
\date{Oct 21, 2021}
\author{Nate Stemen (he/they)}
\institute{AMATH 875}


\begin{document}

\maketitle

\begin{frame}{Problem Statement}
	\large
	\begin{prob}
		\problemstatement
	\end{prob}
\end{frame}

\begin{frame}{Recall}
	\begin{itemize}
		\item Let $(M, g)$ and $(N, h)$ be two (pseudo) Riemannian manifolds.
		\item Let $f:M\to N$ be a diffeomorphism.
	\end{itemize}
	\pause
	\begin{exampleblock}{Pullback of metric tensor}
		We can pull back the metric on $N$ to one on $M$ by precomposition:
		\begin{equation*}
			f^* h \defeq h(f(-), f(-)).
		\end{equation*}
	\end{exampleblock}
	\pause
	\begin{definition}[Isometry]
		If $f$ satisfies $f^*h = g$, then $f$ is called an isometry.
	\end{definition}
\end{frame}

\begin{frame}{Our Manifolds}
	\begin{alertblock}{Setup}
		Take $M = N = \R^n$, and equip $M$ with the standard Euclidean metric $g = \ip{-}{-}$, and $N$ with $h = \ip{A\cdot -}{A\cdot -}$ where $A\in\SO(n)$ is a rotation.
	\end{alertblock}
	\pause
	\begin{itemize}[<+->]
		\item Think of this as $\R^n$ and a rotated copy $A\cdot \R^n$.
		\item Define $f:M\to N$ by $v\mapsto A^\intercal \cdot v$.
		\item $(f^*h)(a, b) \defeq h(f(a), f(b)) = \ip{A A^\intercal\, a}{A A^\intercal\, b} = \ip{a}{b} = g(a, b)$.
		\item $a$ and $b$ are arbitrary, thus $f^* h = g$, and $f$ is an isometry (as expected).
	\end{itemize}
\end{frame}

\begin{frame}{Extensions}
	\begin{itemize}
		\item This works for any $A\in\GL{n}{\R}$ where we define $f$ by $f(v) = A^{-1}\cdot v$ instead of transpose.
		\item Can also be generalized to flows on manifolds, and often Lie groups\footnote{Manifolds with a group structure that is compatible with the differential structure.} generate flows.
	\end{itemize}
\end{frame}

\end{document}
