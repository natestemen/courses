\documentclass[
	pages,
	boxes,
	color=WildStrawberry
]{homework}


\usepackage{cleveref}
\usepackage{macros}
\usepackage{qtree}
\usepackage{booktabs}
\name{Nate Stemen}
\studentid{20906566}
\email{nate@stemen.email}
\term{Fall 2021}
\course{General Relativity for Cosmology}
\courseid{AMATH 875}

\hwname{Lecture}
\hwnum{5}
\duedate{Wed, Mar 24, 2020 10:00 PM}

\colorlet{Ppink}{MidnightBlue!20}
\newtcbox{\mafbox}[1][]{on line, math upper,
	boxsep=4pt, left=0pt,right=0pt,top=0pt,bottom=0pt,
	colframe=white,colback=Ppink,
	highlight math style={enhanced}
}

\begin{document}

\begin{problem}
Consider the adjoint representation of $\slie{3}{\C}$ as a representation of $\slie{2}{\C}$ by restriction to the subalgebra $\mfr{g}_1 = \vspan_{\C}\{H_1, X_1, Y_1\} \simeq \slie{2}{\C}$.
\begin{parts}
	\part{Decompose this representation as a direct sum of irreducible representations of $\slie{2}{\C}$.}\label{part:1a}
	\part{Which isomorphism types appear in the decomposition in (a), and with what multiplicity?}\label{part:1b}
\end{parts}
\end{problem}

\begin{problem}
Recall how we constructed an irreducible complex $\slie{3}{\C}$ representation with highest weight $(1, 1)$ by considering the tensor product representation $\C^3 \otimes (\C^3)^*$.
\begin{parts}
	\part{Use the same method to construct an irreducible complex $\slie{3}{\C}$-representation with highest weight $(2, 0)$, acting on a subspace of $\C^3 \otimes \C^3$.}\label{part:2a}
	\part{Determine the dimension of this representation, along with all the weights and their multiplicities. (The multiplicity of a weight is the dimension of its weight space.)}\label{part:2b}
	\part{Decompose $\C^3 \otimes \C^3$, the tensor product of two copies of the standard $\slie{3}{\C}$-representation, into a direct sum of irreducible representations.}\label{part:2c}
\end{parts}
\end{problem}

\begin{problem}
Let $V_{m}(\C^3) = \vspan_{\C}\qty{z_1^k\, z_2^l\, z_3^{m - k - l} : 0 \leq k + l \leq m}$ and define $(\Pi_m(A)f)(z) = f(A^{-1}z)$ for $A \in \SU{3}$ and $f \in V_m(\C^3)$.
\begin{parts}
	\part{Prove that $(\Pi_m, V_m(\C^3))$ is a complex representation of $\SU{3}$.}\label{part:3a}
	\part{Find the weights for $\pi_1$ and $\pi_2$, the $\slie{3}{\C}$-representations associated to $\Pi_1$ and $\Pi_2$, respectively.}\label{part:3b}
	\part{Prove that $(\pi_1, V_1(\C^3))$ and $(\pi_2, V_2(\C^3))$ are irreducible representations of $\slie{3}{\C}$. What are their highest weights?}\label{part:3c}
\end{parts}
\end{problem}

\begin{solution}
	\ref{part:3a}
	\begin{equation*}
		\Pi_m(A)\qty(\qty\Big[\Pi_m(B)f])(z) = \qty\Big[\Pi_m(B)f](A^{-1}z) = f(B^{-1}A^{-1}z) = \qty\Big[\Pi_m(AB)f](z)
	\end{equation*}

	\ref{part:3b}
	The action of an arbitrary element $X\in\slie{3}{\C}$ under the representation $\pi_m$ is given by
	\begin{align*}
		\pi_m(X) & = -\qty(X_{11} z_1 + X_{12}z_2 + X_{13}z_3)\pdv{z_1}    \\
		         & \quad -\qty(X_{21}z_1 + X_{22}z_2 + X_{23}z_3)\pdv{z_2} \\
		         & \quad -\qty(X_{31}z_1 + X_{32}z_2 + X_{33}x_3)\pdv{z_3}
	\end{align*}
	Thus, for $H_1$ and $H_2$ we have
	\begin{align*}
		\pi_m(H_1) & = z_2\pdv{z_2} - z_1\pdv{z_1} \\
		\pi_m(H_2) & = z_3\pdv{z_3} - z_2\pdv{z_2}
	\end{align*}

	Take $m = 1$ where $V_1 = \vspan_\C\qty{z_1, z_2, z_3}$. Applying $\pi_1(H_1)$ and $\pi_1(H_2)$ to an arbitrary element $f = az_1 + bz_2 + cz_3$ and ensuring it is an eigenvector yields the following two equations:
	\begin{align*}
		(m_1 + 1)az_1 + (m_1 - 1)bz_2 + cm_1z_3 & = 0 \\
		m_2az_1 + (m_2 + 1)bz_2 + (m_2 - 1)cz_3 & = 0
	\end{align*}
	From here we can see there are three weights possible.
	\begin{table}[h]
		\centering\begin{tabular}{c c c}
			Weight    & Eigenvector & Multiplicity \\ \toprule
			$(1, -1)$ & $bz_2$      & 1            \\
			$(-1, 0)$ & $az_1$      & 1            \\
			$(0, 1)$  & $cz_3$      & 1
		\end{tabular}
		\caption{Weight Decomposition for $(\pi_1, V_1(\C^3))$}\label{tab:pione}
	\end{table}

	Take $m = 2$ where $V_2 = \vspan_\C\qty{z_1^2, z_2^2, z_3^2, z_1z_1, z_1z_3, z_2z_3}$ and we can repeat the process as above with an arbitrary element $f = az_1^2 + bz_2^2 + cz_3^2 + dz_1z_2 + ez_1z_3 + gz_2z_3$.
	\begin{align*}
		\pi_2(H_1)f & = -2az_1^2 + 2bz_2^2 - ez_1z_3 + gz_2z_3 \\
		\pi_2(H_2)f & = -2bz_2^2 + 2cz_3^2 - dz_1z_d + ez_1z_3
	\end{align*}
	From here we can read off the weights and eigenvectors, probably much easier than the equation I wrote down for the $m = 1$ case.
	\begin{table}[h]
		\centering\begin{tabular}{c c c}
			Weight    & Eigenvector & Multiplicity \\ \toprule
			$(-2, 0)$ & $az_1^2$    & 1            \\
			$(-2, 2)$ & $bz_2^2$    & 1            \\
			$(0, 2)$  & $cz_3^2$    & 1            \\
			$(0, -1)$ & $dz_1z_1$   & 1            \\
			$(-1, 0)$ & $ez_1z_3$   & 1            \\
			$(1, 0)$  & $gz_2z_3$   & 1
		\end{tabular}
		\caption{Weight Decomposition for $(\pi_2, V_2(\C^3))$}\label{tab:pitwo}
	\end{table}

	\ref{part:3c}
	To show $(\pi_1, V_1(\C^3))$ and $(\pi_2, V_2(\C^3))$ are irreps we will first show they are highest weight cyclic representations. Then using Proposition 6.14 from Hall, and the fact that all representations of $\slie{3}{\C}$ are completely reducible, we can deduce that the aforementioned representations are irreducible.

	For the $m = 1$ case we have highest weight vector $v = cz_3$ with weight $(0, 1)$. This is easily verified (although tedious) by computing $\mu_i - \mu_j = a\alpha_1 + b\alpha_2$ for the weights in \cref{tab:pione}. Thus condition 1 is satisfied. Now we can apply each $X_i$ to $v$ to see if it's annihilated.
	\begin{align*}
		\pi_1(X_1)v & = -z_2\pdv{z_1}\qty(cz_3) = 0 \\
		\pi_1(X_2)v & = -z_3\pdv{z_2}\qty(cz_3) = 0 \\
		\pi_1(X_2)v & = -z_3\pdv{z_1}\qty(cz_3) = 0
	\end{align*}
	Thus we also have condition two that $\pi_1(X_i)v = 0$. Lastly we have to verify $V_1(\C^3)$ is the smallest invariant subspace that contains $v$. We can do this by creating the ``tree'' applying all $\pi_1(Y_i)$. We use the convention of ``left'' means apply $\pi_1(Y_1)$, ``center'' means $\pi_1(Y_2)$ and ``right'' means $\pi_1(Y_3)$.

	\Tree[.$z_3$ [.0 ]
			[.$z_2$ [.$z_1$ ]
					[.0 ]
					[.0 ]]
			[.$z_1$ [.0 ]
					[.0 ]
					[.0 ]]]

	This diagram shows there no invariant subspace containing $v$ that is not the entirety of $V_1(\C^3)$. Thus $(\pi_1, V_1(\C^3))$ is a cyclic representation with highest weight $(0, 1)$ and by the argument given at the outset of~\ref{part:3c} we have an irrep.

	Now take $m = 2$ and we will run through the same process. The highest weight in \cref{tab:pitwo} is $(0, 2)$ again by (tediously) computing $\mu_i - \mu_j = a\alpha_1 + b\alpha_2$ repeatedly. We can now check if $v = cz_3^2$ is annihilated by all $\pi_2(X_i)$.
	\begin{align*}
		\pi_2(X_1)v & = -z_2\pdv{z_1}\qty(cz_3^2) = 0 \\
		\pi_2(X_2)v & = -z_3\pdv{z_2}\qty(cz_3^2) = 0 \\
		\pi_2(X_2)v & = -z_3\pdv{z_1}\qty(cz_3^2) = 0
	\end{align*}
	And again now we need to check if there is a smaller invariant subspace containing $v$.

	\Tree[.$z_3^2$ [.0 ]
			[.$z_2z_3$ [.$z_1z_3$ [.0 ] [.$z_1z_2$ ] [.$z_1^2$ [.0 ] [.0 ] [.0 ]]]
					[.$z_2^2$ [.$z_1z_2$ ] [.0 ] [.0 ]]
					[.$z_1z_2$ ]]
			[.$z_1z_3$ [.0 ]
					[.$z_1z_2$ [.$z_1^2$ ] [.0 ] [.0 ]]
					[.$z_1z_3$ ]]]

	So indeed this representation is highest weight cyclic with weight $(0, 2)$ and is thus irreducible by the above logic.
\end{solution}


\begin{problem}
In each part below, verify that $\mfr{t}$ is a Cartan subalgebra of $\mfr{g} = \operatorname{Lie}(G)$. Then find the maximal torus in $G$ corresponding to $\mfr{t}$.
\begin{parts}
	\part{$G = \SO{2n}$; $\mfr{t} = \qty{ \smqty(0 & \theta_1 & & & \\ -\theta_1 & 0 & & & \\ & & \ddots &  & \\  & & & 0 & \theta_n\\ & & & -\theta_n & 0) : \theta_i \in \R }$.}\label{part:4a}
	\part{$G = \SO{2n + 1}$; $\mfr{t} = \qty{ \smqty(0 & \theta_1 & & & & \\ -\theta_1 & 0 & & & & \\ & & \ddots & & &  \\ & & & 0 & \theta_n & \\ & & & -\theta_n & 0 & \\ & & & & & 0 ) : \theta_i \in \R }$.}\label{part:4b}
\end{parts}
\end{problem}

\begin{solution}
	\ref{part:4a}
	First lets verify $\mfr{t}$ is indeed a Cartan subalgebra. The Lie algebra $\so{2n}$ consists of $2n\times 2n$ skew-symmetric matrices, which clearly $\mfr{t}$ is a subset of. In order to show it's a subalgebra, it must be closed under the commutator, but because this is a \emph{Cartan} subalgebra we have the extra condition that $\comm{X}{Y} = 0$ for all $X, Y\in \mfr{t}$. We'll write elements in $\mfr{t}$ in block form using $R_\alpha = \smqty[0 & \alpha \\ -\alpha & 0]$.
	\begin{align*}
		\comm{\mqty(\dmat{R_{\theta_1}, \ddots, R_{\theta_n}})}{\mqty(\dmat{R_{\phi_1}, \ddots, R_{\phi_n}})} & = \smqty(\dmat{R_{\theta_1}R_{\phi_1} - R_{\phi_1}R_{\theta_1}, \ddots, R_{\theta_n}R_{\phi_n} - R_{\phi_n}R_{\theta_n}})
	\end{align*}
	Now to calculate the terms on the diagonal:
	\begin{align*}
		R_{\theta_i}R_{\phi_i} - R_{\phi_i}R_{\theta_i} & = \mqty(0               & \theta_i \\ -\theta_i & 0 )\mqty(0 & \phi_i \\ -\phi_i & 0) - \mqty(0 & \phi_i \\ -\phi_i & 0 )\mqty(0 & \theta_i \\ -\theta_i & 0) \\
		                                                & = \mqty(-\theta_i\phi_i & 0        \\ 0 & -\theta_i\phi_i) - \mqty(-\theta_i\phi_i & 0 \\ 0 & -\theta_i\phi_i) \\
		                                                & = \mqty(0               & 0        \\ 0 & 0)
	\end{align*}
	Thus everything in $\mfr{t}$ commutes, and is also closed under the bracket/commutator since the zero matrix is skew symmetric.

	Now we must show that anything that commutes with \emph{every} element of $\mfr{t}$ is also in $\mfr{t}$. That is suppose we have some $X\in\so{2n}$ such that $\comm{X}{\mfr{t}} = 0$. Writing things out in coordinates for $C = XA$ and $D = AX$ we have
	\begin{align*}
		C_{ij} & = \sum_{k = 1}^{2n}X_{ik}A_{kj} = X_{i,j + 1}A_{j + 1, j} = -\theta_j X_{i, j + 1} \\
		D_{ij} & = \sum_{k = 1}^{2n}A_{ik}X_{kj} = A_{i, i + 1}X_{i + 1,j} = \theta_j X_{i + 1, j}
	\end{align*}
	And these must be equal, so we have
	\begin{equation}\label{eq:antisym}
		\theta_i X_{i +1, j} + \theta_j X_{i, j + 1} = 0.
	\end{equation}
	When $i = j$ then $X_{i+1, i} + X_{i, i+1} = 0$, which $A$ also satisfies. Since~\cref{eq:antisym} must be satisfied for all $X\in\mfr{t}$, it must be satisfied for $X$ such that $\theta_i = 0$ for all $i\in\Z_n$ except for one $j$ where $\theta_j = 1$. Plugging these into~\cref{eq:antisym} we see $X_{i, j+1} = 0$ for all $i\neq j$. This, combined with the fact that $X\in\so{2n}$ is anti-symmetric shows that $X\in\mfr{t}$.

	Now let's compute the maximal torus corresponding to $\mfr{t}$. It'll be helpful to compute the first few powers of an element of $\mfr{t}$ to get a sense of what's going on.
	\begin{align*}
		A^2 & = \mqty(0 & \theta_1 &  &  & \\ -\theta_1 & 0 & & & \\ & & \ddots &  & \\  & & & 0 & \theta_n\\ & & & -\theta_n & 0)^{\!\!\!2} = \mqty(-\theta_1^1 & & & & \\ & -\theta_1^2 & & & \\ & & \ddots & & \\ & & & -\theta_n^2 & \\ & & & & -\theta_n^2) \\
		A^3 & = \mqty(0 & \theta_1 &  &  & \\ -\theta_1 & 0 & & & \\ & & \ddots &  & \\  & & & 0 & \theta_n\\ & & & -\theta_n & 0)^{\!\!\!3} = \mqty(0 & -\theta_1^3 & & & \\ \theta_1^3 & 0 & & & \\ & & \ddots & & \\ & & & 0 & -\theta_n^3 \\ & & & \theta_n^3 & 0)
	\end{align*}
	These give a pretty good hint what the next terms are. Hence we can write
	\begin{align*}
		\e^{A} & = \1 + A + A^2 + A^3 + \cdots                                                                       \\
		       & = \mqty(1 - \theta_1^2 + \theta_1^4 + \cdots & \theta_1 - \theta_1^3 + \theta_1^5 - \cdots &        \\ -\theta_1 + \theta_1^3 - \theta_1^5 + \cdots & 1 - \theta_1^2 + \theta_1^4 + \cdots & \\ & & \ddots) \\
		       & = \mqty(\cos\theta_1                         & \sin\theta_1                                &   &  & \\ -\sin\theta_1 & \cos\theta_1 & & \\ & & \ddots & & \\ & & & \cos\theta_n & \sin\theta_n \\ & & & -\sin\theta_n & \cos\theta_n)
	\end{align*}
	Thus maximal torus in $\SO{2n}$ is (up to isomorphism) $\diag(R_{\theta_1},\ldots,R_{\theta_n})$ where $R_\alpha = \smqty[\phantom{-}\cos\alpha & \sin\alpha \\ -\sin\alpha & \cos\alpha]$.

	\ref{part:4b}
	The computations performed above are idential for this case, where there is an additional row and column of 0's to work with. Thus the maximal torus of $\SO{2n + 1}$ is  $\diag(R_{\theta_1},\ldots,R_{\theta_n}, \1_{2\times 2})$ which is easily seen to be isomorphic to that of $\SO{2n}$. The $\1_{2\times 2}$ arises from the first term of $\e^{A} = \1 + \cdots$.
\end{solution}


\begin{problem}
\begin{parts}
	\part{Let $n \geq 3$ and let $H$ be the set of diagonal matrices in $\SO{n}$. Prove that $H$ is a maximal closed abelian subgroup of $\SO{n}$, but is not contained in any maximal torus.}\label{part:5a}
	\part{By contrast, let $H$ be any closed abelian subgroup of $\SU{n}$. Prove that $H$ is contained in a maximal torus.}\label{part:5b}
\end{parts}
\end{problem}

\begin{problem}
Let $T$ be the set of diagonal matrices in $\U{n}$ and $W$ its Weyl group. Let $S_n$ be the permutation group of $\qty{1, \ldots, n}$ and define an action of $S_n$ on $T$ by
\[
	\sigma \cdot' \diag(u_1, \ldots, u_n) = \diag(u_{\sigma^{-1}(1)}, \ldots, u_{\sigma^{-1}(n)}).
\]
(Here we put a prime in the notation to distinguish this action from the action of $W$ on $T$.) Also, take a generating element $t_0 = \diag(\e^{2\pi\iu \theta_1}, \ldots, \e^{2\pi\iu \theta_n})$ in $T$.
\begin{parts}
	\part{Given $w \in W$, prove that there exists a unique $\sigma \in S_n$ such that
		\[
			w \cdot t_0 = \sigma \cdot' t_0.
		\]
		Deduce that $w \cdot t = \sigma \cdot' t$ for all $t \in T$.}\label{part:6a}
	\part{In the notation of part (a), prove that the map $w \mapsto \sigma$ defines an injective homomorphism from $W$ into $S_n$.}\label{part:6b}
	\part{Prove that the homomorphism in part (b) is also surjective. (Consequently, $W$ is isomorphic to $S_n$.)}\label{part:6c}
\end{parts}

\end{problem}

\begin{problem}
Let $G$ be a compact connected matrix Lie group.
\begin{parts}
	\part{Let $f: G \to H$ be a surjective Lie group homomorphism from $G$ onto another compact connected matrix Lie group. Prove that if $T$ is a maximal torus in $G$ then $f(T)$ is a maximal torus in $H$. Deduce that if $H$ is abelian then the restriction $f|_{T}$ is surjective already.}\label{part:8a}
	\part{Given $g \in G$ and $n \in \N$, prove that there exists $h \in G$ such that $h^n = g$.}\label{part:8b}
\end{parts}

\end{problem}

\begin{solution}
	\ref{part:8a}
	Since $T$ is connected and compact, and $f$ is a continuous function, $f(T)$ is also connected and compact. Since $f$ is a homomorphism we have $f(a)f(b) = f(ab) = f(ba) = f(b)f(a)$ if $a, b\in T$, and thus $f(T)$ is also commutative and by Theorem 11.2 in Hall, $f(T)$ is a torus. To show $f(T)$ is maximal

	\ref{part:8b}
	Since $G$ is compact and connected, we know exp: $\mathsf{g}\to G$ is surjective, and hence for all $g\in G$ we can fine an $A\in\mathsf{g}$ such that $g = \e^{A}$. In particular we can also define $h\defeq \e^{A/n}$ so that $h^n = g$.
\end{solution}

\end{document}