\documentclass[
	pages,
	boxes,
	color=WildStrawberry
]{homework}


\usepackage{cleveref}
\usepackage{macros}
\usepackage{todonotes}
\usepackage{tikz-cd}
\usepackage{pythonhighlight}
\name{Nate Stemen}
\studentid{20906566}
\email{nate@stemen.email}
\term{Fall 2021}
\course{General Relativity for Cosmology}
\courseid{AMATH 875}

\hwname{Lecture}
\hwnum{5}
\duedate{Wed, Mar 24, 2020 10:00 PM}

\colorlet{Ppink}{MidnightBlue!20}
\newtcbox{\mafbox}[1][]{on line, math upper,
	boxsep=4pt, left=0pt,right=0pt,top=0pt,bottom=0pt,
	colframe=white,colback=Ppink,
	highlight math style={enhanced}
}

\begin{document}

\begin{problem}
Consider the adjoint representation of $\slie{3}{\C}$ as a representation of $\slie{2}{\C}$ by restriction to the subalgebra $\mfr{g}_1 = \vspan_{\C}\{H_1, X_1, Y_1\} \simeq \slie{2}{\C}$.
\begin{parts}
	\part{Decompose this representation as a direct sum of irreducible representations of $\slie{2}{\C}$.}\label{part:1a}
	\part{Which isomorphism types appear in the decomposition in (a), and with what multiplicity?}\label{part:1b}
\end{parts}
\end{problem}

\begin{problem}
Recall how we constructed an irreducible complex $\slie{3}{\C}$ representation with highest weight $(1, 1)$ by considering the tensor product representation $\C^3 \otimes (\C^3)^*$.
\begin{parts}
	\part{Use the same method to construct an irreducible complex $\slie{3}{\C}$-representation with highest weight $(2, 0)$, acting on a subspace of $\C^3 \otimes \C^3$.}\label{part:2a}
	\part{Determine the dimension of this representation, along with all the weights and their multiplicities. (The multiplicity of a weight is the dimension of its weight space.)}\label{part:2b}
	\part{Decompose $\C^3 \otimes \C^3$, the tensor product of two copies of the standard $\slie{3}{\C}$-representation, into a direct sum of irreducible representations.}\label{part:2c}
\end{parts}
\end{problem}

\begin{problem}
Let $V_{m}(\C^3) = \vspan_{\C}\qty{z_1^k\, z_2^l\, z_3^{m - k - l} : 0 \leq k + l \leq m}$ and define $(\Pi_m(A)f)(z) = f(A^{-1}z)$ for $A \in \SU{3}$ and $f \in V_m(\C^3)$.
\begin{parts}
	\part{Prove that $(\Pi_m, V_m(\C^3))$ is a complex representation of $\SU{3}$.}\label{part:3a}
	\part{Find the weights for $\pi_1$ and $\pi_2$, the $\slie{3}{\C}$-representations associated to $\Pi_1$ and $\Pi_2$, respectively.}\label{part:3b}
	\part{Prove that $(\pi_1, V_1(\C^3))$ and $(\pi_2, V_2(\C^3))$ are irreducible representations of $\slie{3}{\C}$. What are their highest weights?}\label{part:3c}
\end{parts}
\end{problem}


\begin{problem}
In each part below, verify that $\mfr{t}$ is a Cartan subalgebra of $\mfr{g} = \operatorname{Lie}(G)$. Then find the maximal torus in $G$ corresponding to $\mfr{t}$.
\begin{parts}
	\part{$G = \SO{2n}$; $\mfr{t} = \qty{ \smqty(0 & \theta_1 & & & \\ -\theta_1 & 0 & & & \\ & & \ddots &  & \\  & & & 0 & \theta_n\\ & & & -\theta_n & 0) : \theta_i \in \R }$.}\label{part:4a}
	\part{$G = \SO{2n + 1}$; $\mfr{t} = \qty{ \smqty(0 & \theta_1 & & & & \\ -\theta_1 & 0 & & & & \\ & & \ddots & & &  \\ & & & 0 & \theta_n & \\ & & & -\theta_n & 0 & \\ & & & & & 0 ) : \theta_i \in \R }$.}\label{part:4b}
\end{parts}
\end{problem}


\begin{problem}
\begin{parts}
	\part{Let $n \geq 3$ and let $H$ be the set of diagonal matrices in $\SO{n}$. Prove that $H$ is a maximal closed abelian subgroup of $\SO{3}$, but is not contained in any maximal torus.}\label{part:5a}
	\part{By contrast, let $H$ be any closed abelian subgroup of $\SU{n}$. Prove that $H$ is contained in a maximal torus.}\label{part:5b}
\end{parts}
\end{problem}

\begin{problem}
Let $T$ be the set of diagonal matrices in $\U{n}$ and $W$ its Weyl group. Let $S_n$ be the permutation group of $\qty{1, \ldots, n}$ and define an action of $S_n$ on $T$ by
\[
	\sigma \cdot' \diag(u_1, \ldots, u_n) = \diag(u_{\sigma^{-1}(1)}, \ldots, u_{\sigma^{-1}(n)}).
\]
(Here we put a prime in the notation to distinguish this action from the action of $W$ on $T$.) Also, take a generating element $t_0 = \diag(\e^{2\pi\iu \theta_1}, \ldots, \e^{2\pi\iu \theta_n})$ in $T$.
\begin{parts}
	\part{Given $w \in W$, prove that there exists a unique $\sigma \in S_n$ such that
		\[
			w \cdot t_0 = \sigma \cdot' t_0.
		\]
		Deduce that $w \cdot t = \sigma \cdot' t$ for all $t \in T$.}\label{part:6a}
	\part{In the notation of part (a), prove that the map $w \mapsto \sigma$ defines an injective homomorphism from $W$ into $S_n$.}\label{part:6b}
	\part{Prove that the homomorphism in part (b) is also surjective. (Consequently, $W$ is isomorphic to $S_n$.)}\label{part:6c}
\end{parts}

\end{problem}

\begin{problem}
Let $G$ be a compact connected matrix Lie group.
\begin{parts}
	\part{Let $f: G \to H$ be a surjective Lie group homomorphism from $G$ onto another compact connected matrix Lie group. Prove that if $T$ is a maximal torus in $G$ then $f(T)$ is a maximal torus in $H$. Deduce that if $H$ is abelian then the restriction $f|_{T}$ is surjective already.}\label{part:8a}
	\part{Given $g \in G$ and $n \in \N$, prove that there exists $h \in G$ such that $h^n = g$.}\label{part:8b}
\end{parts}

\end{problem}

\begin{solution}
\end{solution}

\end{document}