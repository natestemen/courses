\documentclass[
	pages,
	boxes,
	color=WildStrawberry
]{homework}


\usepackage{cleveref}
\usepackage{macros}
\usepackage{booktabs}
\usepackage{pythonhighlight}
\usepackage{lscape}
\name{Nate Stemen}
\studentid{20906566}
\email{nate@stemen.email}
\term{Fall 2021}
\course{General Relativity for Cosmology}
\courseid{AMATH 875}

\hwname{Lecture}
\hwname{Final Assessment}
\hwnum{}
\duedate{Fri, Apr 23, 2020 10:00 PM}

\colorlet{Ppink}{MidnightBlue!20}
\newtcbox{\mafbox}[1][]{on line, math upper,
	boxsep=4pt, left=0pt,right=0pt,top=0pt,bottom=0pt,
	colframe=white,colback=Ppink,
	highlight math style={enhanced}
}

\theoremstyle{plain}
\newtheorem{fact}{Fact}

\begin{document}

\begin{problem}
Recall that the center of a matrix Lie group $G$ is by definition $Z(G)\defeq \qty{g\in G : gh = hg \text{ for all } h\in G}$. Below we let $n\geq 2$.
\begin{parts}
	\part{With the help of Schur's lemma, determine the centers of $\U{n}$ and $\SU{n}$. Deduce that $\U{n}$ and $\SU{n}\times\U{1}$ are not isomorphic as Lie groups.}\label{part:1a}
	\part{Prove that $\U{n}$ and $\SU{n}\times\U{1}$ nonetheless have isomorphic Lie algebras.}\label{part:1b}
\end{parts}
\end{problem}

\begin{solution}
	\ref{part:1a}
	Let $A\in Z(\U{n})$, then by the definition of $\U{n}$ we can view $A$ as a map taking $\U{n}$ to itself $\U{n}\xrightarrow{\,\,\, A\,\,\,} \U{n}$. Let $\rho: \U{n}\to \GL{n}{\C}$ be the representation defined by $\rho(X) = X$. Then $A$ intertwines $\rho$:
	\begin{equation*}
		A\circ\rho(X) = A(\rho(X)) = AX = XA = \rho(X)\circ A.
	\end{equation*}
	By Schur's lemma $A$ must then either be $0$ or a scalar multiple of the identity $\1$. The unitary group does not include $0$ so $A = \lambda \1$, and in addition
	\begin{equation*}
		AA^\dagger = \lambda \overline{\lambda}\1 = \1 \implies \lambda = \e^{\iu \varphi}.
	\end{equation*}
	Thus the center of $\U{n}$ is $\qty{\e^{\iu\varphi}\1 : \varphi\in\R}$ which is isomorphic to $S^1$ the circle group.

	The above argument applies to $\SU{n}$ as well, but we have the extra condition that $\det A = 1$. By properties of the determinant\footnote{$\det(\alpha A) = \alpha^n\det A$ if $A\in\mats{n}{\F}$.} we know $\det A = \e^{\iu n\varphi} = 1$ so the center of $\SU{n}$ is the group of $n$-th roots of unity. The $n$-th roots of unity are known to be isomorphic to $\Z/n\Z$, so we will say $Z(\SU{n}) = \Z/n\Z$.

	Further we have $Z(\SU{n}\times \U{1}) = \Z/n\Z\times S^1$ which is clearly not isomorphic to $S^1$. Now if $\U{n}$ was isomorphic to $\SU{n}\times\U{1}$, then their centers would also be isomorphic, but they are not, so they cannot be isomorphic as groups, and hence Lie groups.

	\ref{part:1b}
	First we define the following function $\Phi: \SU{n}\times\U{1}\to \U{n}$:
	\begin{equation*}
		\Phi(X, \alpha) \defeq \alpha X.
	\end{equation*}
	This is surely an element of $\U{n}$ because $\alpha X (\alpha X)^\dagger = \alpha\overline{\alpha}XX^\dagger = \1$. It is also a group homomorphism
	\begin{equation*}
		\Phi(X, \alpha)\Phi(Y, \beta) = \alpha X \beta Y = \alpha\beta XY = \Phi(XY, \alpha\beta)
	\end{equation*}
	and is clearly continuous since we're just doing scalar multiplication. Thus $\Phi$ is a Lie group homomorphism. To see this map is a surjection, take $X\in\U{n}$ and we will find a pair $(Y, \alpha)\in\SU{n}\times\U{1}$ that hits it. Some inspection yields
	\begin{align*}
		Y = \frac{X}{(\det X)^{1/n}} &  & \alpha = 1.
	\end{align*}
	Thus $\Phi$ is a surjection. We now show $\Phi$ also has a discrete kernel. If $\Phi(X, \alpha) = \alpha X = \1$, then first $X$ must be diagonal, and hence also in the center of $\SU{n}$. As we found above $Z(\SU{n})\cong\Z/n\Z$, and hence $\ker\Phi$ is discrete.

	By Proposition 3.31 of Hall (page 63), we have $\Lie(\ker\Phi) = \ker\phi$ where $\phi$ is a Lie algebra homomorphism between the Lie algebras of $\SU{n}\times\U{1}$ and $\U{n}$. However, the Lie algebra of a discrete group is trivial, and hence $\ker\phi$ is trivial which implies that $\phi$ is an isomorphism of Lie algebras.
\end{solution}

\begin{problem}
\begin{parts}
	\part{Prove that every element of $\SO{3}$ except the identity belongs to exactly one maximal torus in $\SO{3}$.}\label{part:2a}
	\part{Consider the following maximal torus in $\SO{3}$:
		\begin{equation*}
			T = \qty{\smqty(\cos{\theta} & -\sin{\theta} & 0 \\ \sin{\theta} & \phantom{-}\cos{\theta} & 0 \\ 0 & 0 & 1) : \theta\in\R}
		\end{equation*}
		Prove that it's Weyl group is isomorphic to the finite abelian group $\Z_2$.}\label{part:2b}
\end{parts}
\end{problem}

\begin{solution}
	\ref{part:2a}
	Using the maximal torus $\leftrightarrow$ Cartan subalgebra correspondence we can pass to $\so{3}$ and rephrase the problem as ``prove that every vector of $\so{3}$ except the zero vector belongs to exactly one maximal torus in $\so{3}$.'' Here we have the following basis
	\begin{align*}
		e_1 = \mqty[0 & 1 & 0 \\ -1 & 0 & 0 \\ 0 & 0 & 0] && e_2 = \mqty[0 & 0 & 1 \\ 0 & 0 & 0 \\ -1 & 0 & 0] && e_3 = \mqty[0 & 0 & 0 \\ 0 & 0 & 1 \\ 0 & -1 & 0]
	\end{align*}
	with the following commutation relations:
	\begin{align*}
		\comm{e_1}{e_2} = -e_3 &  & \comm{e_1}{e_3} = e_2 &  & \comm{e_2}{e_3} = -e_1.
	\end{align*}
	Since $\so{3}$ is not commutative there are no 3-dimensional Cartan subalgebras. The commutation relations also show there are no 2-dimensional Cartan subalgebras since we never have $\comm{e_i}{e_j} = e_i$. Hence all the Cartan subalgebras are 1-dimensional, and because they must span $\so{3}$, they only intersect at the origin. We can now exponentiate these Cartan subalgebras to obtain maximal tori which also only intersect at $\exp(\smqty[0 & 0 & 0 \\ 0 & 0 & 0 \\ 0 & 0 & 0]) = \1_3$.

	The first way I thought about this problem was to show that rotations around the 3-axis are all maximal tori, but I couldn't figure out how to show they were the \emph{only} maximal tori.

	\ref{part:2b}
	First note that $T\subseteq N(T)$ almost by definition. Now to calculate what else is in $N(T)$ we need a general paremetrization of elements in $\SO{3}$ and for that we look to the Euler angle decomposition. In particular if $R_z(\theta)$ is a rotation of angle $\theta$ about the $z$-axis, then any element $A\in\SO{3}$ can be written as
	\begin{equation*}
		A = R_z(\alpha)R_x(\beta)R_z(\gamma).
	\end{equation*}
	Thus we need to figure out for which $\alpha, \beta, \gamma\in[0, 2\pi)$ does $A R_z(\theta) A^\intercal = R_z(\phi)$. Doing the algebra by hand is possible, but thankfully we have computers. % chktex 9
	% \begin{noindent}
	\begin{python}
import sympy as sym
from sympy.abc import alpha, beta, gamma, theta
from sympy.matrices import rot_axis1, rot_axis3

r = rot_axis3(alpha) * rot_axis1(beta) * rot_axis3(gamma)
normalized = sym.simplify(r * rot_axis3(theta) * r.T)\end{python}
% \end{noindent}
	Now we have \texttt{normalized}\footnote{I'll include a page with the entire matrix printed, but it's too big (and not very helpful) to put here.} computed as $A R_z(\theta) A^\intercal$ and we know it should look like another $z$-rotation. Thus $[A]_{33} = 1$ because $z$-rotations fix the $z$-axis. We can then access the $[A]_{33}$ element with \texttt{normalized[8]}:
	% \begin{noindent}
	\begin{python}
print(normalized[8])\end{python}
% \end{noindent}
	\begin{equation}\label{eq:betagood}
		\displaystyle \sin^{2}{\left(\beta \right)} \cos{\left(\theta \right)} - \sin^{2}{\left(\beta \right)} + 1
	\end{equation}
	Since this must be equal to 1 for all $\theta$ we must have $\beta = 0, \pi$. If $\beta = 0$, then $A$ is purely a rotation around the $z$-axis which we already knew was in $N(T)$. In the case $\beta = \pi$, we can take $\alpha = 0 = \gamma$ to conclude the following matrix is in $N(T)$:
	\begin{equation}\label{eq:flipx}
		\mqty[1 & 0 & 0 \\ 0 & -1 & 0 \\ 0 & 0 & -1].
	\end{equation}
	Since this is clearly not in $T$ (it doesn't have a $+1$ in the bottom right hand corner). Indeed these $\beta$ are the only possible values because of~\cref{eq:betagood}, and we can verify they work with the computer.
	% \begin{noindent}
	\begin{python}
print(normalized.subs( { beta: sym.pi } ))\end{python}
% \end{noindent}
	\begin{equation*}
		\left[\begin{matrix}\cos{\left(\theta \right)} & - \sin{\left(\theta \right)} & 0\\\sin{\left(\theta \right)} & \cos{\left(\theta \right)} & 0\\0 & 0 & 1\end{matrix}\right]
	\end{equation*}
	Indeed we have the same thing with $\beta = 0$, but that can be seen more easily from $A = R_z(\alpha)R_x(\beta=0)R_z(\gamma) = R_z(\alpha + \gamma)$.

	Thus the Weyl group $N(T) / T$ contains two elements: one being the identity $[\1]$, and the other from~\cref{eq:flipx} which we will call $[x]$. They have the following multiplicative structure:
	\begin{align*}
		[\1] \cdot [\1] = [\1] &  & [\1] \cdot [x] = [x] &  & [x] \cdot [x] = \1
	\end{align*}
	This is exactly the structure of $\Z/2\Z$ under the map $[\1]\to [0]$ and $[x]\to [1]$.
\end{solution}

\begin{problem}
Recall the $\SU{2}$-representation $V_m(\C^2)$. For $0\leq k \leq m$ we write $f_k(z_1, z_2) = z_1^k z_2^{m - k}$. Prove the following defines an $\SU{2}$ invariant inner product on $V_m(\C^2)$:
\begin{equation*}
	\qty(\sum_{k = 0}^m \alpha_k f_k, \sum_{l = 0}^m\beta_l f_l) = \sum_{k = 0}^m k!(m - k)!\alpha_k\overline{\beta_k}.
\end{equation*}
\end{problem}

\begin{solution}

\end{solution}

\begin{problem}
Define $V_{1, 1}(\C^3)\defeq \vspan_\C\qty{z_i\overline{z_j} : i, j\in\qty{1,2,3}}$. For $f\in V_{1,1}(\C^3)$ and $A\in \SU{3}$ we let $(\Pi_{1,1}(A)\cdot f)(z)\defeq f(A^{-1}z)$. Also, let $\laplace \equiv \sum_{j = 1}^3 \pdv{z_j}\pdv{\overline{z_j}}$ and define $\mathcal{H}_{1,1}(\C^3)\defeq \qty{f\in V_{1, 1}(\C^3) : \laplace{f} = 0}$.
\begin{parts}
	\part{Prove that $(\Pi_{1,1}, V_{1,1}(\C^3))$ is a complex representation of $\SU{3}$ and that $\mathcal{H}_{1,1}(\C^3)$ is an invariant subspace.}\label{part:4a}
	\part{Consider the associated $\slie{3}{\C}$-representation on $\mathcal{H}_{1,1}(\C^3)$. Prove that this is irreducible, and find it's dimension and highest weight.}\label{part:4b}
\end{parts}
\end{problem}

\begin{solution}
	\ref{part:4a}
	First let's the action of $\Pi_{1,1}(A)$ does indeed bring us back into $V_{1,1}(\C^3)$. We'll represent an arbitrary vector $f\in V_{1,1}(\C^3)$ as $f = \sum_{i,j=1}^3 a_{ij}z_i\overline{z_j}$.
	\begin{align*}
		\qty\Big[\Pi_{1,1}(A)\cdot f](z) & = f(A^{-1}z)                                                                                                                                                                                          \\
		                                 & = \sum_{i,j=1}^3 a_{ij}\qty(a_{i1}^{-1}z_1 + a_{i2}^{-1}z_2 + a_{i3}^{-1}z_1)\qty(\overline{a_{j1}^{-1}}\overline{z_j} + \overline{a_{j2}^{-1}}\overline{z_2} + \overline{a_{j3}^{-1}}\overline{z_3})
	\end{align*}
	Written out this way it's not hard to see this is again an element of $V_{1,1}(\C^3)$. To see this map is also a Lie group homomorphism we can look at the action of $A$ and $B$ on $f$.
	\begin{align*}
		\qty(\Pi_{1,1}(A)\cdot\qty\Big[\Pi_{1,1}(B)\cdot f])(z) & = \qty\Big[\Pi_{1,1}(B)\cdot f](A^{-1}z) \\
		                                                        & = f(B^{-1}A^{-1}z)                       \\
		                                                        & = \qty\Big[\Pi_{1,1}(AB)\cdot f](z)
	\end{align*}
	That $\Pi_{1,1}$ is continuous follows from the fact that matrix inversion is continuous.

	Now we compute what $\mathcal{H}_{1,1}(\C^3)$ looks like with respect to $\laplace$ defined above. I'll use $\partial_i\equiv \pdv{z_i}$ and $\partial_{\overline{i}}\equiv \pdv{\overline{z_i}}$. We have
	\begin{align*}
		\partial_k \partial_{\overline{k}}\sum_{i,j=1}^3 a_{ij}z_i\overline{z_j} = \partial_k \sum_{i,j=1}^3 a_{ij}z_i\delta_{jk} = \partial_k \sum_{i=1}^3a_{ik}z_i = a_{kk}
	\end{align*}
	and thus for a general element $f$ we have
	\begin{align*}
		\laplace{f} = \sum_{i=1}^3a_{ii}
	\end{align*}
	and thus
	\begin{equation*}
		\mathcal{H}_{1,1}(\C^3) = \qty{f\in V_{1,1}(\C^3) : \sum a_{ii} = 0}.
	\end{equation*}
	To show this is an invariant subspace we will show that the action of $U\in \SU{3}$ commutes with the action of the Laplacian $\laplace$.
	\begin{align*}
		\partial_{\overline{i}}(f(Uz))           & = \sum_{k = 1}^3 \partial_{\overline{i}}f(Uz)\overline{u_{ki}}                              \\
		\partial_i\partial_{\overline{i}}(f(Uz)) & = \sum_{k, l = 1}^3\partial_i\partial_{\overline{i}}f(Uz)\overline{u_{ki}}u_{li}            \\
		\laplace{\qty\big[f(Uz)]}                & = \sum_{i, k, l = 1}^3\partial_i\partial_{\overline{i}}f(Uz)\delta_{kl} = (\laplace{f})(Uz)
	\end{align*}
	Thus if $\laplace{f}(z)\equiv 0$, then $\laplace{f(Uz)} = 0$.

	\ref{part:4b}
	The above representation of $\SU{3}$ gives us a representation of the Lie algebra $\su{3}$, and by complex linearity a representation on the complexification $\su{3}_\C \cong \slie{3}{\C}$ which we can calculate with
	\begin{equation*}
		\pi_{1,1}(X) \defeq \dv{t}\eval{\Pi_{1,1}(\e^{tX})}_{t=0}.
	\end{equation*}
	Acting on an element $f\in V_{1,1}(\C^3)$ and $z\equiv z(t) = \smqty[z_1(t) \\ z_2(t) \\ z_3(t)]$ we have
	\begin{align*}
		\qty\Big[\pi_{1,1}(X)\cdot f](z) & = \dv{t}\eval{f(\e^{-tX}z)}_{t=0}                                                                                       \\
		                                 & = \pdv{f}{z_1}\eval{\pdv{z_1}{t}}_{t=0} + \pdv{f}{z_2}\eval{\pdv{z_2}{t}}_{t=0} + \pdv{f}{z_3}\eval{\pdv{z_3}{t}}_{t=0} \\
		                                 & = -\pdv{f}{z_1}\qty(X_{11}z_1 + X_{12}z_2 + X_{13}z_3)                                                                  \\
		                                 & \qquad - \pdv{f}{z_2}\qty(X_{21}z_1 + X_{22}z_2 + X_{23}z_3)                                                            \\
		                                 & \qquad - \pdv{f}{z_3}\qty(X_{31}z_1 + X_{32}z_2 + X_{33}z_3)
	\end{align*}
	We can now calculate $\pi_{1,1}(X)$ for every basis element of $\slie{3}{\C}$ using the basis defined in Hall on page 142.
	\begin{equation*}
		\pi_{1,1}(H_1) = -z_1\pdv{z_1} + z_2\pdv{z_2} \qquad \pi_{1,1}(H_2) = -z_2\pdv{z_2} + z_3\pdv{z_3}      \\
	\end{equation*}
	\begin{align*}
		\pi_{1,1}(X_1) = -z_2\pdv{z_1} &  & \pi_{1,1}(X_2) = -z_3\pdv{z_2} &  & \pi_{1,1}(X_3) = -z_3\pdv{z_1} \\
		\pi_{1,1}(Y_1) = -z_1\pdv{z_2} &  & \pi_{1,1}(Y_2) = -z_2\pdv{z_3} &  & \pi_{1,1}(Y_3) = -z_1\pdv{z_3}
	\end{align*}
	Now, $H_1$ and $H_2$ applied to an arbitrary element of $\mathcal{H}_{1,1}(\C^3)$:
	\begin{align*}
		\pi_{1,1}(H_1)\cdot f & = -a_{11}z_1\conj{z_1} - a_{12}z_1\conj{z_2} - a_{13}z_1\conj{z_3} + a_{21}z_2\conj{z_1} + a_{22}z_2\conj{z_2} + a_{23}z_2\conj{z_3}   \\
		\pi_{1,1}(H_2)\cdot f & =  - a_{21}z_2\conj{z_1} - a_{22}z_2\conj{z_2} - a_{23}z_2\conj{z_3} + a_{31}z_3\conj{z_1} + a_{32}z_3\conj{z_2} + a_{33}z_3\conj{z_3}
	\end{align*}
	Where we have to remember $a_{33}$ can be written as $-(a_{11} + a_{22})$ because $f$ is harmonic. From the action of $H_1$ and $H_2$ we can see we have three weight spaces.
	\begin{table}[h]
		\centering\begin{tabular}{c c}
			Vector $v$                                                        & Weight $\mu$ \\ \toprule
			$a_{11}z_1\conj{z_1} + a_{12}z_1\conj{z_2} + a_{13}z_1\conj{z_3}$ & $(-1, 0)$    \\
			$a_{21}z_2\conj{z_1} + a_{22}z_2\conj{z_2} + a_{23}z_2\conj{z_3}$ & $(1, -1)$    \\
			$a_{31}z_3\conj{z_1} + a_{32}z_3\conj{z_2} + a_{33}z_3\conj{z_3}$ & $(0, 1)$
		\end{tabular}
		\caption{Weight Spaces for $\mathsf{h} = \vspan(H_1, H_2)$}\label{tab:weightspaces}
	\end{table}
	Since we are looking for an \emph{irreducible} representation we will look at the vector subspace with a weight composed of \emph{non-negative} integers. We now check if the associated vector gives us the satisfying conditions to make this representation one of highest weight cyclic. It's not hard to see because $\pi_{1,1}(X_i)$ only involves the partial derivatives of $z_1$ and $z_2$, that all of $\pi_{1,1}(X_i)\cdot v = 0$ for all $i = 1,2,3$.

	We now apply $\pi_{1,1}(Y_i)$ to this vector to see if $v$ ``generates'' the entire vector space.
	\begin{align*}
		\pi_{1,1}(Y_1) \cdot v & = 0                                                                \\
		\pi_{1,1}(Y_2) \cdot v & = -a_{31}z_2\conj{z_1} - a_{32}z_2\conj{z_2} - a_{33}z_2\conj{z_3} \\
		\pi_{1,1}(Y_3) \cdot v & = -a_{31}z_1\conj{z_1} - a_{32}z_1\conj{z_2} - a_{33}z_1\conj{z_3}
		% \pi_{1,1}(Y_1)\pi_{1,1}(Y_2)\cdot v & = a_{31}z_1\conj{z_1} + a_{32}z_1\conj{z_2} + a_{33}z_1\conj{z_3}
	\end{align*}
	Thus the only invariant subspace containing $v$ is $\mathcal{H}_{1,1}(\C^3)$ itself, and hence this representation is highest weight cyclic. We now use two facts from Hall, and one from class:
	\begin{fact}[Theorem 10.9, Hall page 275]
		Every finite-dimensional representation of a semisimple Lie algebra is completely reducible.
	\end{fact}
	\begin{fact}[Lecture 65]
		$\slie{n}{\C}$ is semisimple.
	\end{fact}
	\begin{fact}[Proposition 6.14, Hall page 150]
		Suppose $(\pi, V)$ is a \emph{completely reducible} representation of $\slie{3}{\C}$ that is also a highest weight cyclic. Then $\pi$ is irreducible.
	\end{fact}
	Combining all these we see $(\pi_{1,1}, \mathcal{H}_{1,1}(\C^3))$ is indeed irreducible with dimension $8$ and highest weight $(0, 1)$.

	I'll be honest this \emph{feels} wrong (like it should have a higher highest weight), but I really can't find anything wrong.
\end{solution}

\begin{problem}
Recall the following basis for $\slie{2}{\R}$ (also for $\slie{2}{\C}$):
\begin{equation*}
	H = \mqty(1 & \phantom{-}0 \\ 0 & -1),\quad X = \mqty(0 & 1 \\ 0 & 0),\quad Y = \mqty(0 & 0 \\ 1 & 0)
\end{equation*}
\begin{parts}
	\part{Prove that $\ad_X: \slie{2}{\C} \to \slie{2}{\C}$ is not diagonalizable, while $\ad_A: \gl{n}{\C} \to \gl{n}{\C}$ is diagonalizable for all $A\in \u{n}$. Deduce that $\u{n}$ contains no subalgebra isomorphic to $\slie{2}{\R}$.}\label{part:5a}
	\part{Prove that the only ideals of $\slie{2}{\R}$ are $\qty{0}$ and $\slie{2}{\R}$. Deduce from this and part (a) that there are no non-trivial Lie algebra homomorphisms from $\slie{2}{\R}$ into $\u{n}$.}\label{part:5b}
	\part{Deduce from parts (a) and (b) that every finite-dimensional unitary complex representation $(\Pi, V)$ of $\SL{2}{\R}$ is trivial in the sense that $\Pi(A) = \id_V$ for all $A\in \SL{2}{\R}$. You may take for granted that $\SL{2}{\R}$ is connected.}\label{part:5c}
\end{parts}
\end{problem}

\begin{solution}
	\ref{part:5a}
	Since the action of $\ad_X$ is determined entirely on the action on the basis elements we recall those here:
	\begin{align}\label{eq:commrel}
		\ad_X(H) = -2X &  & \ad_X(X) = 0 &  & \ad_X(Y) = H
	\end{align}
	Since $\ad_X$ is a linear operator on $\slie{2}{\C}$ we can write it as a $3\times 3$ complex matrix if we use the following isomorphism:
	\begin{align*}
		Y\to \smqty[1 \\ 0 \\ 0] && H \to \smqty[0 \\ 1 \\ 0] && X \to \smqty[ 0 \\ 0 \\ 1].
	\end{align*}
	With this we can treat $\slie{2}{\C}$ like $\C^3$ and hence can compute the matrix representation of $\ad_X$. Indeed a short calculation yields
	\begin{equation}\label{eq:adx}
		\ad_X = \mqty[ 0 & 0 & 0 \\ 1 & 0 & 0 \\ 0 & -2 & 0].
	\end{equation}
	To see if this matrix is diagonalizable (which maybe you can just tell, but spelling it out more carefully was helpful for me) we can use the following fact.
	\begin{fact}
		A matrix is diagonalizable if and only if for every eigenvalue, it's corresponding algebraic and geometric multiplicity are equal.
	\end{fact}
	Calculating the characteristic polynomial of~\cref{eq:adx} we find
	\begin{equation*}
		\det(\smqty[ 0 & 0 & 0 \\ 1 & 0 & 0 \\ 0 & -2 & 0] - \lambda \1_3) = -\lambda^3
	\end{equation*}
	Hence the only eigenvalue is $0$, but~\cref{eq:commrel} shows the only eigenvector with eigenvalue $0$ is $X$. Clearly the eigenspace spanned by $X$ is not three dimensional despite having algebraic multiplicity $3$. Thus $\ad_X$ is diagonalizable.

	To see that $\ad_A$ is diagonalizable

	\ref{part:5b}
	Let $\mathsf{i}\subseteq\slie{2}{\R}$ be our nonempty ideal. Since it is nonempty it must contain some element of the form
	\begin{equation}\label{eq:idealelement}
		\alpha H + \beta X + \gamma Y.
	\end{equation}
	Since $\mathsf{i}$ is an ideal we must have $\comm{\slie{2}{\R}}{\mathsf{i}}\subseteq \mathsf{i}$, and in particular we can look at~\cref{eq:idealelement} under $\ad_H$.
	\begin{equation*}
		\comm{H}{\alpha H + \beta X + \gamma Y} = \beta\comm{H}{X} + \gamma\comm{H}{Y} = 2\beta X - 2\gamma Y.
	\end{equation*}
	Thus there are two cases we must consider: $\alpha = 0$ and $\alpha \neq 0$ where $\mathsf{i}$ contains $H$.

	If $H\in\mathsf{i}$, then by definition of an ideal we must have $\comm{X}{H} = -2X\in\mathsf{i}$ and $\comm{Y}{H} = 2Y\in\mathsf{i}$ and thus $\mathsf{i} = \slie{2}{\R}$.

	If $\alpha = 0$ and the ideal $\mathsf{i}$ only contains elements of the form $\beta X + \gamma Y$, we again must have $\comm{\slie{2}{\R}}{\mathsf{i}}\subseteq \mathsf{i}$ by definition. In particular we can choose to take $\comm{X}{2\beta X - 2\gamma Y} = -2\gamma H$, and hence if $\gamma \neq 0$ then $\mathsf{i}$ must contain $H$. We can then use the argument as above to conclude the ideal is equal to the entire group. If $\gamma = 0$, and the ideal only contains $\beta X$, we can look at $\comm{Y}{\beta X} = -\beta H$ and again if $\beta \neq 0$ then the ideal contains $H$. Thus if all $\alpha, \beta, \gamma = 0$, then $\mathsf{i} = \qty{0}$. Thus we conclude $\slie{2}{\R}$ only contains trivial ideals.

	If $\phi: \slie{2}{\R} \to \u{n}$ is a Lie algebra homomorphism, then we know from class that $\ker\phi\subseteq\slie{2}{\R}$ must be an ideal. From above we know we only have two options for this ideal. First $\ker\phi = \slie{2}{\R}$ in which case \emph{everything} gets sent to $0$ and is trivial. In the second case $\ker\phi = \qty{0}$, and by one of the isomorphism theorems we have
	\begin{equation*}
		\slie{2}{\R}/\ker{\phi}\cong\slie{2}{\R}\cong\im{\phi}\subseteq\u{n}.
	\end{equation*}
	But as we've shown above $\u{n}$ does not contain any subalgebra isomorphic to $\slie{2}{\R}$ and hence we conclude the only Lie algebra homomorphism from $\slie{2}{\R}$ into $\u{n}$ is the $0$ map.

	\ref{part:5c}
	Suppose $\Pi:\SL{2}{\R} \to \U{\dim V}$ is a unitary representation, that is $\Pi(A)^\dagger = \Pi(A)$ for all $A \in \SL{2}{\R}$. Using
	\begin{equation*}
		\pi(X) \defeq \dv{t}\eval{\Pi(\e^{tX})}_{t=0}
	\end{equation*}
	we can pass to the associated Lie algebra representation $\pi: \slie{2}{\R}\to\u{\dim V}$, and by complex linearity of $\pi$ we can pass to the associated representation of the complexification of $\slie{2}{\R}_\C$ as $\pi_\C: \slie{2}{\C} \to \u{\dim V}$. However, as shown above the only Lie algebra homomorphism from $\slie{2}{\C}$ into $\u{\dim V}$ is trivial. Traversing back up the chain to $\Pi$ we find that $\Pi$ must also be trivial, and in order to satisfy the fact that it is a representation it must send the identity to the identity. Therefore $\Pi(A) = \id_V$. Note here we used the fact that $\SL{2}{\R}$ is connected to go to and from the Lie group and Lie algebra representation. In particular because we need that $\Pi$ and $\pi$ to have ``the same'' invariant subspaces.

	Also, Proposition 4.8 in Hall states ``If $G$ is connected and $\pi(X)$ is skew self-adjoint for all $X\in\mfr{g}$, then $\Pi$ is unitary'' which is exactly the case we have here. Although in our case $\pi_\C$ is only skew self-adjoint trivially.
\end{solution}

\end{document}