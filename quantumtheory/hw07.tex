\documentclass[boxes,pages]{homework}


\name{Nate Stemen}
\studentid{20906566}
\email{nate.stemen@uwaterloo.ca}
\term{Fall 2020}
\course{Advanced Quantum Theory}
\courseid{AMATH 673}
\hwnum{7}
\duedate{Mon, Nov 23, 2020 11:59 AM}

\problemname{Exercise}

\usepackage{physics}
\usepackage{cleveref}
\usepackage{wasysym}


\newcommand{\iu}{\mathrm{i}\mkern1mu}
\newcommand{\tpose}[1]{#1^\intercal}
\newcommand{\conj}[1]{\overline{#1}}
\newcommand{\hilb}{\mathcal{H}}
\newcommand{\inv}[1]{{{#1}^{-1}}}
\newcommand{\herm}[1]{{{#1}^\dagger}}
\newcommand{\e}{\mathrm{e}}
\newcommand{\R}{\mathbb{R}}
\usepackage{xparse}
\NewDocumentCommand{\green}{ O{t} }{G(x, x', #1)}
\DeclareMathOperator{\spec}{spec}
\DeclareMathOperator{\im}{im}

\makeatletter
\numberwithin{@problem}{section}
\makeatother

\begin{document}

\setcounter{section}{8}
\problemnumber{8}

\begin{problem}
Show that $\psi(x,t)$ of Eq.8.89 does obey the Schr\"odinger equation Eq.8.84 and the initial condition Eq.8.86.
\end{problem}

\begin{solution}
	First, we show it obeys the Schr\"odinger equation.
	\begin{align*}
		\iu\hbar\dv{t}\psi(x, t) & = \iu\hbar\dv{t}\int_\R\green\psi_0(x')\dd{x'}                \\
		                         & = \int_\R\iu\hbar\dv{t}\green \psi_0(x')\dd{x'}               \\
		                         & = \int_\R \hat{H}_S(\hat{x}, \hat{p})\green \psi_0(x')\dd{x'} \\
		                         & = \hat{H}_S(\hat{x}, \hat{p})\int_\R \green \psi_0(x')\dd{x'} \\
		                         & = \hat{H}_S(\hat{x}, \hat{p})\psi(x, t)
	\end{align*}
	That works, sweet. Now let's check the initial conditions.
	\begin{align*}
		\psi(x, t_0) & = \int_\R\green[t_0]\psi_0(x')\dd{x'}    \\
		             & = \int_\R\delta(x - x')\psi_0(x')\dd{x'} \\
		             & = \psi_0(x)
	\end{align*}
	It indeed works!
\end{solution}

\setcounter{section}{9}
\problemnumber{1}

\begin{problem}
Assume that $\spec(\hat{Q}) = \{0,1\}$ for a normal operator $\hat{Q}$. Does this mean that $\hat{Q}$ is a projector, and why?
\end{problem}

\begin{solution}
	Take $\hat{Q}:V\to V$ and $v$ be a vector in $V$. We can then write $V = \ker{\hat{Q}}\oplus \im{\hat{Q}}$ where $\im{\hat{Q}}$ denotes the image of $\hat{Q}$.
	This means any vector $v\in V$ can be decomposed as $v = x + y$ with $x \in \ker{\hat{Q}}$ and $y\in\im{\hat{Q}}$.
	Now we know $\spec(\hat{Q}) = \{0,1\}$ so eigenvectors of $\hat{Q}$ satisfy either of the two equations:
	\begin{align*}
		\hat{Q}v & = 0 & \hat{Q}v & = v.
	\end{align*}
	Thus taking a general $v\in V$ we have $\hat{Q}v = \hat{Q}x + \hat{Q}y = 0 + y$.
	And applying $\hat{Q}$ twice we have $\hat{Q}^2v = \hat{Q}y = y$.
	This shows that $\hat{Q}^2 = \hat{Q}$.
	Now I know I need to show $\hat{Q} = \herm{\hat{Q}}$, and I suppose I should use the fact that $\hat{Q}$ is normal, but I can't figure out how \frownie.
	To be honest, I don't even feel like my above answer is correct, but it's the best I got.
\end{solution}

\begin{problem}
Verify that this operator $\hat{Q}$ is a projector.
\end{problem}

\begin{solution}
	We'll first show $\herm{\hat{Q}} = \hat{Q}$.
	\begin{equation*}
		\herm{\hat{Q}} = \herm{\qty[\sum_{a = 1}^N\ketbra{f_{j_a}}]} = \sum_{a = 1}^N\herm{\qty(\ketbra{f_{j_a}})} = \sum_{a = 1}^N\ketbra{f_{j_a}} = \hat{Q}
	\end{equation*}
	Now we can show $\hat{Q}^2 = \hat{Q}$.
	\begin{align*}
		\hat{Q}^2 & = \qty[\sum_{a = 1}^N\ketbra{f_{j_a}}]\qty[\sum_{b = 1}^N\ketbra{f_{j_b}}] \\
		          & = \sum_{a, b = 1}^N\ket{f_{j_a}}\braket{f_{j_a}}{f_{j_b}}\bra{f_{j_b}}     \\
		          & = \sum_{a, b = 1}^N\ket{f_{j_a}}\delta_{ab}\bra{f_{j_b}}                   \\
		          & = \sum_{a = 1}^N\ketbra{f_{j_a}} = \hat{Q}
	\end{align*}
\end{solution}

\begin{problem}
Let us check that the prescription Eq.9.15 describes the collapse correctly. To see this we need to check if it obeys the condition that it describes the outcome of an ideal measurement, i.e., of a measurement that, when immediately repeated, will yield the same result. Show that when the collapse described by Eq.9.15 is applied twice, it yields the same state as after the first application
\end{problem}

\begin{solution}
	Let's first define the following 3 states.
	\begin{itemize}
		\item $\ket{\psi}$ be the state before measurement
		\item $\ket{\psi_1}$ be the state just after the first measurement
		\item $\ket{\psi_2}$ be the state just after the second measurement
	\end{itemize}
	With these we are trying to show that $\ket{\psi_2} = \ket{\psi_1}$.
	\begin{align*}
		\ket{\psi_2} & = \frac{1}{\norm{\hat{Q}\ket{\psi_1}}}\hat{Q}\ket{\psi_1}                                              \\
		             & = \frac{1}{\norm{\hat{Q}\ket{\psi_1}}}\hat{Q}\qty(\frac{1}{\norm{\hat{Q}\ket{\psi}}}\hat{Q}\ket{\psi}) \\
		             & = \frac{1}{\norm{\hat{Q}\ket{\psi}}}\hat{Q}^2\ket{\psi}                                                \\
		             & = \frac{1}{\norm{\hat{Q}\ket{\psi}}}\hat{Q}\ket{\psi} \eqqcolon \ket{\psi_1}
	\end{align*}
	Where we've used the fact that $\norm{\hat{Q}\ket{\psi_1}} = 1$ which can be shown as follows.
	\begin{align*}
		\norm{\hat{Q}\ket{\psi_1}} & = \norm{\frac{1}{\norm{\hat{Q}\ket{\psi}}}\hat{Q}^2\ket{\psi}}  \\
		                           & = \frac{\norm{\hat{Q}\ket{\psi}}}{\norm{\hat{Q}\ket{\psi}}} = 1
	\end{align*}
	Where we've use the fact that $\hat{Q}^2 = \hat{Q}$ and that you can pull (the absolute value of) a scalar out of a norm.
\end{solution}

\begin{problem}
\begin{parts}
	\part{Consider a free electron in one dimension. Write down the 1-bit observable $\hat{Q}$ which yields the measurement outcome 1 if the electron is measured in the interval $[x_a, x_b]$ and 0 if it is found outside this interval.}\label{part:4a}
	\part{Consider a one-dimensional harmonic oscillator. Write down the 1-bit observable $\hat{Q}$ which yields the measurement outcome 1 if the energy of the oscillator is up to $7\hbar\omega/2$ and is 0 if the energy is above $7\hbar\omega/2$.}\label{part:4b}
\end{parts}
\end{problem}

\begin{solution}
	\ref{part:4a}
	\begin{equation*}
		\hat{Q} = \int_{x_a}^{x_b}\dd{x}\ketbra{x}
	\end{equation*}

	\ref{part:4b}
	\begin{equation*}
		\hat{Q} = \sum_{n = 0}^3\ketbra{E_n}
	\end{equation*}
\end{solution}

\setcounter{section}{10}
\problemnumber{1}

\begin{problem}
\begin{parts}
	\part{Prove that $\tr(\hat{\rho}) = 1$}\label{part:1a}
	\part{Prove that $\tr(\hat{\rho}^2) \leq 1$ and that $\tr(\hat{\rho}^2) = 1$ if and only if $\hat{\rho}$ is the density operator of a pure state.}\label{part:1b}
\end{parts}
\end{problem}

\begin{solution}
	\ref{part:1a}
	\begin{align*}
		\tr(\hat{\rho}) & = \sum_n\expval{\hat{\rho}}{n}                \\
		                & = \sum_n\expval{\qty[\sum_ip_i\ketbra{i}]}{n} \\
		                & = \sum_{n, i}p_i\delta_{ni}                   \\
		                & = \sum_i p_i = 1
	\end{align*}
	Where we've used the fact that $\braket{i}{n} = \delta_{in}$ and the sum of the probabilities equal 1.

	\ref{part:1b}
	In a moment we will show that $\tr \hat{\rho}^2 \leq 1$, but first let us show the other two properties that are asked of us.
	We will use the proof that $\tr \hat{\rho}^2 = 1$ for pure states in our proof that $\tr \hat{\rho}^2 \leq 1$.

	First we show that if $\hat{\rho} = \ketbra{\psi}$, i.e. it is a pure, state, then $\tr(\hat{\rho}^2) = 1$. First note that $\hat{\rho}^2 = \ket{\psi}\braket{\psi}{\psi}\bra{\psi} = \ketbra{\psi} = \hat{\rho}$. Thus we immediately have $\tr(\hat{\rho}^2) = \tr(\hat{\rho}) = 1$.
	Now we prove the other direction: suppose $\tr(\hat{\rho}^2) = 1$. Let's first calculate $\hat{\rho}^2$.
	\begin{align*}
		\hat{\rho}^2 & = \qty(\sum_{i = 1}^n p_i\ketbra{i}{i})^2 = \sum_{i = 1}^n p_i\ketbra{i}{i}\sum_{j = 1}^n p_j\ketbra{j}{j}   \\
		             & = \sum_{i,j = 1}^n p_ip_j\ket{i}\braket{i}{j}\bra{j} \tag{using a basis where $\braket{i}{j} = \delta_{ij}$} \\
		             & = \sum_{i = 1}^n p_i^2\ketbra{i}{i}
	\end{align*}
	Now taking the trace of $\hat{\rho}^2$ we get a condition on the $p_i$'s.
	\begin{align*}
		\tr \hat{\rho}^2 & = \sum_{k = 1}^n\bra{k}\rho^2\ket{k}                     \\
		                 & = \sum_{k,i = 1}^n\bra{k}\qty(p_i^2\ketbra{i}{i})\ket{k} \\
		                 & = \sum_{i = 1}^n p_i^2 = 1
	\end{align*}
	So we now know $\sum p_i = 1 = \sum p_i^2$. By basic properties of real numbers, we know $p_i^2 \leq p_i$ when $p_i\in[0, 1]$ and equality only holding when $p_i \in\{0, 1\}$. Using this fact we can write
	\begin{equation*}
		\sum_{i = 1}^n p_i^2 \leq \sum_{i = 1}^n p_i
	\end{equation*}
	where again equality only holds if $p_i\in\{0, 1\}$ for all $i$. The only way this can be true is if one of the $p_i$'s is 1 and all of the rest are 0. In that case our summation collapses to one term, and we are left with $\hat{\rho} = \ketbra{i}$ as desired.

	Returning to $\tr \hat{\rho}^2 \leq 1$, we know that $\tr \hat{\rho}^2  = \sum_i p_i^2$ and $p_i^2 \leq p_i$ for all $p_i\in\R_{\geq 0}$. Therefore $\sum_i p_i^2 \leq 1$ and hence $\tr \hat{\rho}^2 \leq 1$.
\end{solution}

\begin{problem}
Use the Schr\"odinger equation and the definition of the density operator to prove the von Neumann equation Eq.10.20.
\end{problem}

\begin{solution}
	\begin{align*}
		\iu\hbar\dv{t}\hat{\rho}(t) & = \iu\hbar\dv{t}\sum_np_n\ketbra{\psi_n(t)}                                                          \\
		                            & = \iu\hbar\sum_n p_n\ketbra{\psi_n'(t)}{\psi_n(t)} + p_n\ketbra{\psi_n(t)}{\psi_n'(t)}               \\
		                            & = \sum_n p_n\hat{H}_S(t)\ketbra{\psi_n(t)}{\psi_n(t)} + p_n\ketbra{\psi_n(t)}{\psi_n(t)}\hat{H}_S(t) \\
		                            & = \hat{H}_S(t)\sum_n p_n\ketbra{\psi_n(t)}{\psi_n(t)} + p_n\ketbra{\psi_n(t)}{\psi_n(t)}\hat{H}_S(t) \\
		                            & = \comm{\hat{H}_S(t)}{\hat{\rho}(t)}
	\end{align*}
\end{solution}

\begin{problem}
Prove Eq.10.22 using Eq.10.21.
\end{problem}

\begin{solution}
	First, let's take $\hat{\rho}$ to be in a basis that diagonalizes itself.
	That is
	\begin{equation*}
		\hat{\rho} = \mqty[\dmat{p_0, p_1,\ddots}] \quad \text{and} \quad \log\hat{\rho} = \mqty[\dmat{\log p_0, \log p_1, \ddots}]
	\end{equation*}
	where all the off diagonal terms are 0.
	In this case it's easy to see their product is also a diagonal matrix:
	\begin{equation*}
		\hat{\rho}\log\hat{\rho} = \mqty[\dmat{p_0\log p_0, p_1\log p_1, \ddots}]
	\end{equation*}
	Taking the trace we now have
	\begin{equation*}
		\tr(\hat{\rho}\log\hat{\rho}) = \sum_n p_n\log p_n
	\end{equation*}
	and adding a minus sign to both sides shows $S \coloneqq -\sum_n p_n\log p_n = -\tr(\hat{\rho}\log\hat{\rho})$.
\end{solution}

\begin{problem}
Use Eq.10.24 to  show that the Shannon entropy definition Eq.10.23 obeys the additivity condition Eq.10.25. Hint: the ignorance about the combined system, $S[\{\tilde{\rho}_{n,m}\}]$, is calculated with the same formula, namely:
\[
	S[\{\tilde{\rho}_{n,m}\}] = -\sum_{n, m}\tilde{\rho}_{n, m}\ln(\tilde{\rho}_{n,m})
\]
Also, you may use that $1 = \sum_n \rho_n = \sum_m\rho'_m = \sum_{n, m}\tilde{\rho}_{n,m}$.
\end{problem}

\begin{solution}
	We start by expanding out $S[\{\tilde{\rho}_{n,m}\}]$ which we will denote as $S[\tilde{\rho}]$.
	\begin{align*}
		S[\tilde{\rho}] & = -\sum_{n, m}\tilde{\rho}_{n, m}\log\tilde{\rho}_{n, m}                                                                 \\
		                & = -\sum_{n, m}\rho_n\rho_m'\log(\rho_n\rho_m')                                                                           \\
		                & = -\sum_{n, m}\rho_n\rho_m'\qty[\log\rho_n + \log\rho_m']                                                                \\
		                & = - \sum_m\rho_m'\qty[\sum_n\rho_n\log\rho_n] - \sum_n\rho_n\qty[\sum_m\rho_m'\log\rho_m']                               \\
		                & = - \qty[\sum_n\rho_n\log\rho_n]\underbrace{\sum_m\rho_m'}_1 - \qty[\sum_m\rho_m'\log\rho_m']\underbrace{\sum_n\rho_n}_1 \\
		                & = - \sum_n\rho_n\log\rho_n - \sum_m\rho_m'\log\rho_m'                                                                    \\
		                & = S[\rho] + S[\rho']
	\end{align*}
\end{solution}

\end{document}
