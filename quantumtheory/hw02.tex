\documentclass[boxes,pages]{homework}


\name{Nate Stemen}
\studentid{20906566}
\email{nate.stemen@uwaterloo.ca}
\term{Fall 2020}
\course{Advanced Quantum Theory}
\courseid{AMATH 673}
\hwnum{2}
\duedate{Fri, Sep 25, 2020 11:59 AM}

%\hwname{(Name)}
\problemname{Exercise}
%\solutionname{(Name)}

\usepackage{physics}
\usepackage[makeroom,Smaller]{cancel}
\usepackage{marginnote}
\renewcommand*{\marginfont}{\footnotesize}

\newcommand{\mmtm}[2]{p_#1^{(#2)}}
\newcommand{\pstn}[2]{x_#1^{(#2)}}
\newcommand{\iu}{\mathrm{i}}
\newcommand{\tpose}[1]{#1^\intercal}

\DeclareMathOperator{\id}{id}

\makeatletter
\numberwithin{@problem}{section}
\makeatother

\begin{document}

\setcounter{section}{3}

\begin{problem}
Find and list the defining property of an involution.
\end{problem}

\begin{solution}
	An \emph{involution} is a function which is it's own inverse. Symbolically this can be written as $f(f(x)) = x$. This is not to be confused\marginnote{I confused the two, maybe that's just me though!} with \emph{idempotence} which means $f(f(x)) = f(x)$.
\end{solution}

\begin{problem}
For classical mechanics, formula Eq.2.37 provided a convenient representation of the Poisson bracket. However, Eq.2.37 is not a valid representation of the Poisson bracket in the case of quantum mechanics. In quantum mechanics, we have a (not so convenient) representation of the Poisson bracket through Eq.3.11:
\begin{equation*}
	\pb{\hat{u}}{\hat{v}} = \frac{1}{\iu\hbar}\qty(\hat{u}\hat{v} - \hat{v}\hat{u})
\end{equation*}
Use this representation to evaluate the Poisson bracket $\pb{\hat{x}^2}{\hat{p}}$.
\end{problem}

\begin{solution}
	The axioms of the Poisson bracket allow us to rewrite $\pb{\hat{x}^2}{\hat{p}}$ as $\hat{x}\pb{\hat{x}}{\hat{p}} + \pb{\hat{x}}{\hat{p}}\hat{x}$. We can now use the canonical commutation relations to evaluate this.
	\begin{align*}
		\pb{\hat{x}^2}{\hat{p}} & = \hat{x}\pb{\hat{x}}{\hat{p}} + \pb{\hat{x}}{\hat{p}}\hat{x}                                                                                         \\
		                        & = \frac{1}{\iu\hbar}\hat{x}\underbrace{\comm{\hat{x}}{\hat{p}}}_{\iu\hbar} + \frac{1}{\iu\hbar}\underbrace{\comm{\hat{x}}{\hat{p}}}_{\iu\hbar}\hat{x} \\
		                        & = 2\hat{x}
	\end{align*}
\end{solution}

\begin{problem}
Reconsider the system with the Hamiltonian Eq.2.4, which consists of two particles that are attracted to another through a harmonic force (a force which is proportional to their distance). In practice, for example the force that binds diatomic molecules and the force that keeps nucleons (i.e., neutrons and protons) inside a nucleus are essentially harmonic. In those cases the effect of $\hbar$ cannot be neglected. One obtains the correct quantum theoretic Hamiltonian from the classical Hamiltonian of Eq.2.4 by simply placing hats on the $x$ and $p$’s. Find explicitly all the equations of motion which the $\hat{x}_i^{(r)}$ and $\hat{p}_j^{(r)}$ (where $r\in \{1,2\}$) of this system must obey.
\end{problem}

\begin{solution}
	First remember the Hamiltonian is given by the following equation where we've used $\hat{p}_1$ to denote the momentum of particle 1. This is much easier on the eyes than $\hat{p}^{(1)}$ and the subscripts are not needed for dimensions.
	\begin{equation*}
		\widehat{H} = \frac{\hat{p}_1^2}{2m_1} + \frac{\hat{p}_2^2}{2m_2} + \frac{k}{2}\qty(\hat{x}_1 - \hat{x}_2)^2
	\end{equation*}
	Now to get the equations of motions we need to take the commutator of our coordinate with $\widehat{H}$.
	\begin{align*}
		\dv{t}\hat{x}_i & = \pb{\hat{x}_i}{\widehat{H}} + \partial_t \hat{x}_i \\
		                & = \pb{\hat{x}_i}{\frac{\hat{p}_i^2}{2m_i}}           \\
		                & = \frac{2\hat{p}_i}{2m_i} = \frac{\hat{p}_i}{m_i}
	\end{align*}
	Now for the slightly more complicated coordinate of momentum.
	\begin{align*}
		\dv{t}\hat{p}_i & = \pb{\hat{p}_i}{\widehat{H}} + \partial_t \hat{p}_i                                 \\
		                & = \pb{\hat{p}_i}{\frac{k}{2}\qty(\hat{x}_1 - \hat{x}_2)^2}                           \\
		                & = \frac{k}{2}\pb{\hat{p}_i}{\hat{x}_1^2 + \hat{x}_2^2 - 2\hat{x}_1\hat{x}_2}         \\
		                & = \frac{k}{2}\qty[\pb{\hat{p}_i}{\hat{x}_i^2} - 2\pb{\hat{p}_i}{\hat{x}_i\hat{x}_j}] \\
		                & = \frac{k}{2}\qty[-2\hat{x}_i + 2\hat{x}_j]                                          \\
		                & = -k\qty(\hat{x}_i - \hat{x}_j)
	\end{align*}
	Where here we have let $j$ be the ``other'' particle. That is if $i = 1$ then $j = 2$ and if $i = 2$ then $j = 1$.
\end{solution}

\begin{problem}
To obtain the quantum Hamiltonian from the classical Hamiltonian and vice versa by placing or removing hats on the $x$ and $p$’s is generally not as straight forward as in the previous exercise! Namely, there can occur so-called ``ordering ambiguities'': Consider the two Hamiltonians $\hat{H}_1 = \hat{p}^2/2m + a(\hat{x}^2\hat{p}\hat{x} - \hat{x}\hat{p}\hat{x}^2)$
and
$\hat{H}_2 = \hat{p}^2/2m + b(\hat{p}\hat{x}\hat{p}^2 - \hat{p}^2\hat{x}\hat{p})$ where $a$ and $b$ are nonzero constants with appropriate units. Check whether or not these two Hamiltonians are the same in classical mechanics. Also check whether or not they can be the same in quantum mechanics for some nonzero choices of $a$ and $b$.
\end{problem}

\begin{solution}
	We start by removing the hats to obtain the classical Hamiltonians.
	\begin{align*}
		H_1 & = \frac{p^2}{2m} + a\qty(x^2px - xpx^2) & H_2 & = \frac{p^2}{2m} + b\qty(pxp^2 - p^2xp) \\
		    & = \frac{p^2}{2m} + a\qty(x^3p - x^3p)   &     & = \frac{p^2}{2m} + b\qty(xp^3 - xp^3)   \\
		    & = \frac{p^2}{2m}                        &     & = \frac{p^2}{2m}
	\end{align*}
	So because the position and momentum operators commute in classical mechanics, they represent the same Hamiltonian.

	To check whether they can be the same Hamiltonian in quantum mechanics we must simplify the second terms.
	\begin{align*}
		\hat{x}^2\hat{p}\hat{x} - \hat{x}\hat{p}\hat{x}^2 & = \hat{x}(\hat{x}\hat{p})\hat{x} -\hat{x}(\hat{p}\hat{x})\hat{x}                      \\
		                                                  & = \hat{x}(\iu\hbar + \hat{p}\hat{x})\hat{x} -\hat{x}(\hat{x}\hat{p} -\iu\hbar)\hat{x} \\
		                                                  & = -\qty(\hat{x}^2\hat{p}\hat{x} - \hat{x}\hat{p}\hat{x}^2) + 2\iu\hbar \hat{x}^2      \\
		                                                  & = \iu\hbar \hat{x}^2 = \comm{\hat{x}}{\hat{p}}\hat{x}^2
	\end{align*}
	This same procedure can be done for the second term in $\widehat{H}_2$ making sure to expand the inner terms to obtain
	\begin{equation*}
		\hat{p}\hat{x}\hat{p}^2 - \hat{p}^2\hat{x}\hat{p} = \iu\hbar \hat{p}^2
	\end{equation*}
	Now if we set the two Hamiltonians equal to each other we obtain
	\begin{equation*}
		a\hat{x}^2 = b\hat{p}^2
	\end{equation*}
	and I do not think it is possible for a constant to transform the position operator to the momentum operator or vice versa, so I conclude there are no constants $a, b$ that make these two Hamiltonians equal in the quantum mechanical sense.

\end{solution}

\begin{problem}
Find a Hamiltonian which contains at least one thousand powers of $\hat{x}$ and which also agrees with the Hamiltonian $\hat{H}_1$ of the previous exercise in classical mechanics. Make sure that your Hamiltonian is formally hermitean, i.e., that it obeys $\hat{H}^\dagger = \hat{H}$. Help: To ensure hermiticity, you can symmetrize. For example, $\hat{x}\hat{p}^2$ is not hermitean but$\qty(\hat{x}\hat{p}^2 + \hat{p}^2\hat{x})/2$ is hermitean.
\end{problem}

\begin{solution}
	Take $\hat{H}_w = \frac{\hat{p}^2}{2m} + a\qty(\hat{x}\hat{p}\hat{x}^{1000} - \hat{x}^{1000}\hat{p}\hat{x})$ with $a$ a purely complex number (i.e. no real part).
\end{solution}

\begin{problem}
Find and list the precise axioms that a set has to obey to be called a vector space.
\end{problem}

\begin{solution}
	A \emph{vector space} $V$ consists of a field $F$, an operation $+:V\times V\to V$ which forms an abelian group, and an operation $\cdot:F\times V\to V$ which must satisfy the following conditions.
	\begin{align*}
		\alpha\cdot(\beta\cdot\vb{v}) & = (\alpha\beta)\cdot\vb{v}              \\
		1\cdot\vb{v}                  & = \vb{v}                                \\
		\alpha\cdot(\vb{u} + \vb{v})  & = \alpha\cdot\vb{u} + \alpha\cdot\vb{v} \\
		(\alpha +_F \beta)\cdot\vb{v} & = \alpha\cdot\vb{v} + \beta\cdot\vb{v}
	\end{align*}
	Where $\alpha,\beta\in F$ and $\vb{v}, \vb{u}\in V$ and $1$ is the identity in the field, and $+_F$ is the addition in the field, not the vector addition defined on $V\times V$.
\end{solution}

\begin{problem}
Check whether or not the multiplication operator, $M$, which maps $M: g(\lambda)\to\lambda g(\lambda)$ is a linear operator.
\end{problem}

\begin{solution}
	Here we just have to verify the function respects addition and scalar multiplication.
	\begin{align*}
		M(g + h) & = \lambda\qty(g(\lambda) + h(\lambda))    & M(\alpha g) & = \lambda(\alpha g(\lambda)) \\
		         & = \lambda g(\lambda) + \lambda h(\lambda) &             & = \alpha\lambda g(\lambda)   \\
		         & = M(g) + M(h)                             &             & = \alpha M(g)                \\
	\end{align*}
	Hence we conclude the multiplication map is a linear operator.
\end{solution}

\begin{problem}
Show that the two operators $D$ and $M$ on $V$ do not commute, namely by calculating $(DM-MD)g(\lambda)$.
\end{problem}

\begin{solution}
	Lets first calculate the commutator of $D$ and $M$ by calculating the terms separately.
	\begin{align*}
		D(M(g(\lambda))) & = D(\lambda g(\lambda))            & M(D(g(\lambda))) & = M(g'(\lambda))      \\
		                 & = g(\lambda) + \lambda g'(\lambda) &                  & = \lambda g'(\lambda)
	\end{align*}
	We can now see the commutator follows.
	\begin{align*}
		\comm{D}{M}g(\lambda) & = (DM-MD)g(\lambda)                                      \\
		                      & = D(M(g(\lambda))) - M(D(g(\lambda)))                    \\
		                      & = g(\lambda) + \lambda g'(\lambda) - \lambda g'(\lambda) \\
		                      & = g(\lambda)
	\end{align*}
	Written slightly differently we can say $\comm{D}{M} = \id_V$ where $\id_V$ is the identity operator on $V$ that takes $g(\lambda)\stackrel{\id_V}{\longmapsto}g(\lambda)$.
\end{solution}

\begin{problem}
Check whether or not the operator $Q$ which acts on functions in $V$ as $Q: g(\lambda)\to \lambda^5g(\lambda)$ is a linear operator.
\end{problem}

\begin{solution}
	Here we again verify the function respects addition and scalar multiplication.
	\begin{align*}
		Q(g + h) & = \lambda^5\qty(g(\lambda) + h(\lambda))      & Q(\alpha g) & = \lambda^5(\alpha g(\lambda)) \\
		         & = \lambda^5 g(\lambda) + \lambda^5 h(\lambda) &             & = \alpha\lambda^5 g(\lambda)   \\
		         & = Q(g) + Q(h)                                 &             & = \alpha Q(g)                  \\
	\end{align*}
	Hence we can conclude $Q$ is indeed a linear operator.
\end{solution}

\begin{problem}
Verify Eqs.3.38.
\end{problem}

\begin{solution}
	\begin{align*}
		\tr(A + B) & = \sum_i(a_{ii} + b_{ii})       \\
		           & = \sum_i a_{ii} + \sum_i b_{ii} \\
		           & = \tr(A) + \tr(B)
	\end{align*}
	In the following derivation we use $(AB)_{ii}$ to denote the element at the $i$th row and $i$th column in the matrix $AB$.
	\begin{align*}
		\tr(AB) & = \sum_i(AB)_{ii}           \\
		        & =\sum_i\sum_k a_{ik}b_{ki}  \\
		        & = \sum_k\sum_i b_{ki}a_{ik} \\
		        & = \sum_k(BA)_{kk}           \\
		        & = \tr(BA)
	\end{align*}
\end{solution}

\begin{problem}
Verify Eq.3.41.
\end{problem}

\begin{solution}
	Lets first calculate the two components of the commutator.
	\begin{align*}
		aa^\dagger & = \mqty(\dmat{1,2,3,4,\ddots}) & a^\dagger a & = \mqty(\dmat{0,1,2,3,\ddots})
	\end{align*}
	From here we can see the commutator $\comm{a}{a^\dagger} = \mathbb{1}$.
\end{solution}

\begin{problem}
Verify that the two matrices defined in Eqs.3.42, 3.43 with the help of Eqs.3.39, 3.40, are formally hermitean. I am using the term ``formally'' here to indicate that, for the purposes of this exercise, you need not worry about potential subtleties that may arise because these matrices are infinite dimensional.
\end{problem}

\begin{solution}
	Here we use the fact that the Hermitian conjugate is an involution and hence ${f^\dagger}^\dagger = f$ along with the fact that $\qty(a + b)^\dagger = a^\dagger + b^\dagger$ and $c^\dagger = \overline{c}$ if $c\in\mathbb{C}$.
	\begin{align*}
		\qty(\hat{x}(t_0))^\dagger & = \qty(L\qty(a^\dagger + a))^\dagger                & \qty(\hat{p}(t_0))^\dagger & = \qty(\frac{\iu\hbar}{2L}\qty(a^\dagger - a))^\dagger      \\
		                           & = \overline{L}\qty({a^\dagger}^\dagger + a^\dagger) &                            & = \frac{-\iu\hbar}{2L}\qty({a^\dagger}^\dagger - a^\dagger) \\
		                           & = L\qty(a + a^\dagger) = \hat{x}(t_0)               &                            & = \frac{\iu\hbar}{2L}\qty(-a + a^\dagger) = \hat{p}(t_0)    \\
	\end{align*}
\end{solution}

\begin{problem}
Show that the hermitean conjugation of matrices reverses the order, i.e., that if $A$ and$B$ are linear maps, then $(AB)^\dagger = B^\dagger A^\dagger$. To this end, write out the matrices with indices and use that hermitean conjugating a matrix means transposing and complex conjugating it.
\end{problem}

\begin{solution}
	Here we use $b'_{ij}$ to denote the elements of $B^\dagger$ and similarly for $A$ where $b'_{ij} = \overline{b_{ji}}$.
	\begin{align*}
		\qty(B^\dagger A^\dagger)_{ij} & = \sum_k b_{ik}'a_{kj}'                                    \\
		                               & = \sum_k \overline{b_{ki}}\overline{a_{jk}}                \\
		                               & = \sum_k \overline{a_{jk}} \overline{b_{ki}}               \\
		                               & = \tpose{\qty(\sum_k \overline{a_{ik}} \overline{b_{kj}})} \\
		                               & = (AB)^\dagger_{ij}
	\end{align*}
	Because we've shown this to be true for all $i, j$ it must be the case that $B^\dagger A^\dagger = (AB)^\dagger$.
\end{solution}

\begin{problem}
Show that the matrices $\hat{x}(t)$ and $\hat{p}(t)$ obey at all times $t > t_0$ all the quantum mechanical conditions, i.e., the equations of motion, the hermiticity condition, and the commutation relation.
\end{problem}

\begin{solution}
	What a lovely problem... I understood why it took being secluded on an island with no one to talk to and nothing to do in order to develop matrix mechanics. Actually, I'm still not sure I understand why because that doesn't sound so bad as to drive me to multiplying matrices for fun.

	Anyway, lets first verify the equations of motion. We'll start with the equation for the position: $\hat{x}(t) = \hat{x}(t_0) + \frac{t - t_0}{m}\hat{p}(t_0)$.
	\begin{align*}
		\hat{x}(t_0) + \frac{t - t_0}{m}\hat{p}(t_0) & = \mqty[0 & \sqrt{1}L & & \\ \sqrt{1}L & 0 & \sqrt{2}L & \\ & \sqrt{2}L & 0 & \\ & & & \ddots] + \mqty[0 & -\sqrt{1}\frac{\iu\hbar(t - t_0)}{2Lm} & \\ \sqrt{1}\frac{\iu\hbar(t - t_0)}{2Lm} & 0 & \\ & & \ddots] \\
		& = \hat{x}(t)
	\end{align*}
	That works out. Second we have to verify $\hat{p}(t) = \hat{p}(t_0)$ which is quickly verified because $\hat{p}$ has no time dependence.

	Now to move on to the hermiticity condition. For $\hat{x}(t)$ the above diagonal non-zero terms can all be represented as $\sqrt{n}\qty(L - \frac{\iu\hbar(t - t_0)}{2Lm})$ with it's ``transpose partner'' being $\sqrt{n}\qty(L + \frac{\iu\hbar(t - t_0)}{2Lm})$. Upon taking the Hermitian conjugate the necessary signs will flip because of the imagary unit in the second term and hence $\hat{x}(t)$ is Hermitian. It's slightly easier to see this is also the case for $\hat{p}(t)$ because there is only one term. Taking the complex conjugate of a general term $\overline{-\sqrt{n}\frac{\iu\hbar}{2L}} =\sqrt{n}\frac{\iu\hbar}{2L}$ we see this is exactly the transpose partner.

	\emph{Finally}, we have to verify the commutation relation. Best for last I guess.
	\begin{align*}
		\hat{x}(t)\hat{p}(t) & = \mqty[\frac{\iu\hbar}{2L}\qty(L - \frac{\iu\hbar(t - t_0)}{2Lm}) & 0 & \\ 0 & \frac{-\iu\hbar}{2L}\qty(L + \frac{\iu\hbar(t - t_0)}{2Lm}) + \frac{\iu\hbar}{L}\qty(L - \frac{\iu\hbar(t - t_0)}{2Lm}) & \\ & & \ddots] \\
		\hat{p}(t)\hat{x}(t) & = \mqty[-\frac{\iu\hbar}{2L}\qty(L + \frac{\iu\hbar(t - t_0)}{2Lm}) & 0 & \\ 0 & \frac{\iu\hbar}{2L}\qty(L - \frac{\iu\hbar(t - t_0)}{2Lm}) - \frac{\iu\hbar}{L}\qty(L + \frac{\iu\hbar(t - t_0)}{2Lm}) & \\ & & \ddots]
	\end{align*}
	From here we have some serious algebra to do. For the first two elements I've verified $\comm{\hat{x}(t)}{\hat{p}(t)} = \iu\hbar$ and I think that's enough the conclude the result holds. 2 out of infinity is enough, right?
\end{solution}

\end{document}