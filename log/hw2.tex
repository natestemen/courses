\documentclass[boxes,pages,color=CornflowerBlue]{homework}

\hypersetup{
    colorlinks=true,
    linkcolor=Bittersweet
}

\usepackage{booktabs}
\usepackage{nicematrix}
% \usepackage[table]{xcolor}

\name{Nate Stemen}
\studentid{20906566}
\email{nate@stemen.email}
\term{Winter 2021}
\course{Logic and Computability}
\courseid{PMATH 632}
\hwnum{2}
\duedate{Jan 28, 2022 11:59PM}
\hwname{Assignment}

\newcommand{\true}{\ensuremath{\mathsf{T}}}
\newcommand{\false}{\ensuremath{\mathsf{F}}}

\begin{document}

\begin{problem}[Truth functions]
\begin{parts}
    \part Prove DeMorgan's law:
    \begin{equation*}
        \dot{\wedge}(x, y) = \dot{\neg}(\dot{\vee}(\dot{\neg}(x), \dot{\neg}(y)))
    \end{equation*}
    for all $x,y \in \qty{\true, \false}$.
    \label{part:1a}
    \part Show that one can similarly express $\dot{\rightarrow}(x,y)$ and $\dot{\leftrightarrow}(x,y)$ in terms of the functions $\dot{\neg}$ and $\dot{\vee}$.
    \label{part:1b}
    \part Express contraposition as a statement about $\dot{\rightarrow}$ and $\dot{\neg}$ and $\dot{\leftrightarrow}$.
    \label{part:1c}
\end{parts}
\end{problem}

\begin{solution}
    \ref{part:1a}
    DeMorgan's law can be seen by building up the right hand side of the equality from it's components.
    \begin{center}
        \begin{tabular}{ccccccc}
            $x$    & $y$    & $\dot{\neg}(x)$ & $\dot{\neg}(y)$ & $\dot{\vee}(\dot{\neg}(x), \dot{\neg}(y))$ & $\dot{\neg}(\dot{\vee}(\dot{\neg}(x), \dot{\neg}(y)))$ & $\dot{\wedge}(x, y)$ \\ \toprule
            \true  & \true  & \false          & \false          & \false                                     & \true                                                  & \true                \\
            \true  & \false & \false          & \true           & \true                                      & \false                                                 & \false               \\
            \false & \true  & \true           & \false          & \true                                      & \false                                                 & \false               \\
            \false & \false & \true           & \true           & \true                                      & \false                                                 & \false
        \end{tabular}
    \end{center}

    \ref{part:1b}
    We give the following two characterizations by truth tables for implication and the biconditional.
    \begin{center}
        \begin{tabular}{ccccc}
            $x$    & $y$    & $\dot{\neg}(x)$ & $\dot{\vee}(\dot{\neg}(x), y)$ & $\dot{\rightarrow}(x, y)$ \\ \toprule
            \true  & \true  & \false          & \true                          & \true                     \\
            \true  & \false & \false          & \false                         & \false                    \\
            \false & \true  & \true           & \true                          & \true                     \\
            \false & \false & \true           & \true                          & \true
        \end{tabular}
    \end{center}
    \begin{center}
        \begin{tabular}{cccccc}
            $x$    & $y$    & $\dot{\wedge}(x, y)$ & $\dot{\wedge}(\dot{\neg}(x), \dot{\neg}(y))$ & $\dot{\vee}(\dot{\wedge}(x, y), \dot{\wedge}(\dot{\neg}(x), \dot{\neg}(y)))$ & $\dot{\leftrightarrow}(x, y)$ \\ \toprule
            \true  & \true  & \true                & \false                                       & \true                                                                        & \true                         \\
            \true  & \false & \false               & \false                                       & \false                                                                       & \false                        \\
            \false & \true  & \false               & \false                                       & \false                                                                       & \false                        \\
            \false & \false & \false               & \true                                        & \true                                                                        & \true
        \end{tabular}
    \end{center}
    Now since we must use only negation and disjunction we can use DeMorgan's law to write the following.
    \begin{align*}
        \dot{\leftrightarrow}(x, y) & = \dot{\vee}(\dot{\wedge}(x, y), \dot{\wedge}(\dot{\neg}(x), \dot{\neg}(y)))                                                                     \\
                                    & = \dot{\vee}(\dot{\neg}(\dot{\vee}(\dot{\neg}(x), \dot{\neg}(y))), \dot{\neg}(\dot{\vee}(\dot{\neg}(\dot{\neg}(x)), \dot{\neg}(\dot{\neg}(y))))) \\
                                    & = \dot{\vee}(\dot{\neg}(\dot{\vee}(\dot{\neg}(x), \dot{\neg}(y))), \dot{\neg}(\dot{\vee}(x, y)))
    \end{align*}
    Where we've used the contentious\footnote{Okay, maybe not that contentious, but some people don't like it, right?} idea that negation is an involution.

    \ref{part:1c}
    When attempting to prove $p \implies q$ we can sometimes try and prove $\neg q \implies \neg p$.
    This can be encoded into the following tautology using $\dot{\leftrightarrow}$, $\dot{\neg}$, and $\dot{\rightarrow}$.
    \begin{equation*}
        \dot{\leftrightarrow}(\dot{\rightarrow}(x, y), \dot{\rightarrow}(\dot{\neg}(y), \dot{\neg}(x)))
    \end{equation*}
    To see this is indeed a tautology we can use our expressions above to simplify.
    \begin{multline*}
        \dot{\leftrightarrow}(\dot{\rightarrow}(x, y), \dot{\rightarrow}(\dot{\neg}(y), \dot{\neg}(x))) \\
        = \dot{\vee}(\dot{\neg}(\dot{\vee}(\dot{\neg}(\dot{\vee}(\dot{\neg}(x), y)), \dot{\neg}(\dot{\vee}(y, \dot{\neg}(x))))), \dot{\neg}(\dot{\vee}(\dot{\vee}(\dot{\neg}(x), y), \dot{\vee}(y, \dot{\neg}(x)))))
    \end{multline*}
    Now define $A \coloneqq \dot{\vee}(\dot{\neg}(x), y) = \dot{\vee}(y, \dot{\neg}(x))$ where the last equality holds by the symmetry of or.
    We now have
    \begin{align*}
        \dot{\leftrightarrow}(\dot{\rightarrow}(x, y), \dot{\rightarrow}(\dot{\neg}(y), \dot{\neg}(x))) & = \dot{\vee}(\dot{\neg}(\dot{\vee}(\dot{\neg}(A), \dot{\neg}(A))), \dot{\neg}(\dot{\vee}(A, A))) \\
                                                                                                        & = \dot{\vee}(\dot{\neg}(\dot{\neg}(A)), \dot{\neg}(A))                                           \\
                                                                                                        & = \dot{\vee}(A, \dot{\neg}(A)).
    \end{align*}
    We've now reached the infamous law of excluded middle which we take to be true, always.
    Thus we have a tautology, and hence proof by contraposition is a valid proof (if you take LEM).
\end{solution}


\begin{problem}
Let $S$ be the first-order alphabet $\qty{R_1, R_2, f}$ in which $R_1$ and $R_2$ are unary relation symbols and $f$ is a binary function symbol.
Suppose $\mathcal{A} = (A, \mathfrak{a})$ is an $S$-structure and that $\mathcal{J} = (\mathcal{A}, \beta)$ is an $S$-interpretation and ${\bf \Phi}$ is the set of formulas $\{\phi_1, \phi_2, \phi_3, \phi_4\}$ with
\begin{align*}
    \phi_1 & = \exists v_0 \exists v_1 ((R_1 v_0 \wedge R_1 v_1)\wedge \neg v_0\equiv v_1)                                                            \\
    \phi_2 & = \exists v_0 \exists v_1 ((R_2 v_0 \wedge R_2 v_1)\wedge \neg v_0\equiv v_1)                                                            \\
    \phi_3 & = \forall v_0 \exists v_1 \exists v_2 ((R_1 v_1\wedge R_2 v_2)\wedge fv_1v_2\equiv v_0)                                                  \\
    \phi_4 & = \forall v_1 \forall v_2 \forall v_3 \forall v_4 (((((R_1v_1\wedge R_1 v_2)\wedge R_2 v_3)\wedge R_2 v_4)\wedge fv_1v_3 \equiv fv_2v_4) \\
           & \qquad \rightarrow (v_1\equiv v_2 \wedge v_3\equiv v_4)).
\end{align*}
\begin{parts}
    \part Show that if $\mathcal{J}\vDash {\bf \Phi}$ and $|A|$ is finite then $|A|$ is a composite number (i.e., not prime and not $1$).
    \label{part:2a}
    \part Show that if $|A| < \infty$ is composite then there is an $S$-interpretation $\mathcal{J}$ with universe $A$ such that $\mathcal{J}\vDash {\bf \Phi}$.
    \label{part:2b}
    % \part (Bonus +1): Find a first-order alphabet $S$ and a finite set ${\bf \Psi}$ of $S$-formulas such that there exists an $S$-interpretation $\mathcal{J} = (A, \mathfrak{a}, \beta)$ with $|A| < \infty$ such that $\mathcal{J}\vDash {\bf \Psi}$ if and only if $|A|$ is prime.
    % \label{part:2c}
\end{parts}
\end{problem}

\begin{solution}
    \ref{part:2a}
    We first interpret each equation $\phi_i$ given we know $\abs{A}$ is finite.
    \begin{equation*}
        \phi_1 = \exists v_0, v_1 \in A \quad R_1^Av_0 \text{ and } R_1^Av_1 \text{ and } v_1 \neq v_0
    \end{equation*}
    Thus $\phi_1$ is telling us there are \emph{at least} two distinct elements that satisfy the relation $R_1^A$.
    Put differently, since $R_1^A\subseteq A$, we know $\abs{R_1^A} \geq 2$.
    Similarly for $\phi_2$ we have
    \begin{equation*}
        \phi_2 = \exists v_0, v_1 \in A \quad R_2^Av_0 \text{ and } R_2^Av_1 \text{ and } v_1 \neq v_0
    \end{equation*}
    where again this is telling us that $\abs{R_2^A} \geq 2$, or that there are at least two elements that satisfy $R_2^A$.
    Moving on for $\phi_3$ we have
    \begin{equation*}
        \phi_3 = \forall v_0 \in A \quad \exists v_1, v_2 \in A \quad R_1^A v_2 \text{ and } R_2^A v_2 \text{ and } f^A(v_1, v_2) = v_0.
    \end{equation*}
    This formula tells us that $f^A: A\times A \to A$ when restricted to $R_1^A \times R_2^A$ is a surjection onto $A$.
    That is $\eval{f^A}_{R_1^A\times R_2^A}$ is a surjection.
    Lastly we have
    \begin{multline*}
        \phi_4 = \forall v_1, v_2, v_3, v_4 \in A \quad v_1, v_2 \in R_1^A \text{ and } v_3, v_4\in R_2^A \text{ and } f^A(v_1, v_3) = f^A(v_2, v_4) \\
        \text{ imply } v_1 = v_2 \text{ and } v_3 = v_4.
    \end{multline*}
    This is exactly the condition for $\eval{f^A}_{R_1^A\times R_2^A}$ being injective.
    This fact, together with the previous one implies $f^A$ is a bijection, and hence the domain and range have the same cardinality: $\abs{R_1^A\times R_2^A} = \abs{A}$.
    A basic property about the cardinality of finite sites $B$ and $C$ is $\abs{B\times C} = \abs{B}\cdot \abs{C}$.\footnote{Maybe this holds for larger cardinals too, but I'm not familiar enough to say.}
    We then have $\abs{A} = \abs{R_1^A}\cdot \abs{R_2^A}$ and hence a composite number.

    \ref{part:2b}
    Now take $\abs{A} = n\cdot m$ with $n, m\in \mathbb{N}$ such that $n,m \neq 1$.
    Arrange the elements of $A$ in a grid as follows (in arbitrary order) and give each element a name based on it's grid position.
    \begin{equation*}
        \begin{NiceMatrix}
            a_{11} & a_{12} & \Cdots & a_{1m} \\
            a_{21} & a_{22} &        & \Vdots \\
            \Vdots &        & \Ddots &        \\
            a_{n1} & \Cdots &        & a_{nm}
        \end{NiceMatrix}
    \end{equation*}
    Then form the following two (intersecting) subsets of $A$.
    \begin{equation*}
        \begin{NiceMatrix}
            \CodeBefore
            \rectanglecolor{red!15}{2-2}{2-5}
            \rectanglecolor{blue!15}{2-2}{5-2}
            \cellcolor{purple!35}{2-2}
            \Body
                           & \Block{1-4}{B} &        &                 \\
            \Block{4-1}{C} & a_{11}         & a_{12} & \Cdots & a_{1m} \\
                           & a_{21}         & a_{22} &        & \Vdots \\
                           & \Vdots         &        & \Ddots &        \\
                           & a_{n1}         & \Cdots &        & a_{nm}
        \end{NiceMatrix}
    \end{equation*}
    These can be written $B = \qty{a_{1j}\in A : 1 \leq j \leq m}$, and $C = \qty{a_{i1}\in A : 1 \leq i \leq n}$.
    Since $n, m > 1$ each one of these subsets must have more than one element in each.
    That is $\abs{B} > 1$ and $\abs{C} > 1$.
    We can then take $B$ and $C$ to be relations on $A$ and hence $\phi_1$ and $\phi_2$ are automatically satisfied with $R_1^A = B$ and $R_2^A = C$.
    Next we define $f^A: B\times C \to A$ as
    \begin{equation*}
        f(b, c) = f(a_{1j}, a_{i1}) \coloneqq a_{ij}.
    \end{equation*}
    This is clearly a bijection from $B\times C$ to $A$, and hence $\phi_3$ and $\phi_4$ are also satisfied.
    Hence we have constructed an $\qty{R_1, R_2, f}$-structure where $\mathcal{J}\vDash {\bf \Phi}$.
    % \ref{part:2c}
\end{solution}

\begin{problem}
In the following questions, let $S_\text{gr} = (1, \cdot, i)$ and we only consider $S_\text{gr}$-interpretations $\mathcal{J} = (A, \mathfrak{a}, \beta)$ in which $A$ is a group, $1^A$ is the identity of $A$, $\cdot^A$ is multiplication, and $i^A$ is the inverse map.
For the following formulas $\phi$ give an informal statement of what the formula is saying and say whether $\mathcal{J} \vDash \phi$ for every such interpretation $\mathcal{J}$, for at least one such interpretation but not every such interpretation, or for no such interpretations.
% (For example, if $\phi = \forall v_0 \exists v_1 \cdot v_0 v_1\equiv 1$ then it is saying for every $v_0\in A$ there is some $v_1\in A$ such that $v_0\cdot v_1 = 1$ and this is true in every group, so $\mathcal{J}\vDash \phi$ for every such interpretation.)
\begin{parts}
    \part $\forall v_0 \forall v_1 \forall v_2 \cdot \cdot v_0 v_1v_2\equiv \cdot v_0\cdot v_1v_2$
    \label{part:3a}
    \part $\forall v_0 \forall v_1 \cdot \cdot v_0 v_1 v_1 \equiv \cdot \cdot v_1 v_0 v_1$
    \label{part:3b}
    \part $\exists v_0 ((\neg v_0\equiv 1) \wedge \cdot v_0 v_0\equiv 1)$
    \label{part:3c}
    \part $\exists v_0 \forall v_1 v_2 \equiv \cdot v_0 v_1$
    \label{part:3d}
    \part $\exists v_0 \exists v_1 v_2 \equiv \cdot v_0 v_1$
    \label{part:3e}
    \part $\exists v_0 \exists v_1 (\neg v_0\equiv v_1 \vee \forall v_3 v_3\equiv 1)$
    \label{part:3f}
    \part $\exists v_3 (\cdot v_3 v_2 \equiv 1 \wedge \neg v_3 \equiv iv_2)$
    \label{part:3g}
    \part $\forall v_0 ((\cdot v_0 v_0\equiv 1 \wedge \cdot \cdot v_0v_0v_0\equiv 1)\rightarrow v_0\equiv 1)$
    \label{part:3h}
\end{parts}
\end{problem}

\begin{solution}
    First, here is a summary of my solutions, with more details expounded in each part.
    \begin{table}[h]
        \centering\begin{tabular}{cc}
            Part          & Holds for \_\_\_ interpretations \\ \toprule
            \ref{part:3a} & all                              \\
            \ref{part:3b} & some                             \\
            \ref{part:3c} & some                             \\
            \ref{part:3d} & some                             \\
            \ref{part:3e} & all                              \\
            \ref{part:3f} & all                              \\
            \ref{part:3g} & no                               \\
            \ref{part:3h} & all                              \\
        \end{tabular}
    \end{table}

    \ref{part:3a}
    Written in infix notation this equation reads
    \begin{equation*}
        \forall v_0, v_1, v_2 \in A \quad (v_0 \cdot v_1) \cdot v_2 = v_0 \cdot (v_1 \cdot v_2)
    \end{equation*}
    which clearly shows that the multiplication in the group is associative.
    This facts holds for every such interpretation $\mathcal{J}$ by the definition of group multiplication.

    \ref{part:3b}
    Again, writing in infix notation we have
    \begin{equation*}
        \forall v_0, v_1 \in A \quad (v_0 \cdot v_1) \cdot v_1 = (v_1 \cdot v_0) \cdot v_1
    \end{equation*}
    Multiplying on thr right by $v_1^{-1}$ we have $v_0\cdot v_1 = v_1 \cdot v_0$ which is clearly only true in Abelian groups.
    The existence of non-Abelian groups (e.g. the permutation group) shows this formula holds in at least one such interpretation $\mathcal{J}$.

    \ref{part:3c}
    In everyday math notation we might write
    \begin{equation*}
        \exists v_0 \in A \quad v_0 \neq 1^A \text{ and } v_0^2 = 1^A.
    \end{equation*}
    This formula holds for at least one interpretation, but not necessarily all.
    To see this take the trivial group $G = (\qty{e}, \cdot)$ where the only multiplication rule we have is $e \cdot e = e$.
    Since $G$ is a group where this formula does not hold it cannot hold in all interpretations.
    That said the group $H = (\qty{1, -1}, \cdot_\mathbb{R})$ is a group where this formula holds with $v_0 = -1$.

    \ref{part:3d}
    Here we have
    \begin{equation*}
        \exists v_0\in A \; \forall v_1 \in A \quad v_2 = v_0 \cdot v_1.
    \end{equation*}
    This can be found to hold, for example, in the trivial group $G = (\qty{e}, \cdot)$.
    We then have $v_0, v_1, v_2 = e$ and the equation reads $e = e\cdot e$ which clearly holds.
    That said this equation does not hold in all interpretations.
    To see this take the group $H = (\qty{1, -1}, \cdot_\mathbb{R})$.
    Now take $v_2 = 1$ and the equation says either $1 = 1 \cdot -1 \wedge 1 = 1\cdot 1$ \emph{or} $1 = -1\cdot -1 \wedge 1 = -1\cdot 1$ which clearly neither hold.

    \ref{part:3e}
    Here we have
    \begin{equation*}
        \exists v_0, v_1 \in A \quad v_2 = v_0 \cdot v_1.
    \end{equation*}
    This can be found to hold in all interpretations by taking $v_0 = v_2$ and $v_1 = 1^A$.

    \ref{part:3f}
    \begin{equation*}
        \exists v_0, v_1 \in A \quad v_0 \neq v_1 \text{ or } \forall v_3 \in A \quad v_3 = 1^A
    \end{equation*}
    This formula says you either have
    \begin{itemize}[topsep=6pt,itemsep=0pt,partopsep=4pt,parsep=2pt]
        \item two distinct elements in the group, or
        \item all elements in your group are the identity element.
    \end{itemize}
    And this holds for all interpretations $\mathcal{J}$.

    \ref{part:3g}
    Here we use the notation $g^{-1}$ instead of $i^A(g)$ for familiarity.
    \begin{equation*}
        \exists v_3 \in A \quad v_3\cdot v_2 = 1^A \text{ and } v_3 \neq v_2^{-1}
    \end{equation*}
    This formula does not hold in any such interpretation $\mathcal{J}$ because of the uniqueness of (left and right) inverses in groups.

    \ref{part:3h}
    Finally, in modern notation we have:
    \begin{equation*}
        \forall v_0 \in A \quad v_0^2 = 1^A \text{ and } v_0^3 = 1^A \implies v_0 = 1^A.
    \end{equation*}
    This formula holds for all such interpretations $\mathcal{J}$ as follows.
    We can write $v_0^3 = v_0\cdot v_0^2 = v_0 \cdot 1^A = v_0 = 1^A$.
    This manipulation does not use anything about a particular group and so this formula holds for all such interpretations $\mathcal{J}$.
\end{solution}

\end{document}