\documentclass[boxes,pages,color=CornflowerBlue]{homework}

\hypersetup{
    colorlinks=true,
    linkcolor=Bittersweet
}

\usepackage{booktabs}

\name{Nate Stemen}
\studentid{20906566}
\email{nate@stemen.email}
\term{Winter 2021}
\course{Logic and Computability}
\courseid{PMATH 632}
\hwnum{2}
\duedate{Jan 28, 2022 11:59PM}
\hwname{Assignment}

\newcommand{\true}{\ensuremath{\mathsf{T}}}
\newcommand{\false}{\ensuremath{\mathsf{F}}}

\begin{document}

\begin{problem}[Truth functions]
\begin{parts}
    \part Prove DeMorgan's law:
    \begin{equation*}
        \dot{\wedge}(x, y) = \dot{\neg}(\dot{\vee}(\dot{\neg}(x), \dot{\neg}(y)))
    \end{equation*}
    for all $x,y \in \qty{\true, \false}$.
    \label{part:1a}
    \part Show that one can similarly express $\dot{\rightarrow}(x,y)$ and $\dot{\leftrightarrow}(x,y)$ in terms of the functions $\dot{\neg}$ and $\dot{\vee}$.
    \label{part:1b}
    \part Express contraposition as a statement about $\dot{\rightarrow}$ and $\dot{\neg}$ and $\dot{\leftrightarrow}$.
    \label{part:1c}
\end{parts}
\end{problem}

\begin{solution}
    \ref{part:1a}
    DeMorgan's law can be seen by building up the right hand side of the equality from it's components.
    \begin{center}
        \begin{tabular}{ccccccc}
            $x$    & $y$    & $\dot{\neg}(x)$ & $\dot{\neg}(y)$ & $\dot{\vee}(\dot{\neg}(x), \dot{\neg}(y))$ & $\dot{\neg}(\dot{\vee}(\dot{\neg}(x), \dot{\neg}(y)))$ & $\dot{\wedge}(x, y)$ \\ \toprule
            \true  & \true  & \false          & \false          & \false                                     & \true                                                  & \true                \\
            \true  & \false & \false          & \true           & \true                                      & \false                                                 & \false               \\
            \false & \true  & \true           & \false          & \true                                      & \false                                                 & \false               \\
            \false & \false & \true           & \true           & \true                                      & \false                                                 & \false
        \end{tabular}
    \end{center}

    \ref{part:1b}
    We give the following two characterizations by truth tables for implication and the biconditional.
    \begin{center}
        \begin{tabular}{ccccc}
            $x$    & $y$    & $\dot{\neg}(x)$ & $\dot{\vee}(\dot{\neg}(x), y)$ & $\dot{\rightarrow}(x, y)$ \\ \toprule
            \true  & \true  & \false          & \true                          & \true                     \\
            \true  & \false & \false          & \false                         & \false                    \\
            \false & \true  & \true           & \true                          & \true                     \\
            \false & \false & \true           & \true                          & \true
        \end{tabular}
    \end{center}
    \begin{center}
        \begin{tabular}{cccccc}
            $x$    & $y$    & $\dot{\wedge}(x, y)$ & $\dot{\wedge}(\dot{\neg}(x), \dot{\neg}(y))$ & $\dot{\vee}(\dot{\wedge}(x, y), \dot{\wedge}(\dot{\neg}(x), \dot{\neg}(y)))$ & $\dot{\leftrightarrow}(x, y)$ \\ \toprule
            \true  & \true  & \true                & \false                                       & \true                                                                        & \true                         \\
            \true  & \false & \false               & \false                                       & \false                                                                       & \false                        \\
            \false & \true  & \false               & \false                                       & \false                                                                       & \false                        \\
            \false & \false & \false               & \true                                        & \true                                                                        & \true
        \end{tabular}
    \end{center}
    Now since we must use only negation and disjunction we can use DeMorgan's law to write the following.
    \begin{align*}
        \dot{\leftrightarrow}(x, y) & = \dot{\vee}(\dot{\wedge}(x, y), \dot{\wedge}(\dot{\neg}(x), \dot{\neg}(y)))                                                                     \\
                                    & = \dot{\vee}(\dot{\neg}(\dot{\vee}(\dot{\neg}(x), \dot{\neg}(y))), \dot{\neg}(\dot{\vee}(\dot{\neg}(\dot{\neg}(x)), \dot{\neg}(\dot{\neg}(y))))) \\
                                    & = \dot{\vee}(\dot{\neg}(\dot{\vee}(\dot{\neg}(x), \dot{\neg}(y))), \dot{\neg}(\dot{\vee}(x, y)))
    \end{align*}
    Where we've used the contentious\footnote{Okay, maybe not that contentious, but some people don't like it, right?} idea that negation is an involution.

    \ref{part:1c}
    \begin{equation*}
        \dot{\leftrightarrow}(\dot{\rightarrow}(x, y), \dot{\rightarrow}(\dot{\neg}(y), \dot{\neg}(x)))
    \end{equation*}
\end{solution}

\begin{problem}
Let $S$ be the first-order alphabet $\qty{R_1, R_2, f}$ in which $R_1$ and $R_2$ are unary relation symbols and $f$ is a binary function symbol.
Suppose $\mathcal{A} = (A, \mathfrak{a})$ is an $S$-structure and that $\mathcal{J} = (\mathcal{A}, \beta)$ is an $S$-interpretation and ${\bf \Phi}$ is the set of formulas $\{\phi_1, \ldots, \phi_4\}$ with
\begin{align*}
    \phi_1 & = \exists v_0 \exists v_1 ((R_1 v_0 \wedge R_1 v_1)\wedge \neg v_0\equiv v_1)                                                            \\
    \phi_2 & = \exists v_0 \exists v_1 ((R_2 v_0 \wedge R_2 v_1)\wedge \neg v_0\equiv v_1)                                                            \\
    \phi_3 & = \forall v_0 \exists v_1 \exists v_2 ((R_1 v_1\wedge R_2 v_2)\wedge fv_1v_2\equiv v_0)                                                  \\
    \phi_4 & = \forall v_1 \forall v_2 \forall v_3 \forall v_4 (((((R_1v_1\wedge R_1 v_2)\wedge R_2 v_3)\wedge R_2 v_4)\wedge fv_1v_3 \equiv fv_2v_4) \\
           & \qquad \rightarrow (v_1\equiv v_2 \wedge v_3\equiv v_4)).
\end{align*}
\begin{parts}
    \part Show that if $\mathcal{J}\vDash {\bf \Phi}$ and $|A|$ is finite then $|A|$ is a composite number (i.e., not prime and not $1$).
    \label{part:2a}
    \part Show that if $|A|<\infty$ is composite then there is an $S$-interpretation $\mathcal{J}$ with universe $A$ such that $\mathcal{J}\vDash {\bf \Phi}$.
    \label{part:2b}
    \part (Bonus +1): Find a first-order alphabet $S$ and a finite set ${\bf \Psi}$ of $S$-formulas such that there exists an $S$-interpretation $\mathcal{J}=(A,\mathfrak{a},\beta)$ with $|A|<\infty$ such that $\mathcal{J}\vDash\Psi$ if and only if $|A|$ is prime.
    \label{part:2c}
\end{parts}
\end{problem}

\begin{solution}
    \ref{part:2a}
    \ref{part:2b}
    \ref{part:2c}
\end{solution}

\begin{problem}
In the following questions, let $S_\text{gr} = (1, \cdot, i)$ and we only consider $S_\text{gr}$-interpretations $\mathcal{J} = (A, \mathfrak{a}, \beta)$ in which $A$ is a group, $1^A$ is the identity of $A$, $\cdot^A$ is multiplication, and $i^A$ is the inverse map.
For the following formulas $\phi$ give an informal statement of what the formula is saying and say whether $\mathcal{J} \vDash \phi$ for every such interpretation $\mathcal{J}$, for at least one such interpretation but not every such interpretation, or for no such interpretations.
(For example, if $\phi = \forall v_0 \exists v_1 \cdot v_0 v_1\equiv 1$ then it is saying for every $v_0\in A$ there is some $v_1\in A$ such that $v_0\cdot v_1 = 1$ and this is true in every group, so $\mathcal{J}\vDash \phi$ for every such interpretation.)
\begin{parts}
    \part $\forall v_0 \forall v_1 \forall v_2 \cdot \cdot v_0 v_1v_2\equiv \cdot v_0\cdot v_1v_2$
    \label{part:3a}
    \part $\forall v_0 \forall v_1 \cdot \cdot v_0 v_1 v_1 \equiv \cdot \cdot v_1 v_0 v_1$
    \label{part:3b}
    \part $\exists v_0 ((\neg v_0\equiv 1) \wedge \cdot v_0 v_0\equiv 1)$
    \label{part:3c}
    \part $\exists v_0 \forall v_1 v_2 \equiv \cdot v_0 v_1$
    \label{part:3d}
    \part $\exists v_0 \exists v_1 v_2 \equiv \cdot v_0 v_1$
    \label{part:3e}
    \part $\exists v_0 \exists v_1 (\neg v_0\equiv v_1 \vee \forall v_3 v_3\equiv 1)$.
    \label{part:3f}
    \part $\exists v_3 (\cdot v_3 v_2 \equiv 1 \wedge \neg v_3 \equiv iv_2)$
    \label{part:3g}
    \part $\forall v_0 ((\cdot v_0 v_0\equiv 1 \wedge \cdot \cdot v_0v_0v_0\equiv 1)\rightarrow v_0\equiv 1)$
    \label{part:3h}
\end{parts}
\end{problem}

\begin{solution}
    \ref{part:3a}
    Written in infix notation this equation reads
    \begin{equation*}
        \forall v_0, v_1, v_2 \in A \quad (v_0 \cdot v_1) \cdot v_2 = v_0 \cdot (v_1 \cdot v_2)
    \end{equation*}
    which clearly shows that the multiplication in the group is associative.
    This facts holds for every such interpretation $\mathcal{J}$ by the definition of group multiplication.

    \ref{part:3b}
    Again, writing in infix notation we have
    \begin{equation*}
        \forall v_0, v_1 \in A \quad (v_0 \cdot v_1) \cdot v_1 = (v_1 \cdot v_0) \cdot v_1
    \end{equation*}

    \ref{part:3c}
    \begin{equation*}
        \exists v_0 \in A \quad v_0 \neq 1^A \text{ and } v_0^2 = 1^A
    \end{equation*}

    \ref{part:3d}
    \begin{equation*}
        \exists v_0\in A \; \forall v_1 \in A \quad v_2 = v_0 \cdot v_1
    \end{equation*}

    \ref{part:3e}
    \begin{equation*}
        \exists v_0, v_1 \in A \quad v_2 = v_0 \cdot v_1
    \end{equation*}

    \ref{part:3f}
    \begin{equation*}
        \exists v_0, v_1 \in A \quad v_0 \neq v_1 \text{ or } \forall v_3 \in A \quad v_3 = 1^A
    \end{equation*}

    \ref{part:3g}
    Here we use the notation $g^{-1}$ instead of $i^A(g)$ for familiarity.
    \begin{equation*}
        \exists v_3 \in A \quad v_3\cdot v_2 = 1^A \text{ and } v_3 \neq v_2^{-1}
    \end{equation*}

    \ref{part:3h}
    \begin{equation*}
        \forall v_0 \in A \quad v_0^2 = 1^A \text{ and } v_0^3 = 1^A \implies v_0 = 1^A
    \end{equation*}
\end{solution}

\end{document}